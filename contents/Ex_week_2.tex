\section{Modules}
\begin{exercise}[Đồng dư modulo một module con; tính tương thích phép toán]
Cho $M$ là một $R$-module và $S\le_R M$ là một module con. Định nghĩa
\[
u \equiv v \pmod S \quad \Longleftrightarrow \quad u-v\in S.
\]
\begin{enumerate}
  \item[(a)] Chứng minh rằng quan hệ $\equiv \pmod S$ là một quan hệ tương đương trên $M$ (phản xạ, đối xứng, bắc cầu).
  \item[(b)] Chứng minh tính tương thích với các phép toán của $M$:
  \[
  \big(u_1\equiv v_1 \pmod S,\; u_2\equiv v_2 \pmod S\big)\ \Longrightarrow\ 
  u_1+u_2 \equiv v_1+v_2 \pmod S,
  \]
  và nếu $u\equiv v \pmod S$ và $r\in R$ thì $ru\equiv rv \pmod S$.
  \item[(c)] Các phép toán
  \[
  (u+S)+(v+S):=(u+v)+S,\qquad r\cdot(u+S):=(ru)+S
  \]
  được xác định tốt (không phụ thuộc vào đại diện) trên tập các lớp tương đương $M/S$, nhờ đó $M/S$ trở thành một $R$-module (gọi là \emph{module thương}).
\end{enumerate}

\begin{proof}
\textbf{(a) Quan hệ tương đương.}
\begin{itemize}
  \item \emph{Phản xạ:} Với mọi $u\in M$, ta có $u-u=0\in S$. Do đó $u\equiv u\pmod S$.
  \item \emph{Đối xứng:} Nếu $u\equiv v\pmod S$ thì $u-v\in S$. Do đó $v - u = -(u-v)\in S$, tức $v\equiv u\pmod S$.
  \item \emph{Bắc cầu:} Nếu $u\equiv v\pmod S$ và $v\equiv w\pmod S$ thì $u - w = (u-v)+(v-w)\in S$, nên $u\equiv w\pmod S$.
\end{itemize}

\noindent\textbf{(b) Tương thích với phép cộng và nhân vô hướng.}
\begin{itemize}
  \item Nếu $u_1\equiv v_1\pmod S$ và $u_2\equiv v_2\pmod S$ thì $u_1-v_1\in S$ và $u_2-v_2\in S$. Khi đó
  \[
  (u_1+u_2)-(v_1+v_2)=(u_1-v_1)+(u_2-v_2)\in S \iff u_1+u_2\equiv v_1+v_2\pmod S,
  \]
  
  \item Nếu $u\equiv v\pmod S$ và $r\in R$ thì $u-v\in S$. Vì $S$ đóng dưới phép nhân vô hướng, $r(u-v)=ru-rv\in S$, nên $ru\equiv rv\pmod S$.
\end{itemize}

\textbf{Các phép toán trên $M/S$.}

Giả sử $[u] = [u'],~[v]=[v'], r\in R$. Khi đó $u\equiv u'\pmod S,~v\equiv v'\pmod S$. Dẫn đến
\[
(u+v) - (u'+v') = (u-u') + (v-v') \in S \iff (u+v)+S=(u'+v')+S \iff [u+v] = [u'+v'],
\]
\[
ru-rv = r(u-v) \in S \iff ru+S=rv+S \iff [ru] =  [rv],
\]

Vậy các phép toán được định nghĩa tốt, và $M/S$ có cấu trúc của một $R$-module.
\end{proof}
\end{exercise}
\begin{exercise}[Mệnh đề 5]
    
\end{exercise}
\begin{proof}
    Với mọi $u,v \in M,~r\in R$ ta có
    \[\pi(u+v) = [u+v]=[u]+[v]=\pi(u) + \pi(v),\quad \pi(ru)=[ru]=r[u]\]
    Dẫn đến $\pi$ là một đồng cấu $R-$module. Hơn nữa, $\pi$ là một toàn cấu vì với mỗi $[v] \in M/S$, tồn tại $v \in M$ sao cho $\pi(v) = [v]$. Và $\ker(\pi)=\{v \in M:[0]= \pi(v)=[v]\} = \{v\in M: v\in S\} = S$    
\end{proof}
\begin{exercise}[Mệnh đề 6: Bù tuyến tính và thương module]
Cho $M$ là một $R$-module và $S\le M$ là module con. Giả sử tồn tại $T\le M$ sao cho 
\[
M = S \oplus T.
\]
Khi đó, hạn chế của phép chiếu chính tắc $\pi : M \to M/S$ cho ta đẳng cấu
\[
\pi|_T : T \to M/S.
\]

\begin{proof}
Với $t\in T$, ta có.  
\[\pi|_T(t)=[0] \iff t+S = 0 + S \iff t \in S \cap T = \{0\} \iff t = 0\]
do đó $\pi|_T$ là đơn cấu.  

Với $m+S\in M/S$, tồn tại duy nhất $s\in S, t\in T$ để $m=s+t$, khi đó $\pi|_T(t)=t+S=m+S$, nên $\pi|_T$ là toàn cấu.  
Vậy $\pi|_T$ là đẳng cấu.
\end{proof}
\end{exercise}

\begin{exercise}[Mệnh đề 7: Tính chất phổ dụng của module thương]
Cho $f : M \to N$ là một đồng cấu các $R$-module, và $S \subseteq \ker(f)$ là một module con của $M$.
Khi đó, tồn tại duy nhất một đồng cấu tuyến tính
\[
\bar{f} : M/S \to N
\]
sao cho biểu đồ
\[
\begin{tikzcd}
M \arrow{r}{f} \arrow{d}[swap]{\pi} & N \\
M/S \arrow{ur}[swap]{\bar{f}} &
\end{tikzcd}
\]
giao hoán, tức là $f = \bar{f} \circ \pi$.

Hơn nữa,
\[
\operatorname{Im}(\bar{f}) = \operatorname{Im}(f), \qquad 
\ker(\bar{f}) = \ker(f)/S.
\]

\begin{proof}
\begin{itemize}
    \item Định nghĩa 
\[\bar{f}([m]) = f(m)~ \forall m \in M\]
$\bar{f}$ được định nghĩa tốt vì nếu $[m]=[m']$ thì $m - m' \in S \subseteq \ker(f)$ nên $f(m) = f(m')$.


    \item $\bar{f}$ là một đồng cấu.
    
    Với $r \in R$, $m_1, m_2 \in M$ ta có:
\[
\bar{f}([m_1] + [m_2]) = \bar{f}([m_1 + m_2]) = f(m_1 + m_2) = f(m_1) + f(m_2)
= \bar{f}([m_1]) + \bar{f}([m_2]).
\]
và
\[
\bar{f}(r[m]) = \bar{f}([rm]) = f(rm) = r f(m) = r \bar{f}([m]).
\]
Rõ ràng $f = \bar{f} \circ \pi$.

    \item \textbf{Tính duy nhất.} 
    
    Nếu $g : M/S \to N$ là đồng cấu sao cho $f = g \circ \pi$, thì với mọi $m \in M$, 
\[
g([m]) = g(\pi(m)) = f(m),
\]
nên $g = \bar{f}$.

    \item \textbf{Ảnh và hạt nhân.}
\[
\operatorname{Im}(\bar{f}) = \{\, \bar{f}([m])\mid m \in M \,\} = \{\, f(m) \mid m \in M \,\}=\operatorname{Im}(f)
\]
và
\[
\ker(\bar{f}) = \{\, [m] \in M/S \mid \bar{f}([m]) = 0 \,\}
= \{\, [m] \mid f(m) = 0 \,\} = \{[m]\mid m \in \ker(f)\}=\ker(f)/S.
\]
\end{itemize}
\end{proof}
\end{exercise}

\begin{exercise}[Mệnh đề 8: Định lý đẳng cấu thứ nhất cho $R$-module]
Cho $f:M\to N$ là đồng cấu $R$-module. Gọi $\pi:M\to M/\ker(f)$ là phép chiếu chính tắc.
Khi đó ánh xạ
\[
\bar f: M/\ker(f)\longrightarrow \im(f),\qquad \bar f([m])=f(m)
\]
là một đẳng cấu và biểu đồ sau giao hoán:
\[
\begin{tikzcd}
M \arrow{r}{f} \arrow{d}[swap]{\pi} & N \\
M/\ker(f) \arrow{r}[swap]{\bar f}[above]{\cong} & \im(f) \arrow[hook]{u}
\end{tikzcd}
\]

\begin{proof}
\begin{itemize}
    \item $\bar{f}$ xác định tốt, vì nếu $[m]=[m']$ thì $m-m'\in\ker(f)$ nên $f(m)=f(m')$. Hơn nữa $f$ rõ ràng là một đồng cấu.

    \item $\bar{f}$ là toàn cấu vì $\forall y = f(x)\in \im(f)$, tồn tại $x$ sao cho $y=f(m)=\bar f([m])$.

    \item $\bar{f}$ đơn cấu vì 
    \[\forall [m] \in \ker(\bar{f}) \iff \bar f([m])=0 \iff f(m)=0 \iff m\in\ker(f) \iff [m]=[0]\]
    Vậy $\ker(\bar f)=0$.

Vậy $\bar f$ là đẳng cấu $M/\ker(f)\xrightarrow{\;\cong\;}\im(f)$. Hơn nữa $f=(\iota\circ\bar f)\circ \pi$
với $\iota:\im(f)\hookrightarrow N$ là đơn cấu nhúng. Do đó biểu đồ đã cho là giao hoán.
\end{itemize}
\end{proof}
\end{exercise}

\begin{exercise}
    Cho $M$ là một $R$-module và $S,T$ là các module con của $M$. Khi đó
    \[\dfrac{S+T}{S} \cong \dfrac{T}{S \cap T}.\]
\end{exercise}
\begin{proof}
    Định nghĩa
    \[f: T \to \dfrac{S+T}{S},\quad t\mapsto [t]=t+S.\]
    \begin{itemize}
        \item $f$ là một đồng cấu vì 
        \[f(t_1+t_2) = [t_1+t_2]=[t_1]+[t_2]=f(t_1)+f(t_2)\quad \forall t_1,t_2 \in T\]
        \[f(rt) = [rt]=r[t]=rf(t)\quad \forall t\in T, r\in R\]

        \item $f$ toàn ánh vì $\forall [s+t]  \in \dfrac{S+T}{S}$, tồn tại $t \in T$ sao cho $f(t) = [t]=[s+t]$, vì $s \in S$.
        \item $\ker(f) = \{t \in T \mid [0]=f(t)=[t]\}=\{ t \in T \mid t\in S\} = S\cap T$.

        \item Theo định lý đẳng cấu thứ nhất ta được
        \[\dfrac{S+T}{S} = \im(f) \cong T/\ker(f) = \dfrac{T}{S\cap T}.\]
    \end{itemize}
\end{proof}

\begin{exercise}[Mệnh đề 10: Định lý đẳng cấu thứ hai cho $R$-module]
Cho $M$ là $R$-module và $S\subseteq T\subseteq M$ là các module con. Khi đó
\[
\frac{M/S}{\,T/S\,}\ \cong\ M/T .
\]

\begin{proof}
Định nghĩa
\[
\varphi:\ M/S \longrightarrow M/T,\qquad \varphi(m+S)=m+T.
\]

\begin{itemize}
    \item $\varphi$ được định nghĩa tốt vì nếu $m+S=m'+S$ thì $m-m'\in S\subseteq T$, do đó $m+T=m'+T$.

    \item Rõ ràng $\varphi$ là đồng cấu và với mọi $m+T\in M/T$, $\varphi(m+S)=m+T$, nên $\varphi$ là toàn cấu.

    \item $\ker(\varphi)=\{\,m+S:\ m\in T\,\}=T/S$.

Áp dụng Định lý đẳng cấu thứ nhất cho $\varphi$ ta được
\[
(M/S)/(T/S)=(M/S)/\ker(\varphi)\ \cong\ \im(\varphi)=M/T,
\]
\end{itemize}
\end{proof}
\end{exercise}

\begin{exercise}[Bài tập 6 (chi tiết): Đẳng cấu phạm trù $\mathbf{Mod}_{\mathbb{Z}} \cong \mathbf{Ab}$]
Chứng minh rằng phạm trù $\mathbf{Mod}_{\mathbb{Z}}$ của các $\mathbb{Z}$-module
đẳng cấu với phạm trù $\mathbf{Ab}$ của các nhóm abel: có các hàm tử nghịch đảo
trên-điểm và trên-cấu xạ.
\end{exercise}

\begin{proof}
% \textbf{1. Hai hàm tử ứng viên.}
\begin{itemize}
  \item Xét $U:\mathbf{Mod}_{\mathbb{Z}}\to\mathbf{Ab}$,
  \[
    U(M,+,\cdot)=(M,+),\qquad U(\varphi)=\varphi.
  \]
  Rõ ràng $U$ là một hàm tử:
  \[U(\Id_M)=\Id_{U(M)}\quad U(\psi\circ\varphi)=U(\psi)\circ U(\varphi).\]

  \item Xét $T:\mathbf{Ab}\to\mathbf{Mod}_{\mathbb{Z}},\quad T(A,+)=(A,+,\cdot),\quad T(f) = f$ với mọi nhóm abel $A$ và đồng cấu nhóm $f$ trong $\mathbf{Ab}$.
  \[
    n\cdot a=
    \begin{cases}
      \underbrace{a+\cdots+a}_{n\ \text{lần}}, & n>0,\\[2pt]
      0, & n=0,\\[2pt]
      -\underbrace{(a+\cdots+a)}_{(-n)\ \text{lần}}, & n<0.
    \end{cases}
  \]
\end{itemize}

$T(A)$là một $\mathbb{Z}$-module vì với mọi $m,n\in\mathbb{Z}$ và $a,b\in A$ ta có
\[
\begin{aligned}
(m+n)\cdot a 
&= \underbrace{a+\cdots+a}_{m+n} 
 = \underbrace{a+\cdots+a}_{m}\ +\ \underbrace{a+\cdots+a}_{n}
 = m\cdot a + n\cdot a,\\
m\cdot(a+b)
&= \underbrace{(a+b)+\cdots+(a+b)}_{m}
 = \underbrace{a+\cdots+a}_{m}+\underbrace{b+\cdots+b}_{m}
 = m\cdot a + m\cdot b,\\
(mn)\cdot a 
&= \underbrace{a+\cdots+a}_{mn}
 = \underbrace{(\underbrace{a+\cdots+a}_{n})+\cdots+(\underbrace{a+\cdots+a}_{n})}_{m}
 = m\cdot(n\cdot a),\\
1\cdot a &= a.
\end{aligned}
\]
Các trường hợp $m<0$ hoặc $n<0$ tương tự.

Nếu $f:A\to B$ là đồng cấu nhóm abel thì
\[
f(n\cdot a)=\underbrace{f(a)+\cdots+f(a)}_{n}=n\cdot f(a),
\]
nên $f$ là đồng cấu $\mathbb{Z}$-module $T(A)\to T(B)$. Do đó $T$ là một hàm tử.

$U$ và $T$ là nghịch đảo của nhau vì
\[
U\circ T(A,+)=U(A,+,\cdot)=(A,+)=\Id_{\mathbf{Ab}}(A),
\]
\[
T\circ U(M,+,\cdot)=T(M,+)=(M,+,\cdot)=\Id_{\mathbf{Mod}_{\mathbb{Z}}}(M),
\]
% vì phép nhân vô hướng khôi phục bởi $T$ chính là phép đã có trên $M$.
% Trên cấu xạ, cả hai hàm tử đều để nguyên đồng cấu, nên
% $U\circ T=\Id_{\mathbf{Ab}}$ và $T\circ U=\Id_{\mathbf{Mod}_{\mathbb{Z}}}$ như hàm tử.
Vậy
% \textbf{5. Kết luận (đẳng cấu phạm trù).}
% Vì tồn tại các hàm tử $U,T$ với $U\circ T=\Id_{\mathbf{Ab}}$ và
% $T\circ U=\Id_{\mathbf{Mod}_{\mathbb{Z}}}$, hai phạm trù là \emph{đẳng cấu} (thậm chí \emph{đồng nhất hoá} theo đúng nghĩa trên-đối-tượng và trên-mũi-tên):
\[
\mathbf{Mod}_{\mathbb{Z}} \cong \mathbf{Ab}.
\]

\medskip
\noindent\textbf{Nhận xét.}
\begin{enumerate}
  \item Từ đẳng cấu trên suy ra mọi mệnh đề/hệ quả về nhóm abel dịch nguyên vẹn sang ngôn ngữ $\mathbb{Z}$-module và ngược lại (ví dụ, thương, nhân trực tiếp, dãy exact, định lý đẳng cấu, phân tích nhóm abel hữu hạn, v.v.).
  \item Trường hợp tổng quát: với vành giao hoán $R$, $\mathbf{Mod}_R$ khái quát $\mathbf{Ab}$;
  trường hợp $R=\mathbb{Z}$ chính là $\mathbf{Ab}$.
\end{enumerate}
\end{proof}


\begin{exercise}[Bài tập 7 (rất chi tiết): Phạm trù con đầy của $\mathbf{Ab}$ đẳng cấu với $\mathbf{Mod}_{\mathbb{Z}/n\mathbb{Z}}$]
Với $n\in\mathbb{Z}_{>0}$, hãy xác định một \emph{phạm trù con đầy}
$\mathbf{Ab}_n\subset \mathbf{Ab}$ sao cho
\[
\mathbf{Mod}_{\mathbb{Z}/n\mathbb{Z}} \;\simeq\; \mathbf{Ab}_n .
\]
\end{exercise}

\begin{proof}[Lời giải chi tiết với kí hiệu]
{(1) Định nghĩa ứng viên cho phạm trù con đầy.}
Kí hiệu
\[
\mathbf{Ab}_n\;:=\;\bigl\{\,A\in \mathbf{Ab}\ \bigm|\ nA=0\,\bigr\}
\]
là \emph{phạm trù con đầy} của $\mathbf{Ab}$ gồm các nhóm abel $A$ bị tiêu bởi $n$,
tức $\forall a\in A$ thì $n\cdot a=\underbrace{a+\cdots+a}_{n\ \text{lần}}=0$.
Đây còn được gọi là lớp các nhóm abel \emph{có số mũ (exponent) chia hết cho $n$}.
Tính “đầy” nghĩa là, với mọi $A,B\in\mathbf{Ab}_n$,
\[
\Hom_{\mathbf{Ab}_n}(A,B)
\;=\;
\Hom_{\mathbf{Ab}}(A,B).
\]

{(2) Hai hàm tử ứng viên tạo đẳng cấu phạm trù.}
Gọi $\pi:\mathbb{Z}\twoheadrightarrow \mathbb{Z}/n\mathbb{Z}$ là chuẩn chiếu.
Ta xét hai hàm tử
\[
U:\mathbf{Mod}_{\mathbb{Z}/n\mathbb{Z}}\longrightarrow \mathbf{Ab}_n,
\qquad
T:\mathbf{Ab}_n\longrightarrow \mathbf{Mod}_{\mathbb{Z}/n\mathbb{Z}}
\]
được định nghĩa như sau.

\smallskip
\emph{(i) hàm tử quên cấu trúc $U$.}
Trên Vật, $U(M)= (M,+)$ (quên tác động của $\mathbb{Z}/n\mathbb{Z}$);
trên cấu xạ $f:M\to N$, đặt $U(f)=f$.
Vì $M$ là $\mathbb{Z}/n\mathbb{Z}$-module nên $[n]=0$ trong vành tác động,
do đó $n\cdot m=0$ với mọi $m\in M$ (nhìn như $\mathbb{Z}$-module qua hợp thành
$\mathbb{Z}\xrightarrow{\pi}\mathbb{Z}/n\mathbb{Z}$), suy ra $U(M)\in\mathbf{Ab}_n$.

\smallskip
\emph{(ii) hàm tử “trang bị tác động” $T$.}
Với $A\in\mathbf{Ab}_n$, định nghĩa cấu trúc $\mathbb{Z}/n\mathbb{Z}$-module trên nền $A$
bởi quy tắc
\[
[k]\cdot a \;:=\; \underbrace{a+\cdots+a}_{k\ \text{lần}}\quad(k\in\mathbb{Z}\ \text{đại diện của }[k]).
\]
Tính \emph{xác định tốt}: nếu $k'\equiv k\pmod n$ thì $k'=k+qn$, suy ra
\[
k'\cdot a = k\cdot a + q\,(n\cdot a) = k\cdot a + q\cdot 0 = k\cdot a,
\]
vì $A\in\mathbf{Ab}_n$.
Tính chất module (tuyến tính theo cả hai biến) suy ra từ quy tắc cộng lặp trong nhóm abel.
Trên cấu xạ, nếu $g:A\to B$ là đồng cấu nhóm abel (nên $\in\Hom_{\mathbf{Ab}_n}$), đặt $T(g)=g$.
Ta có $g([k]\cdot a) = g(k\cdot a)=k\cdot g(a)=[k]\cdot g(a)$,
nên $T(g)$ là đồng cấu $\mathbb{Z}/n\mathbb{Z}$-module.

{(3) $U$ là hàm tử \emph{đầy} và \emph{trung thực} (fully faithful).}
Với $M,N\in\mathbf{Mod}_{\mathbb{Z}/n\mathbb{Z}}$, ta có đẳng cấu chuẩn
\[
\Hom_{\mathbb{Z}/n\mathbb{Z}}(M,N)
\;\cong\;
\Hom_{\mathbf{Ab}}(U(M),U(N)),
\]
vì với mọi đồng cấu nhóm abel $f:U(M)\to U(N)$, ta có
\[
f([k]\cdot m)=f(\underbrace{m+\cdots+m}_{k})
=\underbrace{f(m)+\cdots+f(m)}_{k}
=[k]\cdot f(m),
\]
tức \emph{mọi} đồng cấu nhóm abel giữa $U(M),U(N)$ tự động là tuyến tính
$\mathbb{Z}/n\mathbb{Z}$-module. Do đó
\[
\Hom_{\mathbb{Z}/n\mathbb{Z}}(M,N) = \Hom_{\mathbf{Ab}}(U(M),U(N)),
\]
và $U$ là fully faithful.

{(4) $U$ \emph{thiết yếu bao trùm} (essentially surjective) lên $\mathbf{Ab}_n$.}
Với $A\in\mathbf{Ab}_n$, áp dụng $T$ như (ii) ta có một $\mathbb{Z}/n\mathbb{Z}$-module $T(A)$
và $U(T(A))=A$ (đồng nhất trên Vật và cấu xạ).
Vì thế, mọi Vật của $\mathbf{Ab}_n$ đều nằm trong ảnh thiết yếu của $U$.

{(5) Tự nhiên hoá: $T$ và $U$ là nghịch đảo tự nhiên.}
Ta có
\[
(U\circ T)(A)=A\quad\text{(đồng nhất trên $A$)},\qquad
(T\circ U)(M)=M\quad\text{(khôi phục đúng tác động ban đầu)}.
\]
Trên cấu xạ, cả $U$ và $T$ đều giữ nguyên ánh xạ, nên
\[
U\circ T \;=\; \Id_{\mathbf{Ab}_n},
\qquad
T\circ U \;=\; \Id_{\mathbf{Mod}_{\mathbb{Z}/n\mathbb{Z}}}.
\]
Do đó $U$ là một \emph{đẳng cấu phạm trù} (strict categorical isomorphism), và đồng thời
$\mathbf{Mod}_{\mathbb{Z}/n\mathbb{Z}} \simeq \mathbf{Ab}_n$.

{(6) Mô tả bằng \emph{ảnh thiết yếu} (essential image).}
Ảnh thiết yếu của $U$ đúng bằng
\[
\mathrm{EssIm}(U)
=\bigl\{\,A\in\mathbf{Ab}\mid \exists\,M\in\mathbf{Mod}_{\mathbb{Z}/n\mathbb{Z}},\ A\simeq U(M)\,\bigr\}
=\mathbf{Ab}_n.
\]
Tính “đầy” của $\mathbf{Ab}_n$ là hiển nhiên, vì tập cấu xạ giữ nguyên.

{(7) Các đặc trưng tương đương của $\mathbf{Ab}_n$.}
Với $A\in\mathbf{Ab}$, các điều kiện sau tương đương:
\begin{enumerate}
  \item $A\in\mathbf{Ab}_n$ (tức $nA=0$);
  \item $A$ là $\mathbb{Z}$-module qua $\pi:\mathbb{Z}\to\mathbb{Z}/n\mathbb{Z}$ sao cho $n$ tác động bằng $0$;
  \item $A$ là \emph{$n$-torsion group} với số mũ $\exp(A)\mid n$;
  \item $A$ là tổng (hoặc tích) trực tiếp của các cyclic $\mathbb{Z}$-module $\mathbb{Z}/d\mathbb{Z}$ với $d\mid n$.
\end{enumerate}

{(8) Ví dụ—phản ví dụ.}
\[
\begin{array}{ll}
\text{Ví dụ:} & (\mathbb{Z}/n\mathbb{Z})^{(I)},\quad \mathbb{Z}/d\mathbb{Z}\ (d\mid n),\quad 
\prod_{i=1}^r \mathbb{Z}/d_i\mathbb{Z}\ (d_i\mid n) \in \mathbf{Ab}_n.\\[2pt]
\text{Phản ví dụ:} & \mathbb{Z},\ \mathbb{Q}/\mathbb{Z},\ \mathbb{Z}/p^2\mathbb{Z}\ \text{khi }n=p
\ \ (\text{vì }p^2\nmid p).
\end{array}
\]

{(9) Kết luận.}
Phạm trù con đầy cần tìm là
\[
\boxed{\ \mathbf{Ab}_n \;=\; \{\,A\in\mathbf{Ab}\mid nA=0\,\}\ } ,
\]
và hai hàm tử $U$ (quên cấu trúc) và $T$ (trang bị tác động) cho một đẳng cấu phạm trù
\[
\mathbf{Mod}_{\mathbb{Z}/n\mathbb{Z}} \;\cong\; \mathbf{Ab}_n,
\]
với $\Hom_{\mathbb{Z}/n\mathbb{Z}}(M,N)=\Hom_{\mathbf{Ab}}(U(M),U(N))$ một cách tự nhiên.
\end{proof}
\begin{exercise}[Bài tập 7 (tổng hợp, gọn mà đủ): Phạm trù con đầy của $\mathbf{Ab}$ đẳng cấu với $\mathbf{Mod}_{\mathbb{Z}/n\mathbb{Z}}$]


% Chứng minh rằng (i) $\mathbf{Ab}_n$ là \emph{phạm trù con đầy} của $\mathbf{Ab}$ và (ii) 
% \[
% \mathbf{Mod}_{\mathbb{Z}/n\mathbb{Z}}\ \cong\ \mathbf{Ab}_n.
% \]
\end{exercise}

\begin{proof}
Với $n\in\mathbb{Z}_{>0}$, đặt
\[
\mathbf{Ab}_n:=\{\,A\in\mathbf{Ab}\mid nA=0\,\}
=\{\,A\in\mathbf{Ab}\mid \forall a\in A,\; \underbrace{a+\cdots+a}_{n}=0\,\}.
\]
$\mathbf{Ab}_n$ là phạm trù con đầy của $\mathbf{Ab}$. Thật vậy, với $A,B\in\mathbf{Ab}_n$ đặt
\[
\Hom_{\mathbf{Ab}_n}(A,B):=\Hom_{\mathbf{Ab}}(A,B)
\]
Khi đó $\Id_A\in\Hom_{\mathbf{Ab}_n}(A,A)$ và nếu
$f\in\Hom_{\mathbf{Ab}_n}(A,B),\,g\in\Hom_{\mathbf{Ab}_n}(B,C)$ thì 
$g\circ f\in\Hom_{\mathbf{Ab}_n}(A,C)$ vì hợp thành trong $\mathbf{Ab}$. 

% Gọi $I:\mathbf{Ab}_n\hookrightarrow\mathbf{Ab}$ là hàm tử bao hàm ($I$ giữ nguyên Vật
% và cấu xạ). Với mọi $A,B\in\Ob(\mathbf{Ab}_n)$ ta có
% \[
% \Hom_{\mathbf{Ab}_n}(A,B)=\Hom_{\mathbf{Ab}}(A,B),
% \]
% nên ứng với mỗi cặp $(A,B)$, ánh xạ tự nhiên
% \[
% \begin{tikzcd}[column sep=large]
% \Hom_{\mathbf{Ab}_n}(A,B) \arrow[r,"I_{A,B}"] \arrow[d, equal] 
% & \Hom_{\mathbf{Ab}}(I(A),I(B)) \arrow[d, equal] \\
% \Hom_{\mathbf{Ab}}(A,B) \arrow[r, equal] & \Hom_{\mathbf{Ab}}(A,B)
% \end{tikzcd}
% \]
% là đồng nhất, do đó \emph{toàn ánh}. Vậy $I$ là hàm tử \emph{đầy} (và hiển nhiên trung thực), nên
% $\mathbf{Ab}_n$ là \emph{phạm trù con đầy} của $\mathbf{Ab}$.

% \medskip

% $\mathbf{Mod}_{\mathbb{Z}/n\mathbb{Z}}\cong\mathbf{Ab}_n$.}
Tiếp theo ta xây dựng hàm tử giữa hai phạm trù $\mathbf{Ab}_n$ và $\mathbf{Mod}_{\Z/n\Z}$. Ta định nghĩa

\[U:\mathbf{Mod}_{\mathbb{Z}/n\mathbb{Z}}\longrightarrow \mathbf{Ab}_n,\quad U(M,+,\cdot)=(M,+),~U(f)=f;\]

\[T:\mathbf{Ab}_n\longrightarrow \mathbf{Mod}_{\mathbb{Z}/n\mathbb{Z}},~T(A,+) = (A,+,\cdot)
\text{ với } ([k])\cdot a:=k\cdot a,\quad T(g)=g.\]

$T$ được định nghĩa tốt vì nếu $[k']=[k] \in \Z/n\Z$ thì tồn tại $q\in \Z$ sao cho $k'=k+qn$.

Khi đó
\[k'\cdot a=k\cdot a+q(\underbrace{n\cdot a}_{=0})=k\cdot a.\]

Nếu $g:A\to B$ là đồng cấu nhóm, thì
$g([k]\cdot a)=g(k\cdot a)=k\cdot g(a)=[k]\cdot g(a)$,
nên $T(g)$ là đồng cấu $\mathbb{Z}/n\mathbb{Z}$-module.

Với $M \in \mathbf{Mod}_{\mathbb{Z}/n\mathbb{Z}}$, ta có $nM = \{n\cdot m=0\mid m \in M\}=0$. Do đó $U(M)\in\mathbf{Ab}_n$.

Hai hàm tử thỏa mãn
\[
U\circ T=\Id_{\mathbf{Ab}_n},\qquad T\circ U=\Id_{\mathbf{Mod}_{\mathbb{Z}/n\mathbb{Z}}},
\]
Tức $\mathbf{Mod}_{\Z/n\Z} \cong \mathbf{Ab}_n$.
% vì một bên “quên” còn bên kia “khôi phục” đúng tác động $\mathbb{Z}/n\mathbb{Z}$. Biểu đồ đẳng cấu:
% \[
% \begin{tikzcd}[row sep=large, column sep=large]
% \mathbf{Mod}_{\mathbb{Z}/n\mathbb{Z}} \arrow[r, shift left=1.0ex, "U"]
% & \mathbf{Ab}_n \arrow[l, shift left=1.0ex, "T"] \\
% \end{tikzcd}
% \qquad
% \begin{tikzcd}[column sep=huge,row sep=large]
% M \arrow[d, mapsto, "\text{forget }(+,\cdot)\mapsto (+)"] \arrow[r, "f"] 
% & N \arrow[d, mapsto] \\
% U(M) \arrow[r, "U(f)"] & U(N)
% \end{tikzcd}
% \]
% giao hoán trên mọi cấu xạ.

% Kết luận: $U$ và $T$ là nghịch đảo, nên có đẳng cấu phạm trù
% \[
% \boxed{\ \mathbf{Mod}_{\mathbb{Z}/n\mathbb{Z}}\ \cong\ \mathbf{Ab}_n
% =\{A\in\mathbf{Ab}\mid nA=0\}\ }.
% \]
\end{proof}

% \begin{exercise}[Bài tập 8: Phạm trù $\mathbf{End}_{\mathbb{K}}$ và $\mathbf{Mod}_{\mathbb{K}[x]}$]
% Định nghĩa $\mathbf{End}_{\mathbb{K}}$ là phạm trù có:
% \begin{itemize}
%   \item \textbf{Vật:} các cặp $(V,f)$, trong đó $V$ là $K$-không gian vectơ và $f:V\to V$ là tự đồng cấu tuyến tính.
%   \item \textbf{cấu xạ:} các ánh xạ tuyến tính $g:(V,f)\to(W,h)$ thoả $g\circ f = h\circ g$.
% \end{itemize}
% Chứng minh rằng
% \[
% \mathbf{End}_{\mathbb{K}} \;\cong\; \mathbf{Mod}_{\mathbb{K}[x]},
% \]
% trong đó $\mathbb{K}[x]$ là vành đa thức một biến trên $K$.
% \end{exercise}

% \begin{proof}
% \textbf{1. Xây dựng hàm tử.}
% Với mỗi $(V,f)\in\Ob(\mathbf{End}_{\mathbb{K}})$, ta định nghĩa một cấu trúc $\mathbb{K}[x]$-module trên $V$ bằng
% \[
% p(x)\cdot v := p(f)(v),\qquad p(x)\in \mathbb{K}[x],\ v\in V.
% \]
% Khi đó $V$ trở thành một $\mathbb{K}[x]$-module (vì $(pq)(f)=p(f)q(f)$ và $1(f)=\Id_V$).

% Ngược lại, nếu $M$ là $\mathbb{K}[x]$-module, định nghĩa $f_M:M\to M$ bởi
% \[
% f_M(m):=x\cdot m.
% \]
% Vì $x$ là phần tử sinh của $\mathbb{K}[x]$, ta thu được cặp $(M,f_M)\in\Ob(\mathbf{End}_{\mathbb{K}})$.

% \textbf{2. Trên cấu xạ.}
% Nếu $g:(V,f)\to(W,h)$ là ánh xạ tuyến tính thỏa $g\circ f=h\circ g$, thì
% \[
% g(p(f)(v)) = p(h)(g(v)) \quad \forall p(x)\in \mathbb{K}[x],
% \]
% nên $g$ là đồng cấu $\mathbb{K}[x]$-module.  
% Ngược lại, nếu $g:V\to W$ là đồng cấu $\mathbb{K}[x]$-module, thì $g(x\cdot v)=x\cdot g(v)$, tức $g(f(v))=h(g(v))$.

% \textbf{3. Đẳng cấu phạm trù.}
% Hai quá trình trên là nghịch đảo:  
% \[
% \begin{tikzcd}[column sep=large, row sep=large]
% \mathbf{End}_{\mathbb{K}} \arrow[r, shift left=1ex, "F"] 
% & \mathbf{Mod}_{\mathbb{K}[x]} \arrow[l, shift left=1ex, "G"]
% \end{tikzcd}
% \]
% với $F(V,f)=(V,+,p(x)\cdot v:=p(f)(v))$ và $G(M)=(M,f_M)$, thỏa
% $G(F(V,f))=(V,f)$ và $F(G(M))=(M,+,x\cdot m)$.

% \textbf{4. Kết luận.}
% Do đó có đẳng cấu phạm trù tự nhiên:
% \[
% \boxed{\ \mathbf{End}_{\mathbb{K}} \;\simeq\; \mathbf{Mod}_{\mathbb{K}[x]}\ }.
% \]
% Từ đó, nghiên cứu cấu trúc tự đồng cấu tuyến tính $(V,f)$ tương đương với nghiên cứu các module hữu hạn sinh trên vành $\mathbb{K}[x]$.
% \end{proof}
% \begin{exercise}[Bài tập 8: $\mathbf{End}_{\mathbb{K}}$ đẳng cấu với $\mathbf{Mod}_{\mathbb{K}[x]}$]
% Mục tiêu: mô tả \emph{tường minh} phạm trù $\mathbf{End}_{\mathbb{K}}$ và xây dựng đẳng cấu phạm trù
% \[
% \mathbf{End}_{\mathbb{K}} \;\cong\; \mathbf{Mod}_{\mathbb{K}[x]}.
% \]
% \end{exercise}

% \begin{proof}
% \textbf{Định nghĩa chi tiết phạm trù $\mathbf{End}_{\mathbb{K}}$.}
% \begin{enumerate}
%   \item \textbf{Vật:} cặp $(V,f)$ với $V$ là $K$-không gian vectơ hữu hạn chiều (tùy yêu cầu) và $f\in\End_{\mathbb{K}}(V)$.
%   \item \textbf{cấu xạ:} với $(V,f)$ và $(W,h)$, một cấu xạ $g:(V,f)\to(W,h)$ là ánh xạ tuyến tính $g:V\to W$ thỏa
%   \[
%   g\circ f = h\circ g.
%   \]
%   \item \textbf{Đơn vị:} $\Id_{(V,f)}:=\Id_V$, vì $\Id_V\circ f=f=f\circ\Id_V$.
%   \item \textbf{Hợp thành:} nếu $g:(V,f)\to(W,h)$ và $u:(W,h)\to(U,k)$, đặt $u\circ g:(V,f)\to(U,k)$;
%   ta có
%   \[
%   (u\circ g)\circ f = u\circ(g\circ f) = u\circ(h\circ g)=(u\circ h)\circ g=(k\circ u)\circ g=k\circ(u\circ g).
%   \]
%   \item \textbf{Tính chất phạm trù:} hợp thành kết hợp và đơn vị là $\Id$ như thường lệ, nên $\mathbf{End}_{\mathbb{K}}$ là một phạm trù.
% \end{enumerate}

% \textbf{Xây dựng đẳng cấu $\mathbf{End}_{\mathbb{K}} \cong \mathbf{Mod}_{\mathbb{K}[x]}$.}
% \begin{enumerate}
%   \item[\textbf{(A)}] \textbf{hàm tử $F:\mathbf{End}_{\mathbb{K}}\to\mathbf{Mod}_{\mathbb{K}[x]}$.}
%   \begin{enumerate}
%     \item Trên \emph{Vật}: $F(V,f)$ là $\mathbb{K}[x]$-module có nền là $V$ và tác động
%     \[
%     p(x)\cdot v := p(f)(v)\qquad(p\in \mathbb{K}[x],\,v\in V).
%     \]
%     Kiểm tra: $(pq)(f)=p(f)q(f)$ và $1(f)=\Id_V$ nên đúng tiên đề module.
%     \item Trên \emph{cấu xạ}: nếu $g:(V,f)\to(W,h)$ thì với mọi $p\in \mathbb{K}[x]$,
%     \[
%     g\big(p(f)v\big)=p(h)\,g(v),
%     \]
%     do $g f = h g$ và tuyến tính, suy ra $g$ là đồng cấu $\mathbb{K}[x]$-module $F(V,f)\to F(W,h)$.
%     \item $F$ bảo toàn đơn vị và hợp thành vì nền là cùng $g$.
%   \end{enumerate}

%   \item[\textbf{(B)}] \textbf{hàm tử $G:\mathbf{Mod}_{\mathbb{K}[x]}\to\mathbf{End}_{\mathbb{K}}$.}
%   \begin{enumerate}
%     \item Trên \emph{Vật}: với $M$ là $\mathbb{K}[x]$-module, đặt $f_M:M\to M$ bởi
%     \[
%     f_M(m):=x\cdot m.
%     \]
%     Khi đó $(M,f_M)\in\Ob(\mathbf{End}_{\mathbb{K}})$.
%     \item Trên \emph{cấu xạ}: nếu $\varphi:M\to N$ là đồng cấu $\mathbb{K}[x]$-module thì
%     \[
%     \varphi\circ f_M(m)=\varphi(x\cdot m)=x\cdot \varphi(m)=f_N\circ\varphi(m),
%     \]
%     nên $\varphi:(M,f_M)\to(N,f_N)$ là cấu xạ trong $\mathbf{End}_{\mathbb{K}}$.
%     \item $G$ cũng bảo toàn đơn vị và hợp thành (vì lấy chính $\varphi$).
%   \end{enumerate}

%   \item[\textbf{(C)}] \textbf{Hai hàm tử là nghịch đảo.}
%   \begin{enumerate}
%     \item Với $(V,f)$,
%     \[
%     G(F(V,f))=(V,\; m\mapsto x\cdot m)=\big(V,\; m\mapsto f(m)\big)=(V,f).
%     \]
%     \item Với $M$,
%     \[
%     F(G(M)):\ p(x)\cdot m := p(f_M)(m)=p(x)\cdot m,
%     \]
%     % tức khôi phục đúng tác động $\mathbb{K}[x]$ ban đầu; 
%     vì thế $F(G(M))=M$.
%   \end{enumerate}
% \end{enumerate}

% % \textbf{Biểu đồ tóm tắt.}
% % \[
% % \begin{tikzcd}[column sep=large,row sep=large]
% % \mathbf{End}_{\mathbb{K}} \arrow[r, shift left=1.1ex, "F"]
% % & \mathbf{Mod}_{\mathbb{K}[x]} \arrow[l, shift left=1.1ex, "G"] \\
% % (V,f) \arrow[d, mapsto] \arrow[r, "g"] & (W,h) \arrow[d, mapsto] \\
% % F(V,f) \arrow[r, "F(g)"] & F(W,h)
% % \end{tikzcd}
% % \qquad
% % \begin{tikzcd}[column sep=huge]
% % G(F(V,f)) \arrow[r, equals] & (V,f) \\
% % F(G(M)) \arrow[r, equals] & M
% % \end{tikzcd}
% % \]

% % \textbf{Kết luận.} 
% Vậy $F$ và $G$ là hai hàm tử nghịch đảo, do đó
% \[
% \boxed{\ \mathbf{End}_{\mathbb{K}} \;\cong\; \mathbf{Mod}_{\mathbb{K}[x]}\ }.
% \]
% % Vì vậy, việc phân loại các tự đồng cấu tuyến tính $(V,f)$ tương đương với bài toán module trên $\mathbb{K}[x]$ (ví dụ: qua phân tích tối giản/Jordan khi $K$ đại số đóng).
% \end{proof}

\begin{exercise}[Bài tập 8 (phiên bản chi tiết): $\mathbf{End}_{\mathbb{K}} \simeq \mathbf{Mod}_{\mathbb{K}[x]}$]
Mục tiêu: mô tả \emph{tường minh} phạm trù $\mathbf{End}_{\mathbb{K}}$ và xây dựng một \emph{đẳng cấu phạm trù tự nhiên}
\[
\mathbf{End}_{\mathbb{K}} \;\simeq\; \mathbf{Mod}_{\mathbb{K}[x]},
\]
trong đó $\mathbb{K}[x]$ là vành đa thức một biến trên $K$.
\end{exercise}

\begin{proof}
\textbf{Phạm trù $\mathbf{End}_{\mathbb{K}}$.}
\begin{enumerate}
  \item \textbf{Vật:} cặp $(V,f)$, $V$ là $\mathbb{K}$-không gian vectơ (không nhất thiết hữu hạn chiều), $f\in\End_{\mathbb{K}}(V)$.
  \item \textbf{Cấu xạ:} một cấu xạ $g:(V,f)\to(W,h)$ là ánh xạ tuyến tính $g:V\to W$ thỏa mãn
  \[
  g\circ f = h\circ g. \tag{$\ast$}
  \]
  \item \textbf{Đơn vị:} $\Id_{(V,f)}:=\Id_V$ (rõ ràng $\Id_V\circ f=f=f\circ\Id_V$).
  \item \textbf{Hợp thành:} nếu $g:(V,f)\to(W,h)$, $u:(W,h)\to(U,k)$, đặt $u\circ g:(V,f)\to(U,k)$; ta có
  \[
  (u\circ g)\circ f=u\circ(g\circ f)=u\circ(h\circ g)=(u\circ h)\circ g=(k\circ u)\circ g=k\circ(u\circ g),
  \]
  nên $(\ast)$ bảo toàn dưới hợp thành. Kết hợp và đơn vị đúng như trong $\mathbf{Vect}_{\mathbb{K}}$, vì thế $\mathbf{End}_{\mathbb{K}}$ là một phạm trù.
\end{enumerate}

\textbf{Định nghĩa hàm tử $F:\mathbf{End}_{\mathbb{K}}\to\mathbf{Mod}_{\mathbb{K}[x]}$.}
\begin{enumerate}
    \item Với $(V,f)$, giữ nguyên nhóm cộng $V$ và đặt tác động
    \[
      p(x)\,\cdot\, v \ :=\ p(f)(v),\qquad p(x)=\sum_{i=0}^m a_i x^i\in \mathbb{K}[x],\; v\in V,
    \]
    trong đó $p(f):=\sum_{i=0}^m a_i f^i\in\End_{\mathbb{K}}(V)$.
    \item Với $g:(V,f)\to(W,h)$ thỏa $(\ast)$, khi đó $F(g):=g$ là một ánh xạ $\mathbb{K}[x]$-tuyến tính:
    \[
      F(g)\big(p(x)\cdot v\big)=g\big(p(f)v\big)=p(h)\,g(v)=p(x)\cdot F(g)(v).
    \]

  \item $F(V,f)$ là một $\mathbb{K}[x]-$module.
  
  Với $p,q\in \mathbb{K}[x]$, $v,w\in V$, $\lambda\in K$:
  \begin{align*}
    (p+q)\cdot v &= (p(f)+q(f))v = p(f)v + q(f)v = p\cdot v + q\cdot v,\\
    p\cdot (v+w) &= p(f)(v+w) = p(f)v + p(f)w = p\cdot v + p\cdot w,\\
    (pq)\cdot v &= (pq)(f)v = p(f)\big(q(f)v\big)=p\cdot(q\cdot v),\\
    1\cdot v &= 1(f)v=\Id_V(v)=v,\\
    (\lambda\in K\subset \mathbb{K}[x])\quad \lambda\cdot v &= \lambda\,\Id_V(v)=\lambda v,
  \end{align*}
  nhờ đồng cấu vành $\mathbb{K}[x]\to\End_{\mathbb{K}}(V)$, $p(x)\mapsto p(f)$. 
\end{enumerate}

\textbf{Định nghĩa hàm tử $G:\mathbf{Mod}_{\mathbb{K}[x]}\to\mathbf{End}_{\mathbb{K}}$.}
  \begin{enumerate}
    \item Với $M\in\Ob(\mathbf{Mod}_{\mathbb{K}[x]})$, đặt $f_M:M\to M$ bởi
    \[
      f_M(m):=x\cdot m.
    \]
    Khi đó $(M,f_M)\in\Ob(\mathbf{End}_{\mathbb{K}})$.
    \item Nếu $\varphi:M\to N$ là đồng cấu $\mathbb{K}[x]$-module thì
    \[
      \varphi\circ f_M(m)=\varphi(x\cdot m)=x\cdot \varphi(m)=f_N\circ\varphi(m),
    \]
    nên $G(\varphi):=(\varphi: (M,f_M)\to(N,f_N))$ là cấu xạ trong $\mathbf{End}_{\mathbb{K}}$.
  \end{enumerate}

% \begin{enumerate}
%   \item \textbf{Đẳng cấu tự nhiên $\eta:\Id_{\mathbf{End}_{\mathbb{K}}}\Rightarrow G\circ F$.}
%   Với $(V,f)$, định nghĩa thành phần
%   \[
%   \eta_{(V,f)}:=\Id_V:\ (V,f)\longrightarrow G(F(V,f))=(V,f_{F(V,f)}),
%   \]
%   trong đó $f_{F(V,f)}$ là phép nhân bởi $x$ trong $F(V,f)$, tức
%   $f_{F(V,f)}(v)=x\cdot v=f(v)$. Vậy $G(F(V,f))=(V,f)$ và $\eta_{(V,f)}$ là đồng nhất.
%   Tính \emph{tự nhiên} (naturality) của $\eta$ theo $g:(V,f)\to(W,h)$ hiển nhiên vì mọi thành phần đều là $\Id$:
%   \[
%   \begin{tikzcd}[column sep=large,row sep=large]
%   (V,f) \arrow[r,"g"] \arrow[d,"\eta_{(V,f)}"',"\cong" sloped] & (W,h) \arrow[d,"\eta_{(W,h)}","\cong" sloped]\\
%   G F(V,f) \arrow[r,"G F(g)"] & G F(W,h)
%   \end{tikzcd}
%   \]
%   giao hoán.

%   \item \textbf{Đẳng cấu tự nhiên $\varepsilon:F\circ G\Rightarrow \Id_{\mathbf{Mod}_{\mathbb{K}[x]}}$.}
%   Với $M\in\Ob(\mathbf{Mod}_{\mathbb{K}[x]})$, $F(G(M))$ là $M$ với tác động $p(x)\cdot m:=p(f_M)(m)$.
%   Nhưng $f_M$ chính là tác động của $x$ trên $M$, nên
%   \[
%   p(f_M)(m)=\Big(\sum a_i f_M^i\Big)(m)=\sum a_i\,x^i\cdot m=p(x)\cdot m,
%   \]
%   tức $F(G(M))$ có đúng tác động gốc; đặt $\varepsilon_M:=\Id_M$.
%   Tính tự nhiên theo $\varphi:M\to N$ cũng hiển nhiên:
%   \[
%   \begin{tikzcd}[column sep=large,row sep=large]
%   F G(M) \arrow[r,"F G(\varphi)"] \arrow[d,"\varepsilon_M"',"\cong" sloped] & F G(N) \arrow[d,"\varepsilon_N","\cong" sloped]\\
%   M \arrow[r,"\varphi"] & N
%   \end{tikzcd}
%   \]
%   giao hoán.
% \end{enumerate}
% Hai đẳng cấu tự nhiên $\eta,\varepsilon$ chứng minh $F$ và $G$ là \emph{equivalence}; hơn thế,
% vì các thành phần đều là đồng nhất, $F$ và $G$ thực ra là \emph{đẳng cấu phạm trù} (isomorphism of categories).

% \medskip
% \textbf{E. Đẳng cấu trên hom-set (mô tả tường minh).}
% Với $(V,f),(W,h)\in\Ob(\mathbf{End}_{\mathbb{K}})$,
% \[
% \Hom_{\mathbb{K}[x]}(F(V,f),F(W,h))
% \;=\;\{\,g\in\Hom_K(V,W)\mid g\,p(f)=p(h)\,g\ \forall p\in \mathbb{K}[x]\}
% \;=\;\{\,g\in\Hom_K(V,W)\mid g f = h g\,\},
% \]
% vì điều kiện với mọi $p$ tương đương với điều kiện chỉ cho $p(x)=x$.
% Biểu đồ tự nhiên
% \[
% \begin{tikzcd}[column sep=huge,row sep=large]
% \Hom_{\mathbf{End}_{\mathbb{K}}}\big((V,f),(W,h)\big) \arrow[r,"\cong", "{g\mapsto g}"] 
% & \Hom_{\mathbb{K}[x]}\big(F(V,f),F(W,h)\big)
% \end{tikzcd}
% \]
% thể hiện \emph{đầy và trung thực} của $F$.

% \medskip
% \textbf{F. Kết luận \& hệ quả.}
Từ đó, ta có đẳng cấu phạm trù
\[
\boxed{\ \mathbf{End}_{\mathbb{K}} \;\cong\; \mathbf{Mod}_{\mathbb{K}[x]}\ }.
\]
% Do đó, mọi bài toán về cặp $(V,f)$ (ví dụ: phân rã Jordan, đa thức tối thiểu/đặc trưng, phân tích sơ cấp)
% được chuyển hoá thành bài toán về module hữu hạn sinh trên $\mathbb{K}[x]$ (định lý cấu trúc của PID khi $K$ là trường).
\end{proof}
\begin{proof}[Chứng minh ngắn gọn $F,G$ là đẳng cấu]
Nhắc lại:
\[
F:\mathbf{End}_K\to\mathbf{Mod}_{K[x]},\quad
F(V,f)=\big(V,\ p(x)\cdot v:=p(f)(v)\big),\qquad
G:\mathbf{Mod}_{K[x]}\to\mathbf{End}_K,\quad
G(M)=(M,f_M),\ f_M(m)=x\cdot m .
\]

\begin{enumerate}
  \item \textbf{Đẳng cấu tự nhiên $\eta:\Id_{\mathbf{End}_K}\Rightarrow G\!\circ\!F$.}
  Với $(V,f)$ đặt $\eta_{(V,f)}:=\Id_V:(V,f)\to G(F(V,f))$.
  Quả thật, trong $F(V,f)$ ta có $x\cdot v=f(v)$ nên $G(F(V,f))=(V,f)$; do đó $\eta_{(V,f)}$ là đẳng cấu.
  Tính tự nhiên: với $g:(V,f)\to(W,h)$,
  \[
  \begin{tikzcd}[column sep=large,row sep=large]
  (V,f) \arrow[r,"g"] \arrow[d,"\eta_{(V,f)}"',"\cong" sloped] & (W,h) \arrow[d,"\eta_{(W,h)}","\cong" sloped]\\
  G F(V,f) \arrow[r,"G F(g)"] & G F(W,h)
  \end{tikzcd}
  \]
  giao hoán vì các mũi tên dọc đều là $\Id$.

  \item \textbf{Đẳng cấu tự nhiên $\varepsilon:F\!\circ\!G\Rightarrow \Id_{\mathbf{Mod}_{K[x]}}$.}
  Với $M$, trong $G(M)$ ta có $f_M(m)=x\cdot m$, nên
  \[
    F(G(M)):\ p(x)\cdot m:=p(f_M)(m)=\sum a_i f_M^i(m)=\sum a_i x^i\cdot m=p(x)\cdot m,
  \]
  tức trùng đúng tác động gốc. Đặt $\varepsilon_M:=\Id_M$; khi đó
  \[
  \begin{tikzcd}[column sep=large,row sep=large]
  F G(M) \arrow[r,"F G(\varphi)"] \arrow[d,"\varepsilon_M"',"\cong" sloped] & F G(N) \arrow[d,"\varepsilon_N","\cong" sloped]\\
  M \arrow[r,"\varphi"] & N
  \end{tikzcd}
  \]
  giao hoán với mọi đồng cấu $\varphi$.
\end{enumerate}

Vì tồn tại hai đẳng cấu tự nhiên $\eta$ và $\varepsilon$, nên $F$ và $G$ là nghịch đảo (trên đối tượng và mũi tên), do đó
\[
\boxed{\ F\ \text{và}\ G\ \text{là đẳng cấu phạm trù: }\ \mathbf{End}_K \cong \mathbf{Mod}_{K[x]}. \ }
\]
\end{proof}

\begin{exercise}[Bài tập 9 (biểu diễn nhóm và module trên vành nhóm)]
Gọi $K$ là một trường, $G$ là một nhóm. Kí hiệu $\Rep_K(G)$ là phạm trù các biểu diễn $K$-tuyến tính của $G$
(vật là các cặp $(V,\rho)$ với $\rho:G\to\GL_K(V)$, mũi tên $T:(V,\rho)\to(W,\sigma)$ thỏa $T\circ\rho(g)=\sigma(g)\circ T$),
và $\mathbf{Mod}_{K[G]}$ là phạm trù các $K[G]$-module trái.

\emph{Bài toán.} Chứng minh có đẳng cấu phạm trù tự nhiên
\[
\mathbf{Mod}_{K[G]}\;\cong\;\Rep_K(G).
\]
\end{exercise}

\begin{proof}
\textbf{1. Vành nhóm $K[G]$.}

$K[G]$ là một $K-$không gian vectơ trên $K$ có cơ sở $\{e_g\mid g\in G\,\}$ và phép nhân
\[\big(\sum a_g e_g\big)\big(\sum b_h e_h\big)=\sum_{g,h} a_g b_h\,e_{gh}.\]
Phần tử đơn vị là $1_{K[G]}=1_K\cdot e_G$.

\textbf{2. Hai hàm tử ứng viên.}
\begin{enumerate}
  \item[\textbf{(F)}] \textbf{Từ $\mathbf{Mod}_{K[G]}$ sang $\Rep_K(G)$.}
  \begin{enumerate}
    \item \emph{Trên đối tượng:} Với $M\in\Ob(\mathbf{Mod}_{K[G]})$, định nghĩa
    $\rho_M:G\to\GL_K(M)$ bởi $\rho_M(g)(m):=g\cdot m$.
    Vì $g\cdot(h\cdot m)=(gh)\cdot m$ và $e\cdot m=m$, nên $\rho_M$ là đồng cấu nhóm.
    Khi đó $(M,\rho_M)\in\Ob(\Rep_K(G))$.
    \item \emph{Trên mũi tên:} Nếu $\varphi:M\to N$ là đồng cấu $K[G]$-module, thì
    $\varphi(\rho_M(g)m)=\varphi(g\cdot m)=g\cdot\varphi(m)=\rho_N(g)\varphi(m)$.
    Suy ra $\varphi$ là mũi tên xen kẽ $G$-linear: $\varphi\circ\rho_M(g)=\rho_N(g)\circ\varphi$.
  \end{enumerate}
  Kí hiệu hàm tử này là $F:\mathbf{Mod}_{K[G]}\to\Rep_K(G)$, $F(M)=(M,\rho_M)$, $F(\varphi)=\varphi$.

  \item[\textbf{(G)}] \textbf{Từ $\Rep_K(G)$ sang $\mathbf{Mod}_{K[G]}$.}
  \begin{enumerate}
    \item \emph{Trên đối tượng:} Với $(V,\rho)\in\Ob(\Rep_K(G))$, cho $V$ cấu trúc $K[G]$-module bởi
    \[
      \Big(\sum\nolimits_{g\in G} a_g\,g\Big)\cdot v \ :=\ \sum\nolimits_{g} a_g\,\rho(g)(v).
    \]
    Tuyến tính theo $\sum a_g g$ là hiển nhiên; tính tương hợp tích:
    \[
      \Big(\sum a_g g\Big)\cdot\Big(\sum b_h h\Big)\cdot v
      = \sum_{g,h} a_gb_h\,\rho(g)\rho(h)(v)
      = \sum_{g,h} a_gb_h\,\rho(gh)(v)
      = \Big(\sum a_g g\Big)\Big(\sum b_h h\Big)\cdot v.
    \]
    \item \emph{Trên mũi tên:} Nếu $T:(V,\rho)\to(W,\sigma)$ là biến hình biểu diễn
    ($T\rho(g)=\sigma(g)T$), khi đó với mọi $\sum a_g g\in K[G]$,
    \[
      T\!\left(\Big(\sum a_g g\Big)\cdot v\right)
      = \sum a_g\,T(\rho(g)v)
      = \sum a_g\,\sigma(g)T(v)
      = \Big(\sum a_g g\Big)\cdot T(v),
    \]
    nên $T$ là đồng cấu $K[G]$-module.
  \end{enumerate}
  Kí hiệu hàm tử này là $G:\Rep_K(G)\to\mathbf{Mod}_{K[G]}$, $G(V,\rho)=V$ (với tác động trên),
  $G(T)=T$.
\end{enumerate}

\textbf{3. $F$ và $G$ là nghịch đảo tự nhiên.}
\begin{enumerate}
  \item \emph{Từ trái sang phải:} Với $M\in\Ob(\mathbf{Mod}_{K[G]})$,
  trong $F(M)=(M,\rho_M)$ ta có tác động lại của $K[G]$ qua $G$:
  \[
    \Big(\sum a_g g\Big)\star m := \sum a_g\,\rho_M(g)(m)=\sum a_g\,g\cdot m,
  \]
  chính là tác động gốc trên $M$. Do đó $G(F(M))=M$ \emph{đúng như đối tượng và mũi tên}.
  Định nghĩa đẳng cấu tự nhiên $\varepsilon:\,G\circ F\Rightarrow\Id_{\mathbf{Mod}_{K[G]}}$ với
  $\varepsilon_M=\Id_M$. Biểu đồ tự nhiên:
  \[
  \begin{tikzcd}[column sep=large,row sep=large]
  G F(M) \arrow[r,"G F(\varphi)"] \arrow[d,"\varepsilon_M"',"\cong" sloped] &
  G F(N) \arrow[d,"\varepsilon_N","\cong" sloped] \\
  M \arrow[r,"\varphi"] & N
  \end{tikzcd}
  \]
  giao hoán vì các mũi tên dọc đều là đồng nhất.

  \item \emph{Từ phải sang trái:} Với $(V,\rho)\in\Ob(\Rep_K(G))$, $G(V,\rho)$ là $V$
  với tác động $K[G]$ ở (G). Áp dụng $F$ lại, thu được biểu diễn
  $\rho'(g)(v):=g\cdot v=\rho(g)v$, nên $(V,\rho')=(V,\rho)$. Đặt đẳng cấu tự nhiên
  $\eta:\Id_{\Rep_K(G)}\Rightarrow F\circ G$ với $\eta_{(V,\rho)}=\Id_V$ và biểu đồ:
  \[
  \begin{tikzcd}[column sep=large,row sep=large]
  (V,\rho) \arrow[r,"T"] \arrow[d,"\eta_{(V,\rho)}"',"\cong" sloped] &
  (W,\sigma) \arrow[d,"\eta_{(W,\sigma)}","\cong" sloped] \\
  F G(V,\rho) \arrow[r,"F G(T)"] & F G(W,\sigma)
  \end{tikzcd}
  \]
  giao hoán.
\end{enumerate}

\textbf{4. Kết luận.}
Hai đẳng cấu tự nhiên $\varepsilon$ và $\eta$ chứng tỏ $F$ và $G$ là nghịch đảo,
do đó có đẳng cấu phạm trù tự nhiên
\[
\boxed{\ \mathbf{Mod}_{K[G]}\ \cong\ \Rep_K(G)\ }.
\]
Diễn giải: cho mỗi biểu diễn $(V,\rho)$, \emph{tác động tuyến tính} của $G$ trên $V$
mở rộng duy nhất thành \emph{tác động $K[G]$} qua tuyến tính; và ngược lại,
một $K[G]$-module quyết định một biểu diễn bằng $g\mapsto(v\mapsto g\cdot v)$.
\end{proof}

\begin{exercise}[Bài tập 9: Biểu diễn nhóm và module trên vành nhóm]
Cho $K$ là một trường và $G$ là một nhóm.  
Kí hiệu $\Rep_K(G)$ là phạm trù các biểu diễn $K$-tuyến tính của $G$, và $\mathbf{Mod}_{K[G]}$ là phạm trù các $K[G]$-module.  
Chứng minh rằng:
\[
\mathbf{Mod}_{K[G]} \;\cong\; \Rep_K(G).
\]
\end{exercise}

\begin{proof}
% \textbf{1. Từ module sang biểu diễn.}  
Với $M\in\mathbf{Mod}_{K[G]}$, định nghĩa ánh xạ
\[
\rho_M:G\to\GL_K(M),\quad \rho_M(g)(m)=g\cdot m.
\]
Vì $g\cdot(h\cdot m)=(gh)\cdot m$ và $e\cdot m=m$, nên $\rho_M$ là đồng cấu nhóm.  
Với đồng cấu module $\varphi:M\to N$, ta có $\varphi(g\cdot m)=g\cdot\varphi(m)$, suy ra $\varphi$ là ánh xạ $G$-tuyến tính.  
Ta được hàm tử
\[
F:\mathbf{Mod}_{K[G]}\to\Rep_K(G),\quad F(M)=(M,\rho_M),\ F(\varphi)=\varphi.
\]

% \textbf{2. Từ biểu diễn sang module.}  
Với $(V,\rho)\in\Rep_K(G)$, định nghĩa tác động của $K[G]$ lên $V$:
\[
\Big(\sum_{g\in G} a_g g\Big)\cdot v := \sum_{g\in G} a_g\,\rho(g)(v).
\]
Tác động này biến $V$ thành $K[G]$-module.

Nếu $T:(V,\rho)\to(W,\sigma)$ thoả mãn $T\rho(g)=\sigma(g)T$ thì
\[
T\!\left(\Big(\sum a_g g\Big)\cdot v\right)=\sum a_g\,\sigma(g)T(v)=\Big(\sum a_g g\Big)\cdot T(v),
\]
nên $T$ là đồng cấu $K[G]$-module. Từ đó ta thu được hàm tử
\[
G:\Rep_K(G)\to\mathbf{Mod}_{K[G]},\quad G(V,\rho)=V,\ G(T)=T.
\]


Với $M\in\mathbf{Mod}_{K[G]}$, $G(F(M))$ có tác động $g\cdot m$ đúng như ban đầu, nên $G(F(M))=M$.  
Với $(V,\rho)\in\Rep_K(G)$, $F(G(V,\rho))=(V,\rho)$ vì $g\cdot v=\rho(g)v$.  
Do đó
\[
F\circ G=\Id_{\Rep_K(G)},\qquad G\circ F=\Id_{\mathbf{Mod}_{K[G]}},
\]
suy ra $F,G$ là đẳng cấu phạm trù.

\[
\boxed{\ \mathbf{Mod}_{K[G]}\ \cong\ \Rep_K(G)\ }.
\]
Nói cách khác, mỗi biểu diễn tuyến tính của $G$ chính là một $K[G]$-module, và ngược lại.
\end{proof}

\begin{exercise}
$K[\mathbb{Z}/n\mathbb{Z}] \cong K[x]/(x^n-1)$

Cho $K$ là một trường (hoặc vành giao hoán có đơn). Chứng minh rằng vành nhóm của nhóm cyclic
$C_n=\mathbb{Z}/n\mathbb{Z}$ đẳng cấu với một vành thương của $K[x]$.
\end{exercise}

\begin{proof}
Gọi $C_n=\langle g\mid g^n=e\rangle$ và $K[C_n]$ là $K$-không gian có cơ sở $\{e,g,\dots,g^{n-1}\}$
với phép nhân $g^i\cdot g^j=g^{i+j \mod n}$.

Xét đồng cấu vành
\[
\varphi:K[x]\longrightarrow K[C_n],\qquad \varphi\!\left(\sum a_i x^i\right)=\sum a_i g^i .
\]

Rõ ràng $\varphi$ là một toàn cấu.

Hơn nữa $\varphi(x^n-1)=g^n-e=0$, nên $(x^n-1)\subseteq\ker\varphi$.

Ngược lại, nếu $f(x)\in\ker\varphi$, viết phép chia Euclid $f=q(x)(x^n-1)+r(x)$ với $\deg r<n$.
Khi đó $0=\varphi(f)=\varphi(r)$, nhưng $\{e,g,\dots,g^{n-1}\}$ độc lập tuyến tính trên $K$,
suy ra $r=0$. Vậy $\ker\varphi=(x^n-1)$.

Áp dụng Định lý đẳng cấu thứ nhất cho vành:
\[
K[x]/(x^n-1)\ \cong\ \mathrm{Im}(\varphi)=K[C_n].
\]
\end{proof}

\begin{remark}
Tổng quát: $K[\mathbb{Z}] \cong K[x,x^{-1}]$ (vành đa thức Laurent) qua $x\leftrightarrow$ phần tử sinh của $\mathbb{Z}$.
\end{remark}

\begin{exercise}[Bài tập 11: Biểu diễn con, phép toán trên chúng và tổng trực tiếp]
Cho $K$ là một trường, $G$ là một nhóm. Với một biểu diễn
\[
(V,\rho),\qquad \rho:G\to\GL_K(V),
\]
đồng nhất $\Rep_K(G)\simeq \mathbf{Mod}_{K[G]}$ qua vành nhóm $K[G]$ (xem Bài tập 9).
Dưới đây là các khái niệm/thuộc tính được \emph{định nghĩa hoàn toàn tương đương} ở hai ngôn ngữ:
\end{exercise}

\begin{enumerate}
\item[\textbf{(1)}] \textbf{Biểu diễn con.}
\begin{enumerate}
  \item \emph{Định nghĩa (ngôn ngữ biểu diễn).} $W\le V$ là \emph{biểu diễn con} của $(V,\rho)$ nếu
  \[
  \rho(g)W\subseteq W\qquad\forall g\in G.
  \]
  Khi đó ta nhận được biểu diễn con $(W,\rho|_W)$.
  \item \emph{Định nghĩa (ngôn ngữ module).} Qua $K[G]$-tác động $g\cdot v:=\rho(g)v$, $W$ là biểu diễn con 
  \(\Longleftrightarrow\) $W$ là \emph{$K[G]$-module con}:
  \[
  K[G]\cdot W\subseteq W\quad\text{(tương đương với $\rho(g)W\subseteq W$ với mọi $g$)}.
  \]
  % \item \emph{Phần sinh bởi một tập con.} Với $S\subseteq V$,
  % \[
  % \langle S\rangle_G:=\mathrm{span}_K\{\rho(g)s: g\in G,\ s\in S\}
  % \;=\;K[G]\cdot S
  % \]
  % là \emph{biểu diễn con nhỏ nhất} chứa $S$.
\end{enumerate}

\item[\textbf{(2)}] \textbf{Các phép toán trên biểu diễn con.}

Cho $(V,\rho)$ và hai biểu diễn con $W_1,W_2\le V$.
\begin{enumerate}
  \item \emph{Giao và tổng:}
  \[
  W_1\cap W_2\ \text{và}\ W_1+W_2:=\{w_1+w_2: w_i\in W_i\}
  \]
  đều là biểu diễn con (vì $\rho(g)$ là tuyến tính và bảo toàn từng $W_i$).
  \item \emph{Ảnh, hạt nhân, thương (ngôn ngữ module).} Với một biến hình biểu diễn
  $T:(V,\rho)\to(W,\sigma)$ (tức $T\rho(g)=\sigma(g)T$), ta luôn có
  \[
  \ker T\ \text{và}\ \im T\ \text{là biểu diễn con},\qquad
  (V/\ker T,\ \bar\rho)\ \simeq\ (\im T,\ \sigma|_{\im T}),
  \]
  trong đó $\bar\rho(g)(v+\ker T):=\rho(g)v+\ker T$. 
  % Biểu đồ ngắn exact:
  % \[
  % \begin{tikzcd}
  % 0 \arrow[r] & \ker T \arrow[r, hook] & V \arrow[r, "T"] & \im T \arrow[r] & 0
  % \end{tikzcd}
  % \]
  % là dãy exact của các biểu diễn (tương đương dãy exact $K[G]$-module).
  % \item \emph{Lưới con.} Tập tất cả biểu diễn con của $(V,\rho)$ là một \emph{lưới} với hai phép toán $\cap,\ +$,
  % giống hệt lưới các $K[G]$-module con của $V$.
\end{enumerate}

\item[\textbf{(3)}] \textbf{Tổng trực tiếp của các biểu diễn con.}
\begin{enumerate}
  \item Với $V=W_1\oplus W_2$ khi và chỉ khi
  \[
  V=W_1+W_2\quad\text{và}\quad W_1\cap W_2=\{0\}.
  \]
  % Khi đó $\rho(g)$ chéo khối theo phân rã $V=W_1\oplus W_2$ và 
  Ta có đẳng cấu biểu diễn
  \[
  (V,\rho)\ \simeq\ (W_1,\rho|_{W_1})\ \oplus\ (W_2,\rho|_{W_2}).
  \]
  \item Với họ $\{W_i\}_{i\in I}$ các biểu diễn con,
  \[
  \bigoplus_{i\in I} W_i \longrightarrow\ V,\qquad (w_i)_{i \in I}\mapsto \sum_i w_i,
  \]
  là \emph{đẳng cấu} khi ảnh bằng $\sum_i W_i$ và các tổng là \emph{trực tiếp}:
  mọi quan hệ $\sum_i w_i=0$ với $w_i\in W_i$ suy ra $w_i=0$ (tức tổng không giao nhau).
\end{enumerate}
\end{enumerate}

% \begin{remark}
% Tất cả mệnh đề trên là các phát biểu quen thuộc của lý thuyết $K[G]$-module, và nhờ
% $\Rep_K(G)\simeq\mathbf{Mod}_{K[G]}$ chúng chuyển nguyên vẹn giữa hai ngôn ngữ.
% \end{remark}

\begin{exercise}[Bài tập 11: Biểu diễn con, phép toán và tổng trực tiếp]
Sử dụng tương ứng $\Rep_K(G)\simeq \mathbf{Mod}_{K[G]}$, hãy định nghĩa rõ ràng các khái niệm:
\begin{itemize}
  \item Biểu diễn con của một biểu diễn.
  \item Các phép toán trên các biểu diễn con.
  \item Tổng trực tiếp của các biểu diễn con.
\end{itemize}
\end{exercise}

\begin{proof}
Xét một biểu diễn $(V,\rho)$ của nhóm $G$ trên $K$, tức là $\rho:G\to\GL_K(V)$.
Thông qua tương ứng $\Rep_K(G)\cong\mathbf{Mod}_{K[G]}$, ta có $V$ là $K[G]$-module với tác động
$g\cdot v:=\rho(g)v$.

\begin{enumerate}
  \item[\textbf{(1)}] \textbf{Biểu diễn con.}
  \begin{enumerate}
    \item Một không gian con $W\le V$ được gọi là \emph{biểu diễn con} của $(V,\rho)$ nếu nó \emph{bất biến dưới tác động của $G$}:
    \[
      \rho(g)W\subseteq W,\qquad \forall g\in G.
    \]
    Khi đó, phép hạn chế $\rho|_W:G\to\GL_K(W)$ cho ta biểu diễn con $(W,\rho|_W)$.
    \item Dưới ngôn ngữ $K[G]$-module, điều kiện trên tương đương:
    \[
      K[G]\cdot W\subseteq W,
    \]
    % tức là $W$ là một \emph{module con} của $V$.
    \item Biểu diễn con nhỏ nhất chứa một tập $S\subseteq V$ là
    \[
      \langle S\rangle_G = \mathrm{span}_K\{\rho(g)s : g\in G,\, s\in S\} = K[G]\cdot S.
    \]
  \end{enumerate}

  \item[\textbf{(2)}] \textbf{Phép toán trên các biểu diễn con.}
  
  Cho hai biểu diễn con $W_1,W_2\le V$, ta có:
  \begin{enumerate}
    \item \emph{Giao và tổng:}
    \[
      W_1\cap W_2\quad\text{và}\quad W_1+W_2=\{w_1+w_2\mid w_i\in W_i\}
    \]
    đều là biểu diễn con của $(V,\rho)$, vì $\rho(g)$ tuyến tính và bảo toàn từng $W_i$.
    \item \emph{Ảnh và hạt nhân:} Nếu $T:(V,\rho)\to(W,\sigma)$ thoả mãn ($T\rho(g)=\sigma(g)T$), thì
    \[
      \ker T \text{ và } \im T \text{ là biểu diễn con},\quad 
      (V/\ker T,\bar\rho)\cong(\im T,\sigma|_{\im T}),
    \]
    với $\bar\rho(g)(v+\ker T):=\rho(g)v+\ker T$.
    % \item \emph{Phần tử bất biến:} Tập $V^G=\{v\in V\mid\rho(g)v=v,\ \forall g\in G\}$ là biểu diễn con.
    % (thực ra là phần tử cố định của $G$).
  \end{enumerate}

  \item[\textbf{(3)}] \textbf{Tổng trực tiếp các biểu diễn con.}
  \begin{enumerate}
    \item Với $W_1,W_2\le V$, ta nói $V=W_1\oplus W_2$ nếu:
    \[
      V=W_1+W_2\quad\text{và}\quad W_1\cap W_2=\{0\}.
    \]
    Khi đó, $\rho(g)$ tác động trên $V=W_1\oplus W_2$, và ta có:
    \[
      (V,\rho)\cong(W_1,\rho|_{W_1})\oplus(W_2,\rho|_{W_2}).
    \]
    \item Tổng quát, với họ $\{W_i\}_{i\in I}$ các biểu diễn con của $V$, tổng trực tiếp là:
    \[
      V = \bigoplus_{i\in I} W_i
      \quad\Longleftrightarrow\quad
      V=\sum_{i\in I} W_i\ \text{ và }\ 
      \sum_{i\ne j} (W_i\cap W_j)=\{0\}.
    \]
    Khi đó mỗi $v\in V$ biểu diễn duy nhất dưới dạng $v=\sum_i w_i$, $w_i\in W_i$.
  \end{enumerate}
\end{enumerate}

% \textbf{Kết luận.}
% Nhờ tương ứng $\Rep_K(G)\simeq\mathbf{Mod}_{K[G]}$, các khái niệm:
% \[
% \text{“biểu diễn con”} \;\leftrightarrow\; \text{“module con”},\quad
% \text{“tổng trực tiếp biểu diễn”} \;\leftrightarrow\; \text{“tổng trực tiếp module”}
% \]
% hoàn toàn tương đương về cấu trúc đại số.
\end{proof}
