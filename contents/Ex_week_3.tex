\section{Tính hoàn toàn khả phân}

\begin{definition}[Module] Cho $R$ là một $\K-$đại số, một $R-$module là một cặp $(\overline{\rho},V)$ với $V$ là một $\K-$không gian vector và $\overline{\rho}: R\to \End_{\K}V$ là một $K-$đại số đồng cấu. 

\begin{remark}
    Một $\K-$đại số đồng cấu là một đồng cấu vành đồng thời cũng là một ánh xạ $\K-$tuyến tính.

    Ta luôn giả sử $\overline{\rho}(1) = \Id_V$ và module như thế được gọi là unital module.
\end{remark}

\end{definition}
\begin{definition}[Không gian con bất biến]
    Một không gian con $W$ của $V$ được gọi là một \textit{không gian con bất biến}(hay biểu diễn con) cho biểu diễn $\rho: G \to \GL(V)$ nếu $\rho(g)W \subset W$ với mọi $g \in G$. 
    
    Tương tự, một không gian con $W$ của $V$ được gọi là một \textit{không gian con bất biến} (hay module con) cho một $R-$module $\overline{\rho}: R \to \End_{\K}(V)$ nếu $\overline{\rho}(r)W \subset W$ với mọi $r \in R$.

    \[\begin{tikzcd}[ampersand replacement=\&,cramped]
	W \&\& W \\
	\\
	V \&\& V
	\arrow[""{name=0, anchor=center, inner sep=0}, "{\rho(g)|_W}", from=1-1, to=1-3]
	\arrow[hook, from=1-1, to=3-1]
	\arrow[hook, from=1-3, to=3-3]
	\arrow[""{name=1, anchor=center, inner sep=0}, "{\rho(g)}"', from=3-1, to=3-3]
	\arrow[between={0.2}{0.8}, dashed, from=1, to=0]
\end{tikzcd}\]
\end{definition}
\begin{definition}[Biểu diễn đơn, module đơn]
Một biểu diễn (hoặc module) được gọi là \textit{đơn} nếu nó không có một không gian con bất biến thực sự nào tầm thường.
\end{definition}
\begin{definition}[Module khả phân hoàn toàn]
    Một $R-$module được gọi là \textit{khả phân hoàn toàn} nếu nó là tổng trực tiếp của các module đơn.    
\end{definition}
\begin{lemma}
    Mọi không gian vector con của một không gian vector $V$ đều có phần bù tuyến tính.
\end{lemma}
\begin{proof}
    Theo bổ đề Zorn, mọi không gian vector đều tồn tại cơ sở, gọi $B_W$ là một cơ sở của không gian con $W \subset V$. Theo định lý mở rộng cơ sở, mọi tập độc lập tuyến tính trong $V$ đều có thể mở rộng thành một cơ sở, nên gọi $B_V\supset B_W$ là một cơ sở của $V$. 
    Đặt $B_U = B_V\setminus B_W$ và $U=\mathcal{L}(U)$ là không gian con sinh bởi $B_U$. Ta sẽ chỉ ra $U$ chính là phần bù tuyến tính của $W$ trong $V$.
    \begin{itemize}
        \item \textbf{Chứng minh $V = W + U$}.

        Lấy bất kỳ $v \in V$, vì $B_V = B_W \cup B_U$ nên tồn tại một biểu thị tuyến tính 
        \[v = \underbrace{\left(\sum{a_iw_i}\right)}_{w \in W} + \underbrace{\left(\sum{b_ju_j}\right)}_{u\in U},\quad w_i \in W,~u_j \in U\]

        \item \textbf{Chứng minh $W \cap U = \{0\}$}.

        Giả sử $v \in W \cap U$, khi đó tồn tại các biểu thị tuyến tính của $v$ qua $B_W$ và $B_U$
        \[\sum{a_iw_i} = v = \sum{b_jw_j}\]
        Dẫn đến $\sum{a_iw_i}-\sum{b_jw_j} = 0$ là một ràng buộc tuyến tính của các vector trong $B_V$ nên ràng buộc này là tầm thường. Do đó $v=0$.
    \end{itemize}

    \begin{remark}
        Phần bù tuyến tính của một không gian con không duy nhất ngoại trừ hai trường hợp tầm thường là $0,V$.
    \end{remark}
\end{proof}
\begin{problem}[Bài tập 3 — Ví dụ về mô đun không khả phân hoàn toàn]
Cho ví dụ về một trường $K$ và một nhóm $G$ cùng với một mô đun trên đại số nhóm $K[G]$ 
mà không khả phân hoàn toàn.
\end{problem}

\begin{solution}
Một ví dụ cổ điển và quan trọng nhất xuất hiện khi đặc trưng của trường $K$ \emph{chia} bậc của nhóm $G$.

\medskip
\textbf{Bước 1. Chọn dữ liệu cơ bản.}
Lấy nhóm hai phần tử
\[
G = \{1, g\}, \qquad g^2 = 1,
\]
và trường $K$ có đặc trưng $2$, tức là $\mathrm{char}(K)=2$.

\medskip
\textbf{Bước 2. Mô tả vành nhóm.}
Đại số nhóm $K[G]$ có cơ sở $\{1, g\}$ và phép nhân được xác định bởi $g^2=1$.
Ta có:
\[
K[G] \;\cong\; K[x]/(x^2 - 1),
\]
với đẳng cấu $x \mapsto g$.
Nhưng trong đặc trưng $2$, $x^2 - 1 = (x - 1)^2$, do đó
\[
K[G] \;\cong\; K[x]/((x-1)^2).
\]
Đây là một \emph{vành không bán đơn}, vì phần tử $(x-1)$ là nilpotent bậc $2$.

\medskip
\textbf{Bước 3. Mô đun trái chính quy.}
Xét $V = K[G]$ như một $K[G]$-mô đun tả với phép nhân trái (mô đun chính quy trái).

Lấy phần tử $v = g - 1$ (hay tương đương $x - 1$ dưới đẳng cấu ở trên).
Đặt
\[
W := K\cdot v = K(g - 1) \subset V.
\]
Khi đó $W$ là một mô đun con bất biến của $V$, vì
\[
g\cdot (g - 1) = g^2 - g = 1 - g = -(g - 1) = g - 1,
\]
do $\mathrm{char}(K)=2$. Như vậy $g$ tác động lên $W$ bằng đồng nhất.

\medskip
\textbf{Bước 4. Kiểm tra khả phân hoàn toàn.}
Ta có dãy ngắn chính xác
\[
0 \longrightarrow W \longrightarrow V \longrightarrow V/W \longrightarrow 0.
\]
Dễ thấy $\dim_K(V) = 2$ và $\dim_K(W) = 1$, nên $V/W$ cũng có chiều $1$.
Tuy nhiên, không tồn tại mô đun con $U$ bất biến nào sao cho $V = W \oplus U$.

Giải thích: 
Nếu $U = Ka$ là mô đun con một chiều của $V$, thì phải có $g\cdot a = \lambda a$ với $\lambda \in K^\times$.
Tức là $a$ phải là vectơ riêng của toán tử $g$.
Trong cơ sở $\{1, g-1\}$, tác động của $g$ được cho bởi ma trận
\[
[g]_{\{1,\,g-1\}} \;=\;
\begin{pmatrix}
1 & 1 \\[4pt]
0 & 1
\end{pmatrix},
\]
một ma trận Jordan không chéo hoá được. 
Do đó $g$ không có vectơ riêng nào ngoài các bội của $g-1$, nghĩa là $W$ là không gian bất biến duy nhất.
Vì thế không có phần bù bất biến nào của $W$ trong $V$.

\medskip
\textbf{Bước 5. Kết luận.}
Vậy $V = K[G]$ (với $\mathrm{char}(K)=2$ và $G=\{1,g\}$) là một mô đun không khả phân hoàn toàn.
Hay nói cách khác, dãy ngắn chính xác
\[
0 \longrightarrow W \longrightarrow V \longrightarrow V/W \longrightarrow 0
\]
không tách được trong phạm trù các $K[G]$-mô đun.

\end{solution}

\begin{remark}
Ví dụ này cho thấy rằng định lý Maschke không còn đúng nếu $\mathrm{char}(K)$ chia $|G|$.
Trong trường hợp $\mathrm{char}(K)$ \emph{không chia} $|G|$, thì mọi $K[G]$-mô đun đều khả phân hoàn toàn (bán đơn),
tức là $K[G]$ là một đại số bán đơn.
\end{remark}

\begin{theorem}
Với một $R$-module $V$, các mệnh đề sau đây là tương đương:
\begin{enumerate}
    \item Mọi module con $U \le V$ đều có bù bất biến trong $V$ (tức tồn tại $W \le V$ sao cho $V = U \oplus W$).
    \item Mọi dãy ngắn chính xác $0 \to U \xrightarrow{i} V \xrightarrow{\pi} V/U \to 0$ đều chẻ ra.
    \item $V$ là tổng trực tiếp của các module con đơn, tức
    \[
        V = \bigoplus_{i \in I} S_i, \quad S_i \text{ đơn.}
    \]
\end{enumerate}
\end{theorem}

\begin{proof}
\noindent\textbf{(1) $\Rightarrow$ (2).}
Giả sử mọi module con có bù bất biến.  

Xét dãy \[0 \to U \xrightarrow{i} V \xrightarrow{\pi} V/U \to 0.\]  
Vì $U$ có bù bất biến $W$, ta có $V = U \oplus W$ và $\pi|_W : W \to V/U$ là đẳng cấu.  

Đặt $s = (\pi|_W)^{-1}$, khi đó $\pi \circ s = \operatorname{id}_{V/U}$, nên dãy tách.

\medskip
\noindent\textbf{(2) $\Rightarrow$ (1).}
Giả sử mọi dãy ngắn chính xác tách.  
Lấy $U \le V$. Khi đó $0 \to U \xrightarrow{i} V \xrightarrow{\pi} V/U \to 0$ tách, nên tồn tại $s: V/U \to V$ sao cho $\pi \circ s = \operatorname{id}_{V/U}$.  
Đặt $W = \operatorname{Im}s$, ta có $V = U \oplus W$.

\medskip
\noindent\textbf{(1) $\Rightarrow$ (3).}
Giả sử mọi module con có bù bất biến.  
Đặt $V_0 = \sum_{S \le V,\; S \text{ đơn}} S$. Khi đó $V_0$ bất biến, nên tồn tại $U$ bất biến sao cho $V = V_0 \oplus U$.  
Nếu $U \ne 0$, lấy $u \in U \setminus \{0\}$, khi đó $R \cdot u$ có module con cực đại $M$, và $R \cdot u/M$ đơn.  
Suy ra tồn tại module đơn $S_u \subset U$, mâu thuẫn với $U \cap V_0 = 0$.  
Do đó $U = 0$, nên $V = V_0$.  
Áp dụng bổ đề Zorn, ta chọn họ $\{S_i\}_{i \in I}$ các module đơn sao cho $V = \bigoplus_{i \in I} S_i$.

\medskip
\noindent\textbf{(3) $\Rightarrow$ (1).}
Giả sử $V = \bigoplus_{i \in I} S_i$ với mỗi $S_i$ đơn.  
Lấy $U \le V$. Khi đó $U = \bigoplus_{i \in J} S_i$ với $J \subseteq I$.  
Đặt $W = \bigoplus_{i \notin J} S_i$, ta được $V = U \oplus W$.

\medskip
Kết luận: ba mệnh đề trên tương đương. Khi đó $V$ được gọi là \textit{module khả phân hoàn toàn} (hay \textit{semisimple module}).
\end{proof}



\begin{problem}[BT6]
    Giả sử mọi không gian con bất biến của một $R-$module $V$ đều có một bù bất biến. Gọi $W$ là một không gian con bất biến của $V$. Chứng minh rằng mọi không gian con bất biến của $W$ đều có một bù bất biến trong $W$ và mọi không gian con bất biến của $V/W$ đều có một bù bất biến trong $V/W$.
\end{problem}

\begin{proof}
    Giả sử $W_1$ là một không gian con bất biến của $W$, với mọi $r \in R$ thì $\overline{\rho}(r)W_1 \subset W_1$, kết hợp $W_1$ cũng là một không gian con của $V$ nên $W_1$ là bất biến trong $V$. Theo giả thiết $V = W_1 \oplus U$ với $U$ là bù bất biến của $W_1$ trong $V$.

    Ta có
    \[W = W \cap V = W \cap (W_1 \cup U)= (W \cap W_1) \cup (W\cap U)=W_1 \cup (W \cap U).\]

    Khi đó $W_2:= W\cap U$ là một không gian con bù bất biến của $W_1$ trong $W$ vì

    \begin{itemize}
        \item $W_2$ hiển nhiên là một không gian con của $W$. Hơn nữa, $W_2$ cũng bất biến vì $\forall w \in W \cap U,~r \in R$ thì $\overline{\rho}(r)w \in W \cap U$ (do $W,U$ bất biến trong $V$).

        \item $W = W_1 + W_2$. Thật vậy, $\forall w \in W \subset V = W_1 \oplus U,~\exists!~w_1 \in W_1,~u \in U:~w = w_1 + u$. 
        Suy ra $U \ni u = w - w_1 \in W$ nên $u \in W \cap U = W_2$. Hay $W = W_1 + W_2$.

        \item $W_1 \cap W_2 = W_1 \cap (W \cap U) = (W_1 \cap W) \cap U = W_1 \cap U = \{0\}$.

        \item Do đó $W = W_1 \oplus W_2$.
    \end{itemize}

    Tiếp theo, giả sử $A$ là một không gian con bất biến của $V/W$ và $\pi: V\to V/W$ là phép chiêú chính tắc. Rõ ràng $\pi$ là một đồng cấu $R-$module và tác động của $R$ lên $V/W$ xác định bởi
    \[\overline{\rho}_{V/W}(r)[v] = [\overline{\rho}_V(r)(v)].\]

    Đặt $U = \pi^{-1}(A)$. Khi đó $U$ mà một không gian con của $V$. Hơn nữa, $U$ là bất biến trong $V$ vì $\forall u \in U,~r\in R$ thì
    \[[\overline{\rho}_V(r)u] = \overline{\rho}_{V/W}(r)([u])=\overline{\rho}_{V/W}(r)(\underbrace{\pi(u)}_{\in A}) \in A \quad (\text{vì $A$ bất biến})\]
    nên $\overline{\rho}(r)u \in \pi^{-1}(A) = U$. Khi đó, theo giả thiết $V = U \oplus U_1$, với $U_1$ là bù bất biến của $U$.

    Ta sẽ chỉ ra $\pi(U_1)$ là một bù bất biến của $A$ trong $V/W$. Thật vậy
    \begin{itemize}
        \item $\pi(U_1)$ là một không gian con bất biến trong $V/W$ vì $\forall [u_1]=\pi(u_1) \in \pi(U_1),~r\in R$ thì 
        \[\overline{\rho}_{V/W}(r)[u_1]=[\underbrace{\overline{\rho}_V(r)(u_1)}_{\in U_1 \text{ vì $U_1$ bất biến}}] \in \pi(U_1)\]

        \item $V/W = A + \pi(U_1)$. Thật vậy, với mọi $v \in V = U \oplus U_1,~\exists!~u\in U,~u_1 \in U_1: v = u + u_1$. Suy ra $[v] = [u] + [u_1] \in \pi(U) + \pi(U_1)=A + \pi(U_1)$.

        \item $[v] \in A \cap \pi(U_1)$ thì $v \in \pi^{-1}(A)=U,~[v]=\pi(u_1)=[u_1]~(\text{ với }u_1\in U_1)$. Suy ra 
        \[u:= v-u_1 \in \ker(\pi)=W = \pi^{-1}(0) \subset \pi^{-1}(A) = U.\]
        Nên $U_1 \ni u_1 = v-u \in U$. Vậy $v- u \in U \cap U_1 = \{0\}$, tức $u_1=0$, tức $[v]=[0]$.

        \item Do đó $V/W = A \oplus \pi(U_1)$.
    \end{itemize}
\end{proof}

 \begin{problem}[BT4]
    Chứng minh rằng một $R$-module là khả phân hoàn toàn khi và chỉ khi mọi không gian con bất biến đều có một bù bất biến.
\end{problem}
\begin{proof}
    Giả sử $(\overline{\rho},V)$ là một $R$ module và $W \leq V$ là một không gian con bất biến.
    \begin{enumerate}
        \item $(\Longrightarrow)$ Giả sử $V$ là khả phân hoàn toàn, tức $V = \bigoplus_{i \in I}S_i$, với $S_i$ là các module đơn. Ta sẽ chỉ ra $W$ tồn tại $U \leq V$ bất biến và $V = W \oplus U$.

        Xét $\mathcal{F} = \{U \leq V\mid U \text{ bất biến},~ W \cap U = \{0\}\}$, ta có $\mathcal{F} \neq \emptyset$ (vì chứa không gian $0$) và có trang bị quan hệ thứ tự là quan hệ bao hàm thức nên theo bổ đề Zorn, $\mathcal{F}$ có ít nhất một phần tử cực đại. Gọi $U_0$ là một phần tử cực đại của $\mathcal{F}$ và đặt $V_0 = U_0 \oplus W$. 
        
        Khi đó $V = V_0$. Thật vậy, giả sử phản chứng $V_0 \neq V$, khi đó tồn tại $S_i \nsubseteq  V_0$. 
        
        Lại có $S_i \cap V_0 \leq V_0$ và $S_i$ đơn nên $S_i \cap V_0 = \{0\}$ (vì nếu $S_i \cap V_0=S_i$ thì $S_i \subset V_0$).

        % Ta có
        % \[\{0\}=S_i \cap V_0 = S_i \cap (U_0 \cup W)=(S_i \cap U_0) \cup (S_i \cap W)\]
        % nên \[S_i \cap U_0 = \{0\} = S_i \cap W.\]

        Đặt $U'=U_0 \oplus S_i \leq V$ thì $U'$ bất biến và $U_0 \subsetneq U'$.

        Tuy nhiên, $U' \in \mathcal{F}$ vì $U' \cap W = \{0\}$. 
        
        (Thật vậy, giả sử $x \in U'\cap W=(U_0 \oplus S_i) \cap W$ thì $x=u_0 + s_i$ với $u_0 \in U_0,~s_i \in S_i$. Suy ra $S_i \ni s_i = x-u_0 \in V_0$ (vì $x \in W \subset V_0,~u_0 \in U_0 \in V_0$). Do đó $s_i \in V_0 \cap S_i = \{0\}$, tức $s_i = 0$. Hay là $x=u_0 \in W \cap U_0 = \{0\}$, tức $x=0$).

        Suy ra mâu thuẫn với tính cực đại của $U_0$. Do đó $V = W \oplus U_0$ với $U_0$ bất biến.


        \item $(\Longleftarrow)$ Giả sử mọi không gian con bất biến của $V$ đều có bù bất biến. Ta sẽ chỉ ra $V$ là khả phân hoàn toàn.
        
        Đặt 
        \[V_0 = \sum_{S \leq V,~S \text{ đơn}} S\]
        Ta có $V_0$ bất biến nên theo giả thiết tồn tại $U \leq V$ bất biến sao cho $V = V_0 \oplus U$. 

        Ta sẽ chứng minh $V = V_0$ bằng cách chỉ ra $U=\{0\}$. Thật vậy, giả sử phản chứng tồn tại $u \in U\setminus\{0\}$. Khi đó $R\cdot u \leq U$, mà $U$ cũng bất biến nên $R\cdot u$ cũng bất biến. Gọi $M$ là một module con cực đại của $R\cdot u$, ta viết được $R\cdot u = M \oplus S_u$ với $S_u$ là bù bất biến của $M$ trong $R\cdot u$. Ta có $S_u \cong R\cdot u/M$ là đơn (vì $M$ cực đại trong $R\cdot u$). Vậy $S_u \subset U \subset V$ và đơn, nên $S_u \subset V_0$. Do đó $S_u \subset U \cap V_0 = 0$ tức $S_u = 0$, mâu thuẫn với $S_u$ đơn. Dẫn đến giả thiết phản chứng là sai, và do đó 
        \[V=V_0 = \sum_{S \leq V,~S \text{ đơn}} S\]

        Tiếp theo ta chỉ ra tổng trên là một tổng trực tiếp.

        Xét tập $\mathcal{G} = \{J \subset I\mid \sum_{j \in J}S_j \text{ là một tổng trực tiếp}\}$. Rõ ràng $\mathcal{G} \neq \emptyset$ (vì $i \in \mathcal{G}$ với mọi $i \in I$) và được trang bị quan hệ thứ tự bao hàm, do đó theo bổ đề Zorn, $\mathcal{G}$ tồn tại cực đại $J_{\max} \subset I$ sao cho \[V_1:=\sum_{j \in J_{\max}}S_j = \bigoplus_{j \in J_{\max}}S_j.\]
        
        Nếu $V_1 \subsetneq V$, theo giả thiết thì $V=V_1 \oplus U_1$ với $0 \neq U_1 \leq V$ bất biến. Khi đó, chứng minh tương tự như trên thì $\exists u_1 \in U_1\setminus\{0\}$ mà $R\cdot u_1$ chứa một module con đơn $S_{u_1}$. 
        
        Ta có $S_{u_1} \subset U_1$ và $S_{u_1}$ đơn nên $S_{u_1} \subset V_0= V$. Hơn nữa, $S_{u_1} \cap V_1 \leq S_{u_1}$ nên $S_{u_1} \cap V_1 = 0$ (vì nếu $S_{u_1} \cap V_1 = S_{u_1}$ thì $S_{u_1} \subset V_1$, dẫn đến $S_{u_1} \subset U_1 \cap V_1 = 0$).

        Do đó $S_{u_1} + V_1 = S_{u_1} \oplus V_1$ là một tổng trực tiếp các module đơn, mâu thuẫn với tính cực đại của $J_{\max}$.

        Vì vậy $V_1 = V$, tức $V$ là một tổng trực tiếp các module đơn, hay $V$ khả phân hoàn toàn.
        
    \end{enumerate}
\end{proof}

\begin{problem}[BT 5 — Ví dụ về mô đun con không có phần bù bất biến]
Cho một trường $K$, một nhóm $G$, và một $K[G]$-mô đun $V$,
hãy cho ví dụ về một mô đun con bất biến $W \subset V$
sao cho $W$ không có phần bù bất biến trong $V$.
\end{problem}

\begin{solution}
Ta sẽ xây dựng ví dụ cụ thể đơn giản nhất cho hiện tượng này.

\medskip
\textbf{Bước 1.} 
Chọn nhóm $G$ là nhóm hai phần tử:
\[
G = \{1,\, g\}, \qquad g^2 = 1.
\]
Xét trường $K$ có đặc trưng $2$, tức là $\mathrm{char}(K)=2$.
Khi đó vành nhóm $K[G]$ có dạng
\[
K[G] = K\{1,g\} \cong K[x]/(x^2-1).
\]
Nhưng trong đặc trưng $2$, ta có $x^2-1=(x-1)^2$, nên
\[
K[G] \;\cong\; K[x]/((x-1)^2).
\]
Đây là một vành không bán đơn (vì phần tử $x-1$ là nilpotent bậc $2$).

\medskip
\textbf{Bước 2.} 
Xét $V=K[G]$ như một $K[G]$-mô đun tả (theo phép nhân trái).
Gọi $W=(x-1)K[G]$ là mô đun con sinh bởi phần tử $x-1$.
Ta có $W = K(x-1)$, vì $(x-1)^2=0$.

\medskip
\textbf{Bước 3.} 
Ta chứng minh rằng $W$ không có phần bù bất biến trong $V$.

Giả sử tồn tại $K[G]$-mô đun con $U$ của $V$ sao cho
\[
V = W \oplus U.
\]
Do $\dim_K(V)=2$ và $\dim_K(W)=1$, ta có $\dim_K(U)=1$.
Mọi mô đun con một chiều của $V$ là dạng $U = Ka$ với $a \in V$.
Ta phải có $g\cdot a = \lambda a$ với một $\lambda\in K^\times$, 
nghĩa là $a$ là vectơ riêng của phép nhân bởi $g$.

Tuy nhiên, với $\mathrm{char}(K)=2$, ma trận hành động của $g$ trên cơ sở $\{1, x-1\}$ là
\[
[g]_{\{1,\,x-1\}} = 
\begin{pmatrix}
1 & 1 \\[4pt]
0 & 1
\end{pmatrix}.
\]
Đây là ma trận Jordan có duy nhất giá trị riêng $1$, nhưng không chéo hóa được.
Do đó không tồn tại vectơ riêng $a$ độc lập với $x-1$.
Nói cách khác, không có mô đun con $1$-chiều nào $U$ khác $W$ bất biến theo $G$.

\medskip
\textbf{Bước 4.} 
Vì thế, $W=(x-1)K[G]$ là mô đun con bất biến trong $V=K[G]$,
nhưng không có phần bù bất biến trong $V$.
Điều này minh họa rằng khi đặc trưng của trường chia cho bậc của nhóm (ở đây $2\mid |G|$),
thì vành nhóm $K[G]$ không còn bán đơn,
và các mô đun $K[G]$ không nhất thiết khả phân hoàn toàn.

\end{solution}

\begin{remark}
Nếu ta thay trường $K$ bởi trường có đặc trưng không chia $|G|$ (ví dụ $\mathbb{C}$, $\mathbb{Q}$, hoặc $\mathbb{R}$),
thì định lý Maschke bảo đảm rằng mọi $K[G]$-mô đun đều khả phân hoàn toàn,
và không thể xảy ra ví dụ như trên.
Vì vậy ví dụ này phản ánh \emph{sự thất bại của Định lý Maschke} trong trường hợp $\mathrm{char}(K)$ chia $|G|$.
\end{remark}


\begin{problem}[BT6 — Phần bù bất biến trong mô đun khả phân hoàn toàn]
Giả sử mọi không gian con bất biến của một $R$-mô đun $V$ đều có một phần bù bất biến.
Gọi $W$ là một không gian con bất biến của $V$. 
Chứng minh rằng:
\begin{enumerate}
\item Mọi không gian con bất biến của $W$ đều có một phần bù bất biến trong $W$.
\item Mọi không gian con bất biến của $V/W$ đều có một phần bù bất biến trong $V/W$.
\end{enumerate}
\end{problem}

\begin{proof}
\textbf{(a)} 
Giả sử $U$ là một không gian con bất biến của $W$. 
Vì $U \subseteq W \subseteq V$ và $U$ là bất biến trong $V$, theo giả thiết mọi không gian con bất biến của $V$ đều có phần bù bất biến trong $V$, nên tồn tại một không gian con $U'$ bất biến của $V$ sao cho
\[
V = U \oplus U'.
\]
Do $W$ là bất biến trong $V$, lấy giao với $W$ ta được:
\[
W = (U \oplus U') \cap W = (U \cap W) \oplus (U' \cap W) = U \oplus (U' \cap W).
\]
Hơn nữa, $U' \cap W$ cũng là bất biến trong $W$ (vì $U'$ và $W$ đều bất biến trong $V$). 
Vậy $U' \cap W$ là phần bù bất biến của $U$ trong $W$, như cần chứng minh.

\medskip
\textbf{(b)} 
Gọi $\pi:V\to V/W$ là ánh xạ chuẩn. 
Giả sử $\overline{U}$ là một không gian con bất biến của $V/W$. 
Khi đó, lấy tiền ảnh $U=\pi^{-1}(\overline{U})$ ta có $W\subseteq U\subseteq V$ và $U$ là bất biến trong $V$.
Theo giả thiết, tồn tại $U'$ bất biến trong $V$ sao cho
\[
V = U \oplus U'.
\]
Xét ảnh qua $\pi$ ta có $\pi(U)=\overline{U}$ và $\pi(U')\cong U'/(U'\cap W)$. 
Do $U'$ bất biến và $W$ bất biến nên $\pi(U')$ cũng là bất biến trong $V/W$. 
Ta có
\[
V/W = (U\oplus U')/W \cong (U/W) \oplus (U'/(U'\cap W))
     = \overline{U} \oplus \pi(U').
\]
Như vậy $\pi(U')$ là phần bù bất biến của $\overline{U}$ trong $V/W$, chứng minh xong.

\medskip
Kết luận: cả (a) và (b) đều đúng.
\end{proof}

\begin{corollary}[Hệ quả 7]
Mô đun con và mô đun thương của mô đun khả phân hoàn toàn cũng là mô đun khả phân hoàn toàn.
\end{corollary}

\begin{proof}
Giả sử $V$ là mô đun khả phân hoàn toàn, tức là mọi không gian con bất biến của $V$ đều có phần bù bất biến.
Nếu $W$ là mô đun con bất biến, theo phần (a) của Bài tập~6, mọi không gian con bất biến của $W$ đều có phần bù bất biến trong $W$, do đó $W$ là mô đun khả phân hoàn toàn.
Tương tự, theo phần (b), mọi không gian con bất biến của $V/W$ đều có phần bù bất biến trong $V/W$, nên $V/W$ cũng là mô đun khả phân hoàn toàn.
\end{proof}


\begin{problem}[BT 8]
    Chứng minh rằng nếu $R$-module chính quy trái là khả phân hoàn toàn thì mọi $R-$module đều khả phân hoàn toàn.
\end{problem}
\begin{definition}
Với mỗi $r \in R$, kí hiệu $L(r):R \to R,~x\mapsto rx$. Tập các $L(r)$ này cho $R$ thành một $R-$module và được gọi là $R$-\textit{chính quy trái}.
\end{definition}
\begin{proof}
    Giả sử $R-$module chính qui trái $L(R)$ là khả phân hoàn toàn, ta cần chỉ ra mọi $R-$module $M$ là khả phân hoàn toàn.

    Vì mọi module đểu là thương của một module tự do, nên giả sử
    \[R^{(I)} \xrightarrow{\pi} M \to 0\]
    Vì $L(R)$ là khả phân hoàn toàn nên $R^{(I)}$ cũng là khả phân hoàn toàn vì tổng các module khả phân hoàn toàn cũng khả phân hoàn toàn). Do đó $M = R^{(I)}/\ker(\pi)$ cũng là khả phân hoàn toàn vì ta có bổ đề sau
    \begin{lemma}
        Nếu $V$ khả phân hoàn toàn, $U$ là thương của $V$ thì $U$ cũng khả phân hoàn toàn.
    \end{lemma}
    \begin{proof}
        Vì $V$ khả phân hoàn toàn và $U$ là một module thương của $V$. Tức tồn tại không con bất biến $A \leq V$ sao cho $U \cong V/A$. Mà $A$ có bù bất biến $A'$ nên 
        \[U \cong V/A= (A \oplus A')/A \cong A'\]
        Do đó $U$ bất biến.
    \end{proof}
\end{proof}

\begin{problem}[Bài tập 9 — Tính duy nhất các số bội trong phân tích bán đơn]
Giả sử $V$ là một $R$-mô đun khả phân hoàn toàn (bán đơn) hữu hạn chiều và có một phân tích
\begin{equation}\label{eq:decomp}
V \;\cong\; V_1^{\oplus m_1}\;\oplus\; V_2^{\oplus m_2}\;\oplus\; \cdots \;\oplus\; V_r^{\oplus m_r},
\end{equation}
trong đó $V_1,\dots,V_r$ là các $R$-mô đun đơn đôi một không đẳng cấu. Cho $S$ là một $R$-mô đun đơn bất kì. Chứng minh:

\begin{enumerate}
\item[(a)] Nếu không tồn tại $k$ sao cho $V_k \cong S$ thì $\Hom_R(S,V)=\{0\}$.
\item[(b)] Nếu tồn tại $k$ sao cho $V_k \cong S$ thì $\Hom_R(S,V)\neq 0$ và hơn nữa
\[
\dim_K \Hom_R(S,V)\;=\; m_k\,\dim_K \End_R(S).
\]
Tương đương,
\[
\Hom_R(S,V)\;\cong\; \End_R(S)^{\oplus m_k}
\quad \text{với tư cách là không gian vectơ trên }K.
\]
\end{enumerate}

Suy ra: trong mọi phân tích bán đơn \eqref{eq:decomp} của $V$, các mô đun đơn $V_k$ xuất hiện (lên tới đẳng cấu) và các số bội $m_k$ là duy nhất (chỉ có thể thay đổi thứ tự các hạng tử).
\end{problem}

\begin{proof}
Trước hết, dùng hai sự kiện chuẩn:

\begin{itemize}
\item[(F1)] (Tính phân tích của hàm tử $\Hom$ theo tổng trực tiếp) Với mọi họ mô đun $(M_i)_{i\in I}$,
\[
\Hom_R\!\left(S,\;\bigoplus_{i\in I} M_i\right)\;\cong\;\bigoplus_{i\in I}\Hom_R(S,M_i),
\]
đẳng cấu tự nhiên bởi $f\mapsto (\pi_i\circ f)_i$, với $\pi_i$ là phép chiếu toạ độ.
\item[(F2)] (Bổ đề Schur) Nếu $S,T$ là $R$-mô đun đơn thì
\[
\Hom_R(S,T)=
\begin{cases}
0, & \text{khi } S \not\cong T,\\[2mm]
\End_R(S), & \text{khi } S \cong T.
\end{cases}
\]
Ở đây $\End_R(S)$ là một vành chia (nên là không gian vectơ trên $K$ nếu $K$ nằm trong tâm của $\End_R(S)$).
\end{itemize}

Áp dụng (F1) cho \eqref{eq:decomp} và lặp lại theo bội, ta có
\begin{align*}
\Hom_R(S,V)
&\cong \bigoplus_{i=1}^r \Hom_R\!\left(S,\,V_i^{\oplus m_i}\right)
 \;\cong\; \bigoplus_{i=1}^r \Hom_R(S,V_i)^{\oplus m_i}.
\end{align*}
Nhờ (F2), mọi hạng $\Hom_R(S,V_i)$ đều bằng $0$ nếu $V_i\not\cong S$, còn nếu $V_k\cong S$ thì
$\Hom_R(S,V_k)\cong \End_R(S)$. Do đó
\[
\Hom_R(S,V)\;\cong\; \End_R(S)^{\oplus m_k},
\]
trong đó $k$ là chỉ số (duy nhất) sao cho $V_k\cong S$ nếu có, còn nếu không có chỉ số nào như vậy thì toàn bộ tổng trực tiếp ở vế phải là $0$.

Từ đây suy ra ngay:

\begin{itemize}
\item (a) Nếu không có $k$ với $V_k\cong S$, vế phải bằng $0$, do đó $\Hom_R(S,V)=\{0\}$.
\item (b) Nếu có $k$ với $V_k\cong S$, thì $\Hom_R(S,V)\cong \End_R(S)^{\oplus m_k}\neq 0$ và
\[
\dim_K \Hom_R(S,V)\;=\; m_k\,\dim_K \End_R(S).
\]
\end{itemize}

\emph{Hệ quả về tính duy nhất của số bội.}
Công thức trên cho thấy
\[
m_k \;=\; \dim_{\End_R(S)} \Hom_R(S,V),
\]
nên $m_k$ chỉ phụ thuộc vào $V$ và lớp đẳng cấu của $S$ (tức $V_k$), không phụ thuộc vào cách chọn phân tích \eqref{eq:decomp}. Vì các lớp đẳng cấu $V_k$ là đại diện cho các mô đun đơn xuất hiện trong $V$, việc có hay không có $V_k\cong S$ cũng được đặc trưng bởi $\Hom_R(S,V)$ có khác $0$ hay không, nên tập các mô đun đơn xuất hiện (lên tới đẳng cấu) trong mọi phân tích của $V$ là như nhau. Đây chính là tính duy nhất (lên tới hoán vị) của phân tích bán đơn và tính duy nhất của các số bội.
\end{proof}


\begin{problem}[BT13 — Mô đun trên tổng trực tiếp các đại số]
Giả sử $R_1,\dots,R_k$ là các $K$-đại số kết hợp có đơn vị $1_i$ (với $i=1,\dots,k$) và đặt
\[
R \;:=\; R_1 \oplus R_2 \oplus \cdots \oplus R_k,
\]
trong đó phép nhân theo toạ độ và đơn vị của $R$ là $1=\sum_{i=1}^k 1_i$ (các $1_i$ là các idempotent trực giao trung tâm).
\begin{enumerate}
\item[(a)] Nếu với mỗi $i$, $(\widetilde{\rho}_i,M_i)$ là một $R_i$-mô đun có đơn vị (nghĩa là $\widetilde{\rho}_i(1_i)=\mathrm{id}_{M_i}$), hãy chứng minh rằng
\[
M \;:=\; M_1 \oplus M_2 \oplus \cdots \oplus M_k
\]
trở thành một $R$-mô đun có đơn vị bởi tác động
\[
\widetilde{\rho}(r_1+\cdots+r_k)\;:=\; \widetilde{\rho}_1(r_1)\;+\;\widetilde{\rho}_2(r_2)\;+\;\cdots\;+\;\widetilde{\rho}_k(r_k)
\quad\text{trên }M.
\]
\item[(b)] Ngược lại, nếu $(\widetilde{\rho},M)$ là một $R$-mô đun có đơn vị, đặt
\[
M_i\;:=\; \widetilde{\rho}(1_i)M \qquad (i=1,\dots,k).
\]
Chứng minh rằng mỗi $M_i$ là một $R_i$-mô đun có đơn vị với tác động $r_i\cdot m:=\widetilde{\rho}(r_i)m$, và phép cộng trực tiếp tự nhiên cho ta đẳng thức các mô đun
\[
M \;\cong\; M_1 \oplus M_2 \oplus \cdots \oplus M_k .
\]
\item[(c)] Suy ra phép gán
\[
(M,\widetilde{\rho}) \longmapsto (M_1,\dots,M_k),\qquad
(M_1,\dots,M_k) \longmapsto (M,\widetilde{\rho})
\]
thiết lập một tương đương thể loại
\[
\mathrm{Mod}_R \;\simeq\; \prod_{i=1}^k \mathrm{Mod}_{R_i}.
\]
Cụ thể, với mọi $R$-mô đun $M,N$, chứng minh đẳng cấu tự nhiên
\[
\operatorname{Hom}_R(M,N) \;\cong\; 
\bigoplus_{i=1}^k \operatorname{Hom}_{R_i}(M_i,N_i),
\]
và mô tả hai hàm tử nghịch đảo tới nhau trên cả đối tượng lẫn cấu xạ.
\end{enumerate}
\end{problem}

\begin{proof}
\textbf{Chuẩn bị.} Trong $R= \bigoplus_{i=1}^k R_i$, các phần tử $1_i$ là các idempotent trực giao trung tâm:
\[
1_i^2=1_i,\qquad 1_i1_j=0\ (i\neq j),\qquad 1_i r = r 1_i \ \forall r\in R.
\]
Hơn nữa, với mọi $r=(r_1,\dots,r_k)\in R$ ta có $r=\sum_{i=1}^k r_i$ và $r_i=1_i r = r 1_i$.

\smallskip
\emph{(a) Xây dựng tác động theo toạ độ.}
Cho $M:=\bigoplus_{i=1}^k M_i$ và định nghĩa $\widetilde{\rho}$ như trong đề bài. Với $r=(r_1,\dots,r_k)$, $s=(s_1,\dots,s_k)\in R$ và $m=(m_1,\dots,m_k)\in M$:
\[
\widetilde{\rho}(r)\bigl(\widetilde{\rho}(s)m\bigr)
= \sum_{i=1}^k \widetilde{\rho}_i(r_i)\!\left(\sum_{j=1}^k \widetilde{\rho}_j(s_j)m_j\right)
= \sum_{i=1}^k \widetilde{\rho}_i(r_i)\bigl(\widetilde{\rho}_i(s_i)m_i\bigr)
= \sum_{i=1}^k \widetilde{\rho}_i(r_is_i)m_i
= \widetilde{\rho}(rs)m,
\]
nơi đẳng thức thứ hai dùng trực giao theo toạ độ và đẳng thức thứ ba dùng tính kết hợp mô đun của từng $(M_i,\widetilde{\rho}_i)$.
Tính tuyến tính theo $K$ và theo từng biến hiển nhiên.
Lại có
\[
\widetilde{\rho}(1)m = \sum_{i=1}^k \widetilde{\rho}_i(1_i)m_i
= \sum_{i=1}^k m_i = m,
\]
vì $\widetilde{\rho}_i(1_i)=\mathrm{id}_{M_i}$. Do đó $(M,\widetilde{\rho})$ là một $R$-mô đun có đơn vị.

\smallskip
\emph{(b) Phân rã bằng các idempotent trung tâm.}
Cho $M$ là $R$-mô đun có đơn vị. Đặt $e_i:=\widetilde{\rho}(1_i)\in \End_K(M)$. Khi đó $e_i$ là các idempotent trực giao, $e_i e_j=0$ với $i\neq j$, và $\sum_{i=1}^k e_i=\widetilde{\rho}(1)=\mathrm{id}_M$.
Định nghĩa $M_i:=e_i M$. Với $r_i\in R_i$ và $m\in M_i$ (nghĩa là $m=e_i m$), đặt $r_i\cdot m := \widetilde{\rho}(r_i)m$.
Nếu $s_i\in R_i$ thì
\[
r_i\cdot (s_i\cdot m)= \widetilde{\rho}(r_i)\widetilde{\rho}(s_i)m
= \widetilde{\rho}(r_is_i)m
= (r_is_i)\cdot m,
\]
và $1_i\cdot m=\widetilde{\rho}(1_i)m=e_i m=m$. Do đó $M_i$ là $R_i$-mô đun có đơn vị.

Ta chứng minh $M=\bigoplus_{i=1}^k M_i$. Với mọi $m\in M$,
\[
m=\widetilde{\rho}(1)m=\sum_{i=1}^k \widetilde{\rho}(1_i)m=\sum_{i=1}^k e_i m \in \sum_i M_i.
\]
Nếu $\sum_i m_i=0$ với $m_i\in M_i$, nhân $e_j$ cho hai vế được $m_j=0$ (do $e_j m_i=0$ khi $i\neq j$), nên tổng là trực tiếp.

\smallskip
\emph{(c) Tương đương thể loại.}
Định nghĩa hai hàm tử:
\[
\mathcal{F}:\mathrm{Mod}_R \to \prod_{i=1}^k \mathrm{Mod}_{R_i},\quad
\mathcal{F}(M)=\bigl(\widetilde{\rho}(1_1)M,\dots,\widetilde{\rho}(1_k)M\bigr),
\]
và với cấu xạ $f:M\to N$ là $R$-tuyến tính,
\[
\mathcal{F}(f)=(f|_{M_1},\dots,f|_{M_k}),\quad f(M_i)\subseteq N_i \text{ vì } f\circ \widetilde{\rho}(1_i)=\widetilde{\sigma}(1_i)\circ f.
\]
Ngược lại,
\[
\mathcal{G}: \prod_{i=1}^k \mathrm{Mod}_{R_i}\to \mathrm{Mod}_R,\quad
\mathcal{G}(M_1,\dots,M_k)=\left(\bigoplus_{i=1}^k M_i,\ \text{tác động theo toạ độ ở (a)}\right),
\]
và với $(f_i):(M_i)\to (N_i)$, đặt $\mathcal{G}(f_i)=\bigoplus_i f_i$.

Theo (a) và (b), $\mathcal{G}\circ \mathcal{F}(M)\cong M$ tự nhiên qua đẳng cấu
\[
\eta_M:\ \bigoplus_{i=1}^k \widetilde{\rho}(1_i)M \xrightarrow{\;\;\sum\iota_i\;\;} M,\qquad
(m_1,\dots,m_k)\mapsto \sum_i m_i,
\]
có nghịch đảo $m\mapsto \bigl(\widetilde{\rho}(1_1)m,\dots,\widetilde{\rho}(1_k)m\bigr)$.
Ngược lại, $\mathcal{F}\circ \mathcal{G}(M_1,\dots,M_k)\cong (M_1,\dots,M_k)$ qua các đồng nhất $M_i \xrightarrow{\ \cong\ } \widetilde{\rho}(1_i)\!\left(\bigoplus_j M_j\right)$.
Do đó $\mathcal{F}$ và $\mathcal{G}$ là các hàm tử nghịch đảo tới nhau lên tới đẳng cấu tự nhiên, suy ra $\mathrm{Mod}_R\simeq \prod_{i=1}^k \mathrm{Mod}_{R_i}$.

Cuối cùng, với $R$-mô đun $M,N$ và các phân rã $M=\bigoplus_i M_i$, $N=\bigoplus_i N_i$ như trên, mọi $R$-cấu xạ $f:M\to N$ bị chặn bởi $e_i$ nên $f=\bigoplus_i f_i$ với $f_i:=e_i f e_i \in \Hom_{R_i}(M_i,N_i)$; ngược lại $\bigoplus_i f_i$ là $R$-tuyến tính theo định nghĩa tác động theo toạ độ. Điều này cho đẳng cấu tự nhiên
\[
\Hom_R(M,N)\;\cong\; \bigoplus_{i=1}^k \Hom_{R_i}(M_i,N_i).
\]
\end{proof}


\begin{problem}[BT15 — Tâm của tổng trực tiếp các đại số]
Cho $\{A_i\}_{i\in I}$ là các $K$-đại số kết hợp có đơn vị. Kí hiệu $Z(B)$ là tâm của một $K$-đại số $B$.
Hãy chứng minh rằng
\[
Z\!\left(\bigoplus_{i\in I} A_i\right)\;=\;\bigoplus_{i\in I} Z(A_i),
\]
trong đó $\bigoplus_{i\in I} A_i$ là tổng trực tiếp (mọi phần tử có chỉ số khác $0$ chỉ tại hữu hạn $i$ và phép nhân theo toạ độ).
Từ đó suy ra, với $r\in\mathbb{Z}_{>0}$ và các số nguyên $m_i\ge 1$,
\[
\dim_K Z\!\left(\bigoplus_{i=1}^{r}\Mat_{m_i}(K)\right)=
\bigl|\{\, i\in\{1,\dots,r\}\mid m_i\ge 1\,\}\bigr|=r .
\]
\end{problem}

\begin{proof}
Đặt $A:=\bigoplus_{i\in I} A_i$. Ta nhắc rằng $A$ gồm các bộ $a=(a_i)_{i\in I}$ với $a_i\in A_i$ và $a_i=0$ cho mọi $i$ trừ hữu hạn nhiều chỉ số; phép cộng, nhân và vô hướng đều theo toạ độ:
\[
(a_i)_{i\in I}+(b_i)_{i\in I}=(a_i+b_i)_{i\in I},\qquad
\lambda\,(a_i)_{i\in I}=(\lambda a_i)_{i\in I},\qquad
(a_i)_{i\in I}\cdot(b_i)_{i\in I}=(a_i b_i)_{i\in I}.
\]

\textbf{Bao hàm $\subseteq$.}
Giả sử $a=(a_i)_{i\in I}\in Z(A)$. Lấy một chỉ số cố định $j\in I$ và phần tử $b\in A_j$ tuỳ ý.
Xem $b$ như phần tử của $A$ bằng cách đặt $b$ ở toạ độ $j$ và $0$ ở các toạ độ khác, kí hiệu $\tilde b$.
Vì $a\in Z(A)$ nên $a\tilde b=\tilde b a$. Soi vào toạ độ $j$ của đẳng thức này ta được
\[
a_j b=b a_j \quad \text{trong } A_j.
\]
Vì $b\in A_j$ tuỳ ý nên suy ra $a_j\in Z(A_j)$.
Do $a$ chỉ có hữu hạn toạ độ khác $0$, ta kết luận $a\in\bigoplus_{i\in I} Z(A_i)$.
Vậy $Z(A)\subseteq \bigoplus_{i\in I} Z(A_i)$.

\textbf{Bao hàm $\supseteq$.}
Ngược lại, lấy $a=(a_i)_{i\in I}\in \bigoplus_{i\in I} Z(A_i)$, nghĩa là $a_i\in Z(A_i)$ và chỉ có hữu hạn $i$ sao cho $a_i\neq 0$.
Với mọi $x=(x_i)_{i\in I}\in A$, ta có
\[
a x=(a_i x_i)_{i\in I}=(x_i a_i)_{i\in I}=x a,
\]
vì $a_i$ nằm trong tâm $Z(A_i)$ với mọi $i\in I$.
Do đó $a\in Z(A)$, suy ra $\bigoplus_{i\in I} Z(A_i)\subseteq Z(A)$.

Kết hợp hai bao hàm ta có đẳng thức $Z\!\left(\bigoplus_{i\in I} A_i\right)=\bigoplus_{i\in I} Z(A_i)$.

\textbf{Hệ quả cho tổng trực tiếp của các đại số ma trận.}
Với mọi $m\ge 1$, tâm của $\Mat_{m}(K)$ là không gian một chiều $K\!\cdot I_m$ (các bội vô hướng của ma trận đơn vị).
Do đó
\[
Z\!\left(\bigoplus_{i=1}^{r}\Mat_{m_i}(K)\right)
=\bigoplus_{i=1}^{r} Z\!\left(\Mat_{m_i}(K)\right)
=\bigoplus_{i=1}^{r} K\!\cdot I_{m_i}\cong K^{\oplus r},
\]
và vì thế $\dim_K Z\!\left(\bigoplus_{i=1}^{r}\Mat_{m_i}(K)\right)=r$, tức là bằng số lượng chỉ số $i$ có $m_i\ge 1$.
\end{proof}


\begin{problem}[BT17]
    Giả sử $R$ là một $\K-$đại số với $\K$ là một trường đóng đại số. Chứng minh rằng một $R-$module khả phân hoàn toàn $V$ là đơn khi và chỉ khi 
    \[\dim_{\K}\End_RV=1.\]
\end{problem}

\begin{proof}
Vì $V$ khả phân hoàn toàn, tồn tại phân rã
\[
V \;\cong\; V_1^{\oplus m_1} \;\oplus\; V_2^{\oplus m_2}
\;\oplus\; \cdots \;\oplus\; V_r^{\oplus m_r},
\]
trong đó các $V_i$ là các $R$-module đơn đôi một không đẳng cấu và
$m_i \ge 0$.

Theo Định lý~1.3.6, ta có đẳng cấu đại số
\[
\End_R(V) \;\cong\;
\bigoplus_{i=1}^{r} M_{m_i}(K),
\]
do $\End_R(V_i)\!\cong\!K$ khi $K$ là trường đóng đại số (Bổ đề Schur).
Từ đó
\[
\dim_K \End_R(V)
  = \sum_{i=1}^{r} m_i^2.
\]

\noindent
($\Rightarrow$) Nếu $V$ đơn thì $r=1$ và $m_1=1$, nên
$\dim_K\End_R(V)=1$.

\noindent
($\Leftarrow$) Ngược lại, giả sử $\dim_K\End_R(V)=1$.
Khi đó $\sum_{i=1}^{r} m_i^2=1$, suy ra chỉ có một chỉ số $i$ với
$m_i=1$ và các $m_j=0$ với $j\ne i$.
Vì vậy $V\cong V_i$ là đơn.

Kết luận:
\[
V \text{ đơn } \Longleftrightarrow \dim_K \End_R(V)=1.\qedhere
\]
\end{proof}

\begin{problem}[Bài~18]
Giả sử $K$ là trường đóng đại số, $V$ là $R$-module đơn và $W$ là $R$-module khả phân hoàn toàn.
Chứng minh rằng $\dim_K \Hom_R(V,W)$ bằng đúng số bội của $V$ trong $W$. 
\end{problem}

\begin{proof}
Vì $W$ khả phân hoàn toàn hữu hạn chiều, tồn tại phân rã
\[
W \;\cong\; \bigoplus_{i=1}^r V_i^{\oplus m_i},
\]
trong đó $V_1,\dots,V_r$ là các $R$-module đơn đôi một không đẳng cấu và $m_i\ge 0$ là các số bội

Do đó
\[
\Hom_R(V,W)
\;\cong\;
\Hom_R\!\Bigl(V,\ \bigoplus_{i=1}^r V_i^{\oplus m_i}\Bigr)
\;\cong\;
\bigoplus_{i=1}^r \Hom_R\!\bigl(V, V_i^{\oplus m_i}\bigr)
\;\cong\;
\bigoplus_{i=1}^r \Hom_R(V,V_i)^{\oplus m_i}.
\]

Vì $V$ là đơn và $K$ là trường đóng đại số, Bổ đề Schur cho biết
\[
\Hom_R(V,V_i) \;\cong\;
\begin{cases}
K & \text{nếu } V \cong V_i,\\
0 & \text{nếu } V \not\cong V_i.
\end{cases}
\]
Gọi $j$ là chỉ số (nếu có) sao cho $V\cong V_j$. Khi đó
\[
\Hom_R(V,W) \;\cong\; K^{\oplus m_j},
\qquad\text{nên}\qquad
\dim_K \Hom_R(V,W) \;=\; m_j,
\]
chính là số bội của $V$ trong $W$. (Nếu $V$ không xuất hiện trong $W$ thì $m_j=0$ và vế trái bằng $0$.)

Do đó, $\dim_K \Hom_R(V,W)$ đúng bằng số bội của $V$ trong $W$ như yêu cầu.%

\end{proof}

\begin{problem}[Bài~19]
Giả sử $K$ là trường đóng đại số. Một $R$-module khả phân hoàn toàn $V$
có một \emph{phân tích không bội} (tức là $V \cong \bigoplus_{i=1}^r V_i$ với các $V_i$ đơn đôi một không đẳng cấu)
khi và chỉ khi đại số tự đồng cấu $\End_R(V)$ là giao hoán.%
\end{problem}

\begin{proof}
Vì $V$ khả phân hoàn toàn hữu hạn chiều, tồn tại phân rã
\[
V \;\cong\; \bigoplus_{i=1}^{r} V_i^{\oplus m_i},
\]
trong đó các $V_i$ đơn đôi một không đẳng cấu và $m_i\ge 0$.%

Trong trường hợp $K$ đóng đại số, Bổ đề Schur cho $\End_R(V_i)\cong K$, và không có đồng cấu không tầm thường giữa $V_i$ và $V_j$ khi $i\neq j$.
Từ mô tả ma trận khối chuẩn của các đồng cấu giữa các tổng trực tiếp các khối đơn, ta có
\[
\End_R(V)\;\cong\;\bigoplus_{i=1}^{r} M_{m_i}(K).%
\footnote{Mô tả $\Hom$/$\End$ bằng ma trận với phần tử là bội vô hướng của $\mathrm{id}_{V_i}$:
\textup{}}
\]

\smallskip
\noindent\emph{($\Rightarrow$)} Nếu $V$ có phân tích không bội, tức $m_i=1$ với mọi $i$, thì
\[
\End_R(V)\;\cong\;\bigoplus_{i=1}^r M_{1}(K)\;\cong\;K^{\oplus r},
\]
rõ ràng giao hoán.

\smallskip
\noindent\emph{($\Leftarrow$)} Ngược lại, giả sử $\End_R(V)$ giao hoán. Từ đẳng cấu trên,
$\End_R(V)$ là tổng trực tiếp của các đại số ma trận $M_{m_i}(K)$. Nhưng $M_{m}(K)$ giao hoán
khi và chỉ khi $m=1$. Do đó mọi $m_i$ đều bằng $1$, nên $V \cong \bigoplus_{i=1}^r V_i$
là phân tích không bội.

\smallskip
Kết luận: $\End_R(V)$ giao hoán $\Longleftrightarrow$ mọi $m_i=1 \Longleftrightarrow V$ có phân tích không bội.
\end{proof}

\begin{problem}[Bài~20]
Giả sử $K$ là trường đóng đại số.
Nếu $V$ và $W$ là các $R$-module hữu hạn chiều khả phân hoàn toàn sao cho
\[
\dim_K \End_R(V) \;=\; \dim_K \Hom_R(V,W) \;=\; \dim_K \End_R(W),
\]
thì $V \cong W$.%

\end{problem}

\begin{proof}
Vì $V,W$ khả phân hoàn toàn hữu hạn chiều, tồn tại phân rã (như (1.6))
\[
V \;\cong\; \bigoplus_{i=1}^r V_i^{\oplus m_i},
\qquad
W \;\cong\; \bigoplus_{i=1}^r V_i^{\oplus n_i},
\]
trong đó $V_1,\dots,V_r$ là các mô-đun đơn đôi một không đẳng cấu, và (nếu cần)
ta cho phép một vài $m_i$ hoặc $n_i$ bằng $0$ để hai phân rã dùng cùng hệ $V_i$.%


Khi $K$ đóng đại số, Bổ đề Schur cho $\End_R(V_i)\cong K$ và không có đồng cấu
giữa $V_i$ và $V_j$ với $i\ne j$. Khi đó
\[
\End_R(V)\;\cong\; \bigoplus_{i=1}^r M_{m_i}(K),
\qquad
\End_R(W)\;\cong\; \bigoplus_{i=1}^r M_{n_i}(K),
\]
nên
\[
\dim_K \End_R(V) \;=\; \sum_{i=1}^r m_i^2,
\qquad
\dim_K \End_R(W) \;=\; \sum_{i=1}^r n_i^2. 
\]


Mặt khác,
\[
\Hom_R(V,W)
\;\cong\;
\bigoplus_{i=1}^r \Hom_R\!\bigl(V_i^{\oplus m_i},\, V_i^{\oplus n_i}\bigr)
\;\cong\;
\bigoplus_{i=1}^r M_{n_i\times m_i}(K),
\]
suy ra
\[
\dim_K \Hom_R(V,W) \;=\; \sum_{i=1}^r m_i n_i.
\]


Theo giả thiết,
\[
\sum_{i=1}^r m_i^2 \;=\; \sum_{i=1}^r m_i n_i \;=\; \sum_{i=1}^r n_i^2.
\]
Khi đó
\[
\sum_{i=1}^r (m_i - n_i)^2
\;=\;
\sum m_i^2 - 2\sum m_i n_i + \sum n_i^2
\;=\; 0,
\]
nên $m_i = n_i$ với mọi $i$.

Vì các số bội trùng nhau, từ tiêu chuẩn đẳng cấu cho mô-đun semisimple
suy ra $V \cong W$.%
\footnote{Xem thêm tiêu chuẩn $V\cong W \iff (m_i)=(n_i)$:
\textup{}}
\end{proof}

\begin{problem}[Bài~21]
Hai $R$-module $V$ và $W$, có các phân rã
\[
V \;\cong\; \bigoplus_{i=1}^r V_i^{\oplus m_i},
\qquad
W \;\cong\; \bigoplus_{i=1}^r V_i^{\oplus n_i},
\]
trong đó $V_1,\dots,V_r$ là các mô-đun đơn đôi một không đẳng cấu (ký hiệu như trong (1.6)),
là đẳng cấu khi và chỉ khi $m_i=n_i$ với mọi $i=1,\dots,r$.%

\end{problem}

\begin{proof}
\emph{($\Rightarrow$)} Giả sử $V \cong W$. Với mỗi $i$, áp dụng Bài~11 cho mô-đun đơn $V_i$ và mô-đun khả phân hoàn toàn $V$ (tương ứng $W$) ta có
\[
\dim_K \Hom_R(V_i,V) \;=\; m_i, 
\qquad
\dim_K \Hom_R(V_i,W) \;=\; n_i,
\]
trong đó số chiều bằng đúng số bội của $V_i$ trong mô-đun kia.%
\footnote{Bài~11: \(\dim_K\Hom_R(V,W)\) bằng số bội của \(V\) trong \(W\). \textup{}}
Từ $V \cong W$ suy ra
\(\dim_K \Hom_R(V_i,V)=\dim_K \Hom_R(V_i,W)\), nên \(m_i=n_i\) với mọi \(i\).

\smallskip
\emph{($\Leftarrow$)} Ngược lại, nếu \(m_i=n_i\) với mọi \(i\), thì
\[
V \;\cong\; \bigoplus_{i=1}^r V_i^{\oplus m_i}
\;\cong\; \bigoplus_{i=1}^r V_i^{\oplus n_i}
\;\cong\; W,
\]
trong đó đẳng cấu ở giữa thu được bằng cách ghép các đồng cấu đồng nhất trên từng khối \(V_i\) lặp lại \(m_i=n_i\) lần. (Về mặt tuyến tính, ta chọn một đẳng cấu \(K^{m_i}\!\to K^{n_i}\) cho mỗi \(i\) rồi kéo theo đẳng cấu \(V_i^{\oplus m_i}\!\to V_i^{\oplus n_i}\); ghép các khối lại cho ta một đẳng cấu \(V\to W\).)

Kết luận: \(V \cong W \iff (m_i)_{i=1}^r=(n_i)_{i=1}^r\).
\end{proof}

\begin{problem}[Bài~22]
Giả sử $K$ là trường đóng đại số. Nếu $V_1,\dots,V_r$ là các $R$-module đơn đôi một không đẳng cấu và
\[
V \;=\; V_1 \oplus \cdots \oplus V_r,
\]
thì mọi không gian con bất biến (tức $R$-module con) của $V$ đều có dạng $V_{i_1}\oplus\cdots\oplus V_{i_k}$ với một tập con $\{i_1,\dots,i_k\}\subset\{1,\dots,r\}$. Ngược lại, nếu $n\ge 2$ thì $V^{\oplus n}$ có vô số không gian con bất biến khi $K$ là vô hạn.%
\footnote{Phát biểu bài tập: \textup{}}
\end{problem}

\begin{proof}
\textbf{Phần 1: Nếu $V=\bigoplus_{i=1}^r V_i$ (không bội) thì mọi $R$-module con là tổng các $V_i$.}

Ta chứng minh bằng quy nạp theo $r$. Trường hợp $r=1$ là hiển nhiên vì $V_1$ đơn: mô-đun con chỉ có $0$ hoặc $V_1$.

Giả sử mệnh đề đúng với $r-1$ khối và xét $V = (V_1\oplus\cdots\oplus V_{r-1}) \oplus V_r$. Gọi $W\le V$ là mô-đun con bất kỳ. Xét phép chiếu $R$-tuyến tính
\[
\pi_r : V \longrightarrow V_r.
\]
Khi đó $\pi_r(W)\le V_r$ nên hoặc $\pi_r(W)=0$ hoặc $\pi_r(W)=V_r$ vì $V_r$ đơn.

\emph{Trường hợp 1:} $\pi_r(W)=0$. Khi đó mọi phần tử $w\in W$ có tọa độ theo $V_r$ bằng $0$, do vậy $W\subset V_1\oplus\cdots\oplus V_{r-1}$. Áp dụng giả thiết quy nạp cho $W$ như mô-đun con của $\bigoplus_{i=1}^{r-1}V_i$, suy ra $W$ là tổng trực tiếp của một số $V_i$ với $i<r$.

\emph{Trường hợp 2:} $\pi_r(W)=V_r$. Khi đó có dãy ngắn chính xác
\[
0 \to \ker(\pi_r|_W) \longrightarrow W \xrightarrow{\ \pi_r|_W\ } V_r \to 0.
\]
Do $V$ (và do đó mọi mô-đun con của $V$) là khả phân hoàn toàn, dãy trên tách, nên
\[
W \;\cong\; \ker(\pi_r|_W) \oplus V_r.
\]
Mà $\ker(\pi_r|_W)\subset \ker \pi_r = V_1\oplus\cdots\oplus V_{r-1}$, nên áp dụng quy nạp cho $\ker(\pi_r|_W)$ ta được nó là tổng trực tiếp của một số $V_i$ với $i<r$. Suy ra $W$ là tổng trực tiếp của $V_r$ và một số $V_i$ với $i<r$, như phải chứng minh.

Kết hợp hai trường hợp, mọi $W\le V$ có dạng $V_{i_1}\oplus\cdots\oplus V_{i_k}$.

\medskip
\textbf{Phần 2: Nếu $n\ge 2$ và $K$ vô hạn thì $V^{\oplus n}$ có vô số mô-đun con.}

Chỉ cần xét $n=2$ (trường hợp $n>2$ suy ra ngay vì $V^{\oplus 2}$ là mô-đun con trực tiếp của $V^{\oplus n}$).
Chọn một chỉ số $i$ bất kỳ, xét mô-đun con $V_i\oplus V_i \subset V^{\oplus 2}$. Vì $V_i$ đơn và $K$ đóng đại số, theo Bổ đề Schur ta có $\End_R(V_i)\cong K$. Khi đó các mô-đun con của $V_i\oplus V_i$ tương ứng với các đường thẳng con của $K^2$ qua $0$ (cụ thể là các \emph{đồ thị} của các tự đồng cấu $K$-vô hướng trên $V_i$).

Với mỗi $\lambda\in K$, đặt
\[
S_\lambda \;=\; \{(v,\ \lambda v):\ v\in V_i\} \;\subset\; V_i\oplus V_i.
\]
Dễ thấy $S_\lambda$ là mô-đun con (ổn định bởi tác động của $R$), và nếu $\lambda\neq\mu$ thì $S_\lambda\neq S_\mu$ (vì $(v,\lambda v)=(w,\mu w)\neq 0$ kéo theo $\lambda=\mu$). Khi $K$ vô hạn, họ $\{S_\lambda\}_{\lambda\in K}$ là vô hạn, suy ra $V^{\oplus 2}$ (và do đó $V^{\oplus n}$ với $n\ge 2$) có vô số mô-đun con.

\medskip
Kết luận hai phần như bài yêu cầu.
\end{proof}

\begin{problem}[Bài~1.2.9]
Nếu $K$ là trường đóng đại số và $G$ là nhóm abelian, thì mọi biểu diễn đơn hữu hạn chiều
$\rho:G\to\GL(V)$ (tức $V$ là $R$-module đơn với $R=K[G]$) đều có số chiều bằng $1$.
\end{problem}

\begin{proof}[Chứng minh dùng gợi ý (tam giác hoá đồng thời)]
Vì $G$ abelian nên các ma trận $\rho(g)$ \emph{đôi một giao hoà}:
$\rho(g)\rho(h)=\rho(h)\rho(g)$ với mọi $g,h\in G$.
Bộ các toán tử giao hoà trên một trường đóng đại số có thể
\emph{tam giác hoá đồng thời}: tồn tại một cơ sở của $V$ mà theo đó
mọi $\rho(g)$ đều ở dạng tam giác trên.

Suy ra tồn tại một \emph{vectơ riêng chung} $0\neq v\in V$ sao cho
$\rho(g)v=\lambda(g)\,v$ với mọi $g\in G$ và một số vô hướng $\lambda(g)\in K$.
Khi đó không gian con $K v$ là bất biến dưới mọi $\rho(g)$, tức là một
$G$-mô-đun con khác $\{0\}$ của $V$. Vì $V$ là \emph{đơn}, nên $V=K v$.
Do đó $\dim_K V=1$.
\end{proof}

\begin{proof}[Chứng minh khác (bằng không gian riêng bất biến)]
Chọn $g_0\in G$. Vì $K$ đóng đại số, $\rho(g_0)$ có trị riêng $\lambda\in K$
với không gian riêng $E_\lambda(\rho(g_0))\neq\{0\}$. Với mọi $h\in G$,
do $\rho(h)$ giao hoà với $\rho(g_0)$ nên
\[
\rho(g_0)\bigl(\rho(h)w\bigr)
=\rho(h)\bigl(\rho(g_0)w\bigr)
=\rho(h)(\lambda w)
=\lambda\,\rho(h)w,
\quad \forall\,w\in E_\lambda(\rho(g_0)).
\]
Do đó $E_\lambda(\rho(g_0))$ là mô-đun con bất biến của $V$.
Vì $V$ đơn nên $V=E_\lambda(\rho(g_0))$, nghĩa là $\rho(g_0)$ tác động
bằng \emph{vô hướng} trên toàn $V$. Lặp lại lập luận cho mọi $g\in G$,
ta thấy mọi $\rho(g)$ đều là vô hướng trên $V$; hệ quả là \emph{mọi}
không gian con khác $\{0\}$ của $V$ đều bất biến. Tính đơn của $V$
bởi vậy buộc $\dim_K V=1$.
\end{proof}


\begin{problem}[Bài~4]
Giả sử $A$ là một ma trận với các phần tử trong một trường đóng đại số $K$.
Giả sử $A^{\,n}=I$ với một số nguyên dương $n$ không chia hết cho đặc số của $K$.
Sử dụng định lý Maschke và Bài tập~1.2.9 để chứng minh rằng $A$ chéo hoá được.
\end{problem}

\begin{proof}
Xét không gian vectơ $V=K^m$ (với $m=\dim_K V$) và nhóm cyclic $G=\langle g\rangle\cong \mathbb{Z}/n\mathbb{Z}$.
Điều kiện $A^{\,n}=I$ cho phép định nghĩa một biểu diễn
\[
\rho: G \longrightarrow \GL(V),\qquad \rho(g)=A.
\]
Do $\operatorname{char}K \nmid n$, theo \emph{định lý Maschke} mọi biểu diễn hữu hạn chiều của $G$
là khả quy hoàn toàn (tức phân rã thành tổng trực tiếp các mô-đun con đơn).%
\footnote{Định lý Maschke trong tài liệu đính kèm. \textup{:contentReference[oaicite:0]{index=0}}}
Mặt khác $G$ là abelian, nên theo Bài tập~1.2.9, mọi biểu diễn đơn của $G$ trên trường đóng đại số $K$
đều có \emph{số chiều bằng $1$}.
Suy ra $V$ phân rã thành tổng trực tiếp các mô-đun con bất biến $1$-chiều dưới tác động của $G$:
\[
V \;\cong\; \bigoplus_{i=1}^r K v_i,\qquad \rho(g)v_i=\lambda_i v_i \ \ (0\ne v_i\in V,\ \lambda_i\in K).
\]
Vì $\rho(g)=A$, ta có $A v_i=\lambda_i v_i$ với mỗi $i$, tức $v_1,\dots,v_r$ là hệ các vectơ riêng của $A$.
Chọn cơ sở gồm các $v_i$, ma trận của $A$ theo cơ sở này là chéo (đường chéo là các $\lambda_i$).
Do đó $A$ chéo hoá được.
\end{proof}


\section{Định lý Maschke}

\begin{problem}[Bài tập 1 — Trường hợp thất bại của Định lý Maschke]
Chứng minh rằng nếu đặc trưng của trường $K$ là một ước của cấp của nhóm $G$, 
thì tồn tại một $K[G]$-mô đun không khả phân hoàn toàn.
\end{problem}

\begin{proof}
Đây chính là hướng “ngược lại” của Định lý Maschke:  
nếu $\operatorname{char}(K)$ chia $|G|$, thì không phải mọi $K[G]$-mô đun đều khả phân hoàn toàn.

\medskip
\textbf{Bước 1.}  
Gọi $n = |G|$, và giả sử $\operatorname{char}(K)$ chia $n$, nghĩa là tồn tại $p = \operatorname{char}(K)$ sao cho $p\mid n$.

\medskip
\textbf{Bước 2.}  
Xét mô đun tả chính quy $V = K[G]$, tức là $V$ có cơ sở $\{e_g\}_{g\in G}$ và tác động của nhóm được cho bởi
\[
h\cdot e_g = e_{hg}, \qquad h,g \in G.
\]
Ta định nghĩa một mô đun con $W \subset V$ như sau:
\[
W := \Bigl\{\, \sum_{g\in G} a_g e_g \in K[G] \;\Big|\; \sum_{g\in G} a_g = 0 \,\Bigr\}.
\]
Đây là không gian các phần tử có tổng hệ số bằng $0$. 
Rõ ràng $W$ là một mô đun con bất biến, vì tác động của $G$ chỉ hoán vị các cơ sở $e_g$,
nên tổng các hệ số vẫn bằng $0$ sau khi nhân với bất kỳ $h\in G$.

\medskip
\textbf{Bước 3.}  
Ta có dãy ngắn chính xác:
\[
0 \longrightarrow W \longrightarrow V \xrightarrow{\;\;\varepsilon\;\;} K \longrightarrow 0,
\]
trong đó $\varepsilon$ là đồng cấu mô đun $K[G]\to K$ gửi $\sum_{g\in G} a_g e_g \mapsto \sum_{g\in G} a_g$.
Không gian $W = \ker(\varepsilon)$ là mô đun con bất biến như đã thấy.
Hơn nữa, $K$ được coi là mô đun tầm thường (vì $g\cdot 1 = 1$ với mọi $g$).

\medskip
\textbf{Bước 4.}  
Giả sử dãy này tách, tức là tồn tại mô đun con bất biến $U\subseteq V$ sao cho
\[
V = W \oplus U,
\]
với $U \cong K$ (mô đun tầm thường).
Khi đó phải có phần tử $v = \sum_{g\in G} a_g e_g \in V$ sao cho $U = Kv$ và $g\cdot v = v$ với mọi $g\in G$,
và $\varepsilon(v)=1$ (vì $V=W\oplus U$).

Điều kiện $g\cdot v = v$ cho mọi $g$ nghĩa là tất cả các hệ số $a_g$ bằng nhau:
$a_g = a_h =: c$ với mọi $g,h$.  
Khi đó $v = c\sum_{g\in G} e_g$.
Ta có $\varepsilon(v) = c|G| = c n$.

Muốn $\varepsilon(v)=1$, cần $c = 1/n$ trong $K$.
Nhưng nếu $\operatorname{char}(K)$ chia $n$, thì $n=0$ trong $K$,
và phần tử $1/n$ không tồn tại.  
Do đó không tồn tại phần tử $v$ như vậy.

\medskip
\textbf{Bước 5.}  
Kết luận: dãy ngắn chính xác ở Bước 3 không tách, 
tức là $V=K[G]$ không khả phân hoàn toàn.
Vì thế, tồn tại $K[G]$-mô đun không khả phân hoàn toàn khi $\operatorname{char}(K)\mid |G|$.

\end{proof}

\begin{remark}
Ví dụ này chính là ví dụ kinh điển minh hoạ cho \emph{sự thất bại của Định lý Maschke}.  
Nó cho thấy điều kiện $\operatorname{char}(K)\nmid |G|$ là cần thiết để mọi $K[G]$-mô đun đều khả phân hoàn toàn.
Trong trường hợp $\operatorname{char}(K)\mid |G|$, 
phần tử trung bình $\displaystyle \frac{1}{|G|}\sum_{g\in G}g$ không tồn tại trong $K[G]$,
và do đó không thể dựng được phép chiếu $G$-tuyến tính để tách dãy ngắn như trong chứng minh của định lý.
\end{remark}
\begin{theorem}[Định lý Maschke]\label{thm:Maschke}
Giả sử $\operatorname{char}(K)$ \emph{không} chia $|G|$. 
Cho $(\rho,V)$ là một $K$-biểu diễn của nhóm $G$ (tức $V$ là $K[G]$-mô đun). 
Khi đó với mọi không gian con bất biến $W\subseteq V$ đều tồn tại một phần bù \emph{bất biến} $U\subseteq V$ sao cho
\[
V \;=\; W \oplus U,\qquad g\cdot U \subseteq U\ \text{ với mọi } g\in G.
\]
Hệ quả: mọi $K[G]$-mô đun hữu hạn chiều là khả phân hoàn toàn.
\end{theorem}

\begin{proof}
\textbf{Bước 1: Phần bù tuyến tính $\Longleftrightarrow$ phép chiếu.}
Gọi $i:W\hookrightarrow V$ là phép nhúng. 
Chọn \emph{bất kỳ} phần bù tuyến tính $U$ của $W$ trong $V$ (không cần bất biến): $V=W\oplus U$. 
Tồn tại duy nhất một phép chiếu tuyến tính (theo $K$)
\[
P:V\longrightarrow W,\qquad 
P|_W=\operatorname{id}_W,\quad P|_U=0.
\]
Khi đó $P$ thỏa $P^2=P$ và $\operatorname{Im}P=W$, $\ker P=U$.

\medskip
\textbf{Bước 2: Trung bình hoá để thu được phép chiếu $G$-tuyến tính.}
Vì $\operatorname{char}(K)\nmid |G|$, tồn tại phần tử $\dfrac{1}{|G|}\in K$. Đặt
\[
\Pi \;:=\; \frac{1}{|G|}\sum_{g\in G} \rho(g)\, P \,\rho(g)^{-1}\ \in \End_K(V).
\]
Ta kiểm tra ba tính chất cơ bản.

\emph{(i) $\Pi$ là $G$-tuyến tính.} Với $h\in G$,
\[
\rho(h)\,\Pi\,\rho(h)^{-1}
= \frac{1}{|G|}\sum_{g\in G}\rho(hg)\,P\,\rho(hg)^{-1}
= \frac{1}{|G|}\sum_{g'\in G}\rho(g')\,P\,\rho(g')^{-1}
= \Pi,
\]
nên $\rho(h)\Pi=\Pi\rho(h)$, tức $\Pi$ là phép nội xạ $G$-tuyến tính (đồng cấu $K[G]$-mô đun).

\emph{(ii) Ảnh của $\Pi$ bằng $W$.}
Với $w\in W$, vì $W$ bất biến nên $\rho(g)^{-1}w\in W$ và $P(\rho(g)^{-1}w)=\rho(g)^{-1}w$. Do đó
\[
\Pi(w)=\frac{1}{|G|}\sum_{g\in G}\rho(g)\rho(g)^{-1}w=w.
\]
Suy ra $W\subseteq \operatorname{Im}\Pi$. Mặt khác, với $v\in V$ và mọi $g$, 
$P\rho(g)^{-1}v\in W$ nên $\rho(g)P\rho(g)^{-1}v\in W$; vì $W$ là không gian con, trung bình của chúng vẫn nằm trong $W$. Do đó $\operatorname{Im}\Pi\subseteq W$. Suy ra $\operatorname{Im}\Pi=W$.

\emph{(iii) $\Pi$ là phép chiếu: $\Pi^2=\Pi$.}
Từ (i) suy ra $\Pi$ giao hoán với mọi $\rho(g)$ và do đó giao hoán với từng hạng $\rho(g)P\rho(g)^{-1}$ trong tổng. Vì $P^2=P$, tính giao hoán cho phép tính:
\[
\Pi^2
= \frac{1}{|G|^2}\sum_{g,h}\rho(g)P\rho(g)^{-1}\rho(h)P\rho(h)^{-1}
= \frac{1}{|G|^2}\sum_{g,h}\rho(gh)\,P^2\,\rho(gh)^{-1}
= \frac{1}{|G|}\sum_{t\in G}\rho(t)P\rho(t)^{-1}
= \Pi.
\]
Vì vậy $\Pi$ là phép chiếu $G$-tuyến tính lên $W$.

\medskip
\textbf{Bước 3: Lấy phần bù bất biến.}
Đặt $U':=\ker \Pi$. Từ $\Pi$ là $G$-tuyến tính suy ra $U'$ bất biến.
Hơn nữa, vì $\Pi$ là phép chiếu với $\operatorname{Im}\Pi=W$, ta có phân rã trực tiếp
\[
V \;=\; \operatorname{Im}\Pi \;\oplus\; \ker\Pi \;=\; W \oplus U',
\]
trong đó $U'$ là phần bù bất biến mong muốn.

\medskip
Kết luận: với mọi $W$ bất biến, tồn tại phần bù bất biến $U'$. 
Do đó mọi dãy ngắn chính xác của $K[G]$-mô đun hữu hạn chiều đều tách, 
và $V$ là tổng trực tiếp của các mô đun con đơn. Nghĩa là $V$ khả phân hoàn toàn.
\end{proof}

\begin{remark}[Dạng phép chiếu tách dãy]
Với dãy ngắn chính xác $0\to W \xrightarrow{i} V \xrightarrow{\pi} Q \to 0$ trong $\mathrm{Mod}_{K[G]}$, 
chọn bất kỳ bồi hoàn tuyến tính $s:Q\to V$ (không cần bất biến), rồi trung bình hoá
\[
S(q)\;:=\;\frac{1}{|G|}\sum_{g\in G} \rho(g)\,s\!\bigl(\pi(\rho(g)^{-1}q)\bigr).
\]
Khi đó $S$ là bồi hoàn $G$-tuyến tính của $\pi$, và $V \cong W \oplus Q$ như $K[G]$-mô đun. 
Đây là phiên bản “trung bình hoá” tương đương với phép chiếu ở trên.
\end{remark}
\begin{lemma}[Bổ đề 3 — Phép chiếu song song theo phần bù tuyến tính]\label{lem:projection}
Cho $V$ là một $K$-không gian vectơ, và $W, U$ là hai không gian con của $V$ thỏa mãn
\[
V = W \oplus U.
\]
Khi đó, với mỗi $v \in V$, tồn tại duy nhất phần tử $P(v) \in W$ sao cho
\[
v - P(v) \in U.
\]
Hơn nữa, quy tắc ánh xạ $v \mapsto P(v)$ là một tự đồng cấu tuyến tính của $V$, 
được gọi là \emph{phép chiếu song song lên $W$ theo $U$}.
\end{lemma}

\begin{proof}
\textbf{(1) Tồn tại và duy nhất phần tử $P(v)$.}

Do $V = W \oplus U$, mọi $v \in V$ có thể được viết duy nhất dưới dạng
\[
v = w + u, \quad \text{với } w \in W,\, u \in U.
\]
Khi đó, ta định nghĩa $P(v) := w$. 
Từ biểu diễn duy nhất này, $w$ là duy nhất, nên $P(v)$ được xác định duy nhất.

Rõ ràng $v - P(v) = v - w = u \in U$, 
và nếu tồn tại $w' \in W$ khác $w$ mà cũng thỏa $v - w' \in U$ thì
\[
(w - w') = (v - w') - (v - w) \in U,
\]
nhưng $w - w' \in W$, nên $w - w' \in W \cap U = \{0\}$, 
tức là $w' = w$. Do đó $P(v)$ duy nhất.

\medskip
\textbf{(2) Tuyến tính của $P$.}

Với mọi $v_1,v_2\in V$ và $\lambda\in K$, ta có
\[
P(v_1+v_2)=w_1+w_2=P(v_1)+P(v_2),\quad
P(\lambda v_1)=\lambda w_1 = \lambda P(v_1),
\]
vì biểu diễn $v=w+u$ là tuyến tính theo từng toạ độ khi chiếu lên $W$ và $U$. 
Do đó $P$ là ánh xạ tuyến tính.

\medskip
\textbf{(3) Tính chất đặc trưng của phép chiếu.}

Vì $V=W\oplus U$, nên $\operatorname{Im}P=W$ và $\ker P=U$.
Hơn nữa,
\[
P^2(v)=P(P(v))=P(w)=w=P(v),
\]
nên $P^2=P$. 
Đây là đặc trưng của mọi phép chiếu song song tuyến tính.

\medskip
\textbf{(4) Tổng kết.}
Ta đã xây dựng được một phép chiếu tuyến tính $P:V\to W$ thỏa:
\[
P^2=P,\qquad \operatorname{Im}P=W,\qquad \ker P=U,
\]
và với mọi $v\in V$, $v-P(v)\in U$.
Do đó $P$ là phép chiếu song song lên $W$ theo $U$ như cần chứng minh.
\end{proof}

\begin{remark}
Khi $V=W\oplus U$, ngoài phép chiếu $P$ lên $W$ theo $U$, ta còn có phép chiếu bù $Q:=I-P$ lên $U$ theo $W$.  
Hai phép chiếu này thỏa $P+Q=I$, $PQ=QP=0$, 
và $\operatorname{Im}P=W$, $\operatorname{Im}Q=U$.
Các phép chiếu song song này chính là công cụ trung tâm trong chứng minh định lý Maschke:
ta “trung bình hóa” phép chiếu $P$ để biến nó thành phép chiếu $G$-tuyến tính.
\end{remark}
\begin{proposition}[Mệnh đề 4 — Tự đồng cấu lũy đẳng và phép chiếu song song]\label{prop:idempotent}
Một phép chiếu song song là một tự đồng cấu \emph{lũy đẳng} của không gian vectơ.  
Ngược lại, mọi tự đồng cấu lũy đẳng đều là một phép chiếu song song lên ảnh của nó theo nhân của nó.
\end{proposition}

\begin{proof}
\textbf{Chiều thuận:}  
Giả sử $P:V\to V$ là phép chiếu song song lên $W$ theo $U$, tức là $V=W\oplus U$, 
$\operatorname{Im}P=W$ và $\ker P=U$.
Theo Bổ đề~\ref{lem:projection}, với mọi $v\in V$, ta có $v = P(v) + (v - P(v))$ và $v - P(v)\in U$.  
Khi đó:
\[
P^2(v) = P(P(v)) = P(v),
\]
vì $P(v)\in W$ và $P|_W = \mathrm{id}_W$.  
Do đó $P^2 = P$, nghĩa là $P$ là một \emph{tự đồng cấu lũy đẳng} (idempotent endomorphism) của $V$.

\medskip
\textbf{Chiều đảo:}  
Ngược lại, giả sử $T:V\to V$ là một tự đồng cấu lũy đẳng, nghĩa là $T^2 = T$.  
Khi đó, đặt
\[
W := \operatorname{Im}T, \qquad U := \ker T.
\]
Ta chứng minh rằng $V = W \oplus U$ và $T$ là phép chiếu song song lên $W$ theo $U$.

\emph{(i) Giao bằng không:}  
Nếu $v\in W\cap U$, thì tồn tại $x$ sao cho $v=T(x)$ và đồng thời $T(v)=0$.  
Áp dụng $T$ vào $v=T(x)$ ta có $T(v)=T^2(x)=T(x)=v$, do đó $v=0$.  
Suy ra $W\cap U=\{0\}$.

\emph{(ii) Tổng là toàn bộ $V$:}  
Với mọi $v\in V$, ta viết
\[
v = T(v) + (v - T(v)).
\]
Ở đây $T(v)\in W$ và $T(v - T(v)) = T(v) - T^2(v)=0$, nên $(v - T(v))\in U$.  
Do đó $V=W+U$. Kết hợp với (i), ta được $V=W\oplus U$.

\emph{(iii) Tính chất phép chiếu:}  
Từ định nghĩa $T(v)=T^2(v)$, với mọi $v\in V$, ta có $T^2=T$,
$\operatorname{Im}T=W$, $\ker T=U$.  
Vì vậy $T$ chính là phép chiếu song song lên $W$ theo $U$.

\medskip
\textbf{Kết luận:}
Mọi phép chiếu song song đều là tự đồng cấu lũy đẳng, và mọi tự đồng cấu lũy đẳng đều xuất phát từ một phép chiếu song song.  
Hai khái niệm này là tương đương trong phạm trù không gian vectơ tuyến tính hữu hạn chiều (và nói chung, trong mọi mô đun).
\end{proof}

\begin{remark}
Tập hợp các tự đồng cấu lũy đẳng $\{\,T\in \End_K(V)\mid T^2=T\,\}$ tương ứng một-một với các phân rã trực tiếp $V = W\oplus U$.  
Trong ứng dụng, đặc biệt trong chứng minh Định lý Maschke, ta lợi dụng điều này để “trung bình hóa” một phép chiếu $P$ (vốn là lũy đẳng) thành phép chiếu $G$-tuyến tính.
\end{remark}

\begin{lemma}[Bổ đề 5 — Phần bù bất biến và phép chiếu $G$-tuyến tính]\label{lem:invariant_projection}
Cho $V$ là một biểu diễn $K$-tuyến tính của nhóm $G$, và $W$ là một không gian con bất biến của $V$.  
Gọi $U$ là một phần bù tuyến tính của $W$ trong $V$ và $P$ là phép chiếu song song của $V$ lên $W$ theo $U$.  
Khi đó:
\[
U \text{ là một không gian con bất biến của } V 
\quad \Longleftrightarrow \quad
P \text{ là một } G\text{-tự đồng cấu của } V.
\]
\end{lemma}

\begin{proof}
Ta cần chứng minh hai chiều tương đương.

\medskip
\textbf{(1) Chiều thuận: Nếu $U$ là bất biến, thì $P$ là $G$-tuyến tính.}

Giả sử $U$ là một không gian con bất biến, tức là
\[
\rho(g)(U)\subseteq U, \qquad \forall g\in G.
\]
Vì $W$ cũng là bất biến (theo giả thiết), ta có với mọi $g\in G$:
\[
\rho(g)(W)\subseteq W, \qquad \rho(g)(U)\subseteq U.
\]
Khi đó, biểu diễn $V=W\oplus U$ được bảo toàn bởi tác động của nhóm.

Với mỗi $v\in V$, ta có thể viết duy nhất $v=w+u$ với $w\in W$, $u\in U$,
và $P(v)=w$. Khi đó:
\[
\rho(g)\,P(v) = \rho(g)w,\qquad
P\,\rho(g)(v) = P(\rho(g)w + \rho(g)u) = \rho(g)w,
\]
vì $\rho(g)w\in W$ và $\rho(g)u\in U$.
Do đó $\rho(g)\,P(v)=P\,\rho(g)(v)$ với mọi $v\in V$, nên $\rho(g)P = P\rho(g)$, tức là $P$ là $G$-tuyến tính.

\medskip
\textbf{(2) Chiều đảo: Nếu $P$ là $G$-tuyến tính, thì $U$ là bất biến.}

Giả sử $P$ là một $G$-tự đồng cấu của $V$, nghĩa là
\[
P\,\rho(g) = \rho(g)\,P, \qquad \forall g\in G.
\]
Ta sẽ chứng minh rằng $U=\ker P$ là bất biến.

Lấy $u\in U$, tức là $P(u)=0$.  
Với mọi $g\in G$, ta có
\[
P(\rho(g)u) = \rho(g)P(u) = \rho(g)\cdot 0 = 0.
\]
Suy ra $\rho(g)u \in \ker P = U$.  
Như vậy $U$ là không gian con bất biến.

\medskip
\textbf{(3) Kết luận.}
Hai chiều trên chứng minh rằng
\[
U \text{ bất biến } \Longleftrightarrow P \text{ là } G\text{-tuyến tính}.
\]
Điều này hoàn tất chứng minh bổ đề.
\end{proof}

\begin{remark}
Bổ đề này đóng vai trò quan trọng trong chứng minh Định lý Maschke:  
khi muốn tìm một phần bù \emph{bất biến} $U$ cho $W$, ta chỉ cần điều chỉnh phép chiếu $P$ (ban đầu không $G$-tuyến tính) để trở thành $G$-tuyến tính, ví dụ bằng cách “trung bình hoá”:
\[
\Pi \;=\; \frac{1}{|G|}\sum_{g\in G} \rho(g)\,P\,\rho(g)^{-1}.
\]
Khi $\operatorname{char}(K)\nmid |G|$, phép chiếu $\Pi$ thu được là $G$-tuyến tính, và do đó $\ker\Pi$ là phần bù bất biến của $W$ trong $V$.
\end{remark}
\begin{problem}[Bài tập 6 — Chứng minh định lý Maschke]\label{prob:Maschke}
Gọi $U$ là một phần bù tuyến tính của $W$ trong $V$ và $P$ là phép chiếu song song của $V$ lên $W$ dọc theo $U$.  
Định nghĩa tự đồng cấu $\overline{P}$ của $V$ bởi công thức trung bình cộng:
\[
\overline{P}
\;=\;
\frac{1}{|G|}
\sum_{g\in G} \rho(g)\, P\, \rho(g)^{-1}.
\]
Giả thiết rằng $|G|$ không chia hết cho đặc số của trường $K$, để $\frac{1}{|G|}$ có nghĩa trong $K$.  
Chứng minh các mệnh đề sau:

\begin{enumerate}
\item $\overline{P}$ là một $G$-tự đồng cấu của $V$.
\item $\overline{P}$ là một tự đồng cấu lũy đẳng (idempotent) với không gian ảnh chính là $W$.
\item Từ đó suy ra $\ker(\overline{P})$ là một phần bù bất biến của $W$ trong $V$.
\end{enumerate}
\end{problem}

\begin{proof}
\textbf{(a) Chứng minh rằng $\overline{P}$ là $G$-tuyến tính.}

Lấy $h\in G$. Ta tính:
\[
\rho(h)\,\overline{P}\,\rho(h)^{-1}
= \frac{1}{|G|}\sum_{g\in G} \rho(h)\rho(g)P\rho(g)^{-1}\rho(h)^{-1}
= \frac{1}{|G|}\sum_{g\in G} \rho(hg)P\rho(hg)^{-1}.
\]
Đặt $t = hg$. Khi $g$ chạy qua toàn bộ $G$, $t$ cũng chạy qua $G$.  
Do đó
\[
\rho(h)\,\overline{P}\,\rho(h)^{-1}
= \frac{1}{|G|}\sum_{t\in G} \rho(t)P\rho(t)^{-1}
= \overline{P}.
\]
Suy ra $\rho(h)\,\overline{P} = \overline{P}\,\rho(h)$ với mọi $h\in G$, tức $\overline{P}$ là một $G$-tự đồng cấu của $V$.

\medskip
\textbf{(b) Chứng minh rằng $\overline{P}$ là lũy đẳng và có ảnh bằng $W$.}

Vì $\overline{P}$ là trung bình của các tự đồng cấu $gPg^{-1}$, mà mỗi $gPg^{-1}$ đều là phép chiếu song song lên không gian $\rho(g)(W)$, ta chứng minh hai phần:

\smallskip
\emph{(i) Ảnh của $\overline{P}$ bằng $W$.}  
Với $w\in W$, vì $W$ là bất biến nên $\rho(g)^{-1}w\in W$ và do $P|_W = \mathrm{id}_W$, ta có:
\[
\overline{P}(w)
= \frac{1}{|G|}\sum_{g\in G} \rho(g) P \rho(g)^{-1}(w)
= \frac{1}{|G|}\sum_{g\in G} \rho(g)\rho(g)^{-1}w
= \frac{1}{|G|}\sum_{g\in G} w = w.
\]
Do đó $W\subseteq \operatorname{Im}\overline{P}$.  
Ngược lại, mỗi hạng $\rho(g)P\rho(g)^{-1}(v)$ thuộc $\rho(g)(W)=W$ vì $W$ bất biến.  
Vì tổng của các phần tử trong $W$ vẫn thuộc $W$, ta có $\operatorname{Im}\overline{P}\subseteq W$.  
Suy ra $\operatorname{Im}\overline{P}=W$.

\smallskip
\emph{(ii) $\overline{P}$ là lũy đẳng.}  
Vì $\overline{P}$ giao hoán với mọi $\rho(g)$ (đã chứng minh ở phần (a)), 
nên $\overline{P}$ cũng giao hoán với từng hạng trong tổng định nghĩa của nó:
\[
\overline{P}\,\rho(g)P\rho(g)^{-1}
= \rho(g)\,\overline{P}P\rho(g)^{-1}
= \rho(g)P\rho(g)^{-1}\overline{P},
\]
vì $P^2=P$. Khi đó:
\begin{align*}
\overline{P}^2 
&= \frac{1}{|G|^2}\sum_{g,h\in G}\rho(g)P\rho(g)^{-1}\rho(h)P\rho(h)^{-1} \\
&= \frac{1}{|G|^2}\sum_{g,h\in G}\rho(gh)P^2\rho(gh)^{-1} \\
&= \frac{1}{|G|^2}\sum_{g,h\in G}\rho(gh)P\rho(gh)^{-1} \\
&= \frac{1}{|G|}\sum_{t\in G}\rho(t)P\rho(t)^{-1} 
= \overline{P}.
\end{align*}
Do đó $\overline{P}^2=\overline{P}$, tức $\overline{P}$ là một tự đồng cấu lũy đẳng.

\medskip
\textbf{(c) Suy ra sự tồn tại của phần bù bất biến.}

Vì $\overline{P}$ là phép chiếu lũy đẳng với $\operatorname{Im}\overline{P}=W$, ta có:
\[
V = \operatorname{Im}\overline{P} \oplus \ker\overline{P}
= W \oplus \ker\overline{P}.
\]
Hơn nữa, $\overline{P}$ là $G$-tuyến tính, nên nếu $v\in\ker\overline{P}$ và $g\in G$, ta có:
\[
\overline{P}(\rho(g)v) = \rho(g)\,\overline{P}(v) = 0,
\]
tức là $\rho(g)v\in\ker\overline{P}$.  
Do đó $\ker\overline{P}$ là một không gian con bất biến.

Kết luận: $\ker(\overline{P})$ là một phần bù bất biến của $W$ trong $V$.
\end{proof}

\begin{remark}
Kết quả này chính là \emph{Định lý Maschke}.  
Điều kiện $\operatorname{char}(K)\nmid |G|$ là thiết yếu: nếu đặc trưng của $K$ chia $|G|$, thì trung bình $\frac{1}{|G|}\sum_{g\in G}\rho(g)P\rho(g)^{-1}$ không được định nghĩa, và ví dụ trong Bài tập~\ref{prob:Maschke} cho thấy mô đun có thể không khả phân hoàn toàn.
\end{remark}

%========================================
% ĐỊNH LÝ 7: Tính khả phân hoàn toàn của các biểu diễn
%========================================
\begin{theorem}[Tính khả phân hoàn toàn của các biểu diễn (Định lý Maschke, dạng tương đương)]\label{thm:complete_reducibility}
Giả sử đặc số của trường $K$ \emph{không} phải là một ước của cấp của nhóm $G$.  
Khi đó, mọi biểu diễn $K$-tuyến tính của nhóm $G$ đều khả phân hoàn toàn;  
tức là, với mọi biểu diễn $(\rho,V)$ của $G$, và mọi không gian con bất biến $W\subseteq V$, tồn tại một phần bù bất biến $U\subseteq V$ sao cho
\[
V = W \oplus U.
\]
\end{theorem}

\begin{proof}
Đây là hệ quả trực tiếp của \textbf{Định lý Maschke} đã chứng minh trong Bài tập~\ref{prob:Maschke}.  
Thật vậy, điều kiện $\operatorname{char}(K)\nmid |G|$ cho phép định nghĩa phép chiếu trung bình
\[
\overline{P} = \frac{1}{|G|}\sum_{g\in G}\rho(g)P\rho(g)^{-1}
\]
với mọi phép chiếu tuyến tính $P$ lên $W$ theo một phần bù tuyến tính bất kỳ $U$.  
Như đã chứng minh, $\overline{P}$ là phép chiếu $G$-tuyến tính có ảnh $W$, và $\ker\overline{P}$ là phần bù bất biến $U'$.  
Do đó mọi biểu diễn đều phân rã thành tổng trực tiếp của các mô đun con bất biến — tức là biểu diễn khả phân hoàn toàn.

Nói cách khác, $K[G]$ là một \emph{đại số bán đơn} (semisimple algebra) khi $\operatorname{char}(K)\nmid |G|$, và do đó mọi mô đun hữu hạn chiều trên $K[G]$ đều bán đơn.
\end{proof}

\begin{remark}
Định lý này chính là dạng tổng quát hóa đại số của định lý Maschke cổ điển:  
“Trong đặc trưng tốt, mọi biểu diễn của nhóm hữu hạn đều phân rã thành tổng trực tiếp các biểu diễn đơn.”  
Điều này tương đương với việc vành nhóm $K[G]$ là đại số bán đơn (Maschke 1899), 
và cơ sở cho toàn bộ lý thuyết biểu diễn tuyến tính cổ điển.
\end{remark}


%========================================
% BÀI TẬP 8: Áp dụng định lý Maschke – Ma trận chéo hoá được
%========================================
\begin{problem}[Bài tập 8 — Áp dụng định lý Maschke: Ma trận chéo hoá được]\label{prob:Maschke_diagonal}
Giả sử $A$ là một ma trận có các phần tử trong một trường đóng đại số $K$.  
Giả sử tồn tại số nguyên $n>0$ sao cho $A^n = I$ (ma trận đơn vị), và $n$ không chia hết cho đặc số của $K$.  
Sử dụng định lý Maschke, hãy chứng minh rằng $A$ là ma trận chéo hoá được.
\end{problem}

\begin{proof}
\textbf{Bước 1.}  
Điều kiện $A^n=I$ có nghĩa là các giá trị riêng của $A$ đều là \emph{nghiệm của phương trình $x^n=1$}.  
Vì $K$ là trường đóng đại số và $\operatorname{char}(K)\nmid n$, 
đa thức $x^n - 1$ tách thành tích của \emph{các nhân tử tuyến tính phân biệt}:
\[
x^n - 1 = \prod_{\zeta^n=1}(x-\zeta),
\]
với $\zeta$ là các căn bậc $n$ của đơn vị, và tất cả các nghiệm $\zeta$ đều khác nhau (do $n$ khả nghịch trong $K$).

\textbf{Bước 2.}  
Xét nhóm cyclic $G=\langle g\rangle$ có bậc $n$.  
Đặt biểu diễn $\rho:G\to \GL(V)$ bằng $\rho(g)=A$ trên không gian vectơ $V=K^m$.  
Khi đó $V$ trở thành một $K[G]$-mô đun hữu hạn chiều.

Theo định lý Maschke, vì $\operatorname{char}(K)\nmid |G|=n$, 
biểu diễn này là khả phân hoàn toàn, tức là $V$ phân rã thành tổng trực tiếp của các mô đun con đơn:
\[
V = V_1 \oplus V_2 \oplus \cdots \oplus V_r,
\]
mỗi $V_i$ là một mô đun con đơn của $K[G]$.

\textbf{Bước 3.}  
Các mô đun con đơn của $K[G]$ với $G$ cyclic được mô tả rất rõ ràng:  
mỗi mô đun đơn một chiều được xác định bởi một đặc trưng (character) $\chi:G\to K^\times$,
với $\chi(g)=\zeta$ là một căn bậc $n$ của đơn vị.  
Trên $V_i$, tác động của $A=\rho(g)$ chỉ là phép nhân vô hướng bởi $\zeta_i$.  
Như vậy, $A$ là khả chéo hóa: có một cơ sở mà trong đó ma trận $A$ là đường chéo với các giá trị riêng $\zeta_i$ (là các căn bậc $n$ của $1$).

\textbf{Bước 4.}  
Kết luận rằng $A$ chéo hóa được, với các giá trị riêng là các căn bậc $n$ của đơn vị trong $K$.

\end{proof}

\begin{remark}
Kết quả này là một hệ quả quen thuộc của Định lý Maschke:  
mọi biểu diễn hữu hạn chiều của một nhóm cyclic trong đặc trưng không chia bậc của nhóm đều là tổng trực tiếp của các biểu diễn một chiều,  
và do đó mọi phần tử trong nhóm biểu diễn tương ứng đều khả chéo hoá.
\end{remark}

%========================================
% ĐỊNH NGHĨA 9 — Đại số nửa đơn
%========================================
\begin{definition}[Đại số nửa đơn]\label{def:semisimple-algebra}
Cho $R$ là một $K$-đại số hữu hạn chiều.  
Ta nói rằng $R$ là một \emph{đại số nửa đơn} (semisimple algebra) nếu mọi $R$-mô đun hữu hạn chiều $M$ đều \emph{khả phân hoàn toàn},  
tức là $M$ có thể viết được dưới dạng tổng trực tiếp của các mô đun con:
\[
M = M_1 \oplus M_2 \oplus \cdots \oplus M_r,
\]
trong đó mỗi $M_i$ là mô đun con của $M$ (không nhất thiết đơn).  

Tương đương, $R$ là nửa đơn khi và chỉ khi mọi dãy ngắn chính xác của $R$-mô đun hữu hạn chiều đều tách:
\[
0 \to M' \longrightarrow M \longrightarrow M'' \to 0
\quad\Rightarrow\quad
M \cong M' \oplus M''.
\]
\end{definition}

\begin{remark}
Khái niệm \emph{đại số nửa đơn} trừu tượng hóa tính chất “khả phân hoàn toàn” từ lý thuyết biểu diễn nhóm.  
Trong bối cảnh này, $R=K[G]$ là đại số nhóm, và mô đun $R$-mô đun chính là các biểu diễn $K$-tuyến tính của nhóm $G$.  
Do đó, việc $R$ là nửa đơn tương đương với việc mọi biểu diễn của $G$ là khả phân hoàn toàn.
\end{remark}


%========================================
% HỆ QUẢ 10 — Kết nối giữa đại số nửa đơn và định lý Maschke
%========================================
\begin{corollary}[Hệ quả 10 — Đại số nhóm là nửa đơn]\label{cor:semisimple-group-algebra}
Với một nhóm hữu hạn $G$ và một trường $K$,  
$K$-đại số nhóm $K[G]$ là \emph{đại số nửa đơn} khi và chỉ khi cấp của nhóm $G$ \emph{không chia hết cho đặc số của $K$},  
tức là $\operatorname{char}(K)\nmid |G|$.
\end{corollary}

\begin{proof}
\textbf{Chiều thuận.}  
Nếu $\operatorname{char}(K)\nmid |G|$, Định lý Maschke cho ta rằng mọi biểu diễn $K$-tuyến tính hữu hạn chiều của $G$ đều khả phân hoàn toàn,  
nghĩa là mọi $K[G]$-mô đun hữu hạn chiều đều tách thành tổng trực tiếp các mô đun con.  
Do đó $K[G]$ là đại số nửa đơn theo Định nghĩa~\ref{def:semisimple-algebra}.

\medskip
\textbf{Chiều đảo.}  
Ngược lại, giả sử $\operatorname{char}(K)\mid |G|$.  
Theo Bài tập~\ref{prob:Maschke}, tồn tại mô đun $K[G]$ có một mô đun con bất biến $W$ không có phần bù bất biến.  
Nói cách khác, tồn tại dãy ngắn chính xác không tách
\[
0 \to W \to K[G] \to K[G]/W \to 0,
\]
cho thấy $K[G]$ không nửa đơn.  

\medskip
\textbf{Kết luận.}  
Như vậy, $K[G]$ là đại số nửa đơn $\iff \operatorname{char}(K)\nmid |G|$, 
điều này chính là phát biểu đại số tổng quát của Định lý Maschke.
\end{proof}

\begin{remark}
\leavevmode
\begin{enumerate}
\item Trong trường hợp đặc biệt $K=\mathbb{C}$, ta luôn có $\operatorname{char}(K)=0$,  
nên mọi đại số nhóm hữu hạn $\mathbb{C}[G]$ đều nửa đơn, và mọi biểu diễn phức của nhóm hữu hạn đều khả phân hoàn toàn.
\item Định lý cơ bản của \textsc{Wedderburn–Artin} khẳng định rằng mọi đại số nửa đơn hữu hạn chiều trên $K$ có dạng
\[
R \cong \bigoplus_{i=1}^r \operatorname{Mat}_{n_i}(D_i),
\]
với $D_i$ là các vành chia hữu hạn chiều trên $K$.  
Trong trường hợp $R=K[G]$ với $\operatorname{char}(K)\nmid |G|$, 
các $D_i$ thực ra là các trường mở rộng của $K$.
\end{enumerate}
\end{remark}

\section{Tích tensor}
%========================================
% BÀI TẬP 1 — Tích tensor của hai K-không gian vectơ
%========================================
\begin{problem}[Tích tensor của hai $K$-không gian vectơ]
Nhắc lại và chứng minh đầy đủ các mệnh đề cơ bản về tích tensor $V\otimes_K W$ của hai $K$-không gian vectơ $V,W$:
\begin{enumerate}
\item \textbf{Định nghĩa phổ dụng (tính chất phổ dụng) của tích tensor.}
\item \textbf{Sự tồn tại và tính duy nhất (lên tới đẳng cấu duy nhất).}
\item \textbf{Tích tensor của hai ánh xạ tuyến tính; hàm tử và tính tự nhiên.}
\item \textbf{Các đẳng cấu chuẩn tắc: giao hoán, kết hợp, đơn vị.}
\item \textbf{Tương thích với tổng trực tiếp hữu hạn.}
\item \textbf{Cơ sở và số chiều: $\dim_K(V\otimes_K W)=\dim_K V\cdot \dim_K W$.}
\end{enumerate}
\end{problem}

\begin{proof}
\textbf{(1) Định nghĩa phổ dụng.}
Một \emph{tích tensor} của $V$ và $W$ là một cặp $(T,\tau)$ gồm một $K$-không gian vectơ $T$
và một ánh xạ song tuyến tính
\[
\tau:V\times W\longrightarrow T
\]
sao cho với mọi $K$-không gian vectơ $U$ và mọi ánh xạ song tuyến tính
$b:V\times W\to U$, tồn tại duy nhất một ánh xạ tuyến tính
\[
\widetilde b:T\longrightarrow U
\quad\text{thỏa}\quad
\widetilde b\circ \tau=b.
\]
Tính chất này gọi là \emph{tính chất phổ dụng}. Khi tồn tại, $T$ được kí hiệu $V\otimes_K W$ 
và $\tau(v,w)$ được kí hiệu $v\otimes w$.

\medskip
\textbf{(2) Tồn tại và duy nhất.}
\emph{Tồn tại:} Lấy $F$ là không gian vectơ tự do sinh bởi tập $V\times W$ với kí hiệu phần tử sinh $[v,w]$.
Xét $R\subset F$ là không gian con sinh bởi các quan hệ song tuyến tính:
\begin{align*}
[v_1+v_2,w]-[v_1,w]-[v_2,w],\quad
[v,\;w_1+w_2]-[v,w_1]-[v,w_2],\quad
[\lambda v,w]-\lambda[v,w],\quad
[v,\lambda w]-\lambda[v,w],
\end{align*}
với mọi $v,v_1,v_2\in V$, $w,w_1,w_2\in W$, $\lambda\in K$.
Đặt $T:=F/R$ và định nghĩa $\tau(v,w):=[v,w]\!\!\mod R$.
Rõ ràng $\tau$ là song tuyến tính. Nếu $b:V\times W\to U$ là song tuyến tính, 
thì ánh xạ từ tập sinh $V\times W$ gửi $[v,w]\mapsto b(v,w)$ kéo dài duy nhất thành một ánh xạ tuyến tính $F\to U$;
vì $b$ tôn trọng các quan hệ nên $R$ đi vào hạt nhân, do đó cảm sinh một ánh xạ tuyến tính duy nhất 
$\widetilde b:T=F/R\to U$ với $\widetilde b\circ \tau=b$.

\emph{Duy nhất lên tới đẳng cấu:} Nếu $(T,\tau)$ và $(T',\tau')$ đều thỏa tính chất phổ dụng, 
từ $b=\tau'$ suy ra tồn tại duy nhất $\phi:T\to T'$ với $\phi\circ \tau=\tau'$; 
tương tự tồn tại $\psi:T'\to T$ với $\psi\circ \tau'=\tau$. 
Khi đó $\psi\circ\phi$ (tương ứng $\phi\circ\psi$) là tự đồng cấu của $T$ (tương ứng của $T'$) 
giữ cố định $\tau$ (tương ứng $\tau'$), nên là đồng nhất. Suy ra $\phi,\psi$ là đẳng cấu ngược nhau.

\medskip
\textbf{(3) Tích tensor của ánh xạ tuyến tính; tính hàm tử.}
Cho các ánh xạ tuyến tính $f:V\to V'$ và $g:W\to W'$.
Ánh xạ
\[
V\times W \longrightarrow V'\otimes W',\qquad (v,w)\longmapsto f(v)\otimes g(w)
\]
là song tuyến tính; bởi tính chất phổ dụng, tồn tại duy nhất ánh xạ tuyến tính
\[
f\otimes g:\; V\otimes W \longrightarrow V'\otimes W',\qquad
(f\otimes g)(v\otimes w)=f(v)\otimes g(w).
\]
Dễ kiểm tra $({\rm id}_V\otimes {\rm id}_W)={\rm id}_{V\otimes W}$ và
\[
(f_2\circ f_1)\otimes (g_2\circ g_1)=(f_2\otimes g_2)\circ (f_1\otimes g_1).
\]
Vì vậy, phép gán $(V,W)\mapsto V\otimes W$, $(f,g)\mapsto f\otimes g$ là hàm tử đối xứng trong phạm trù các $K$-không gian vectơ.

\medskip
\textbf{(4) Các đẳng cấu chuẩn tắc (tự nhiên).}
\begin{itemize}
\item \emph{Giao hoán:} Định nghĩa
$\tau_{V,W}:V\otimes W\to W\otimes V$ bởi $\tau_{V,W}(v\otimes w)=w\otimes v$.
Ánh xạ này là đẳng cấu tự nhiên (nghịch đảo chính nó).

\item \emph{Kết hợp:} Định nghĩa
\[
\alpha_{U,V,W}:(U\otimes V)\otimes W\longrightarrow U\otimes (V\otimes W),\qquad
\alpha\big((u\otimes v)\otimes w\big)=u\otimes (v\otimes w).
\]
Tính chất phổ dụng đảm bảo $\alpha$ là đẳng cấu tự nhiên; nghịch đảo được cho bởi
$u\otimes(v\otimes w)\mapsto (u\otimes v)\otimes w$.

\item \emph{Đơn vị:} Với $K$ coi như $K$-không gian vectơ một chiều, đặt
\[
\lambda_V:K\otimes V\to V,\quad \lambda_V(a\otimes v)=av,
\qquad
\rho_V:V\otimes K\to V,\quad \rho_V(v\otimes a)=av.
\]
Cả hai là đẳng cấu tự nhiên, nghịch đảo lần lượt gửi $v\mapsto 1\otimes v$ và $v\mapsto v\otimes 1$.
\end{itemize}
Các đẳng cấu trên thoả các tiên đề chặt chẽ (coherence) quen thuộc: 
tam giác và ngũ giác của Mac Lane.

\medskip
\textbf{(5) Tương thích với tổng trực tiếp hữu hạn.}
Cho $V=\bigoplus_{i=1}^r V_i$. Xét song tuyến tính
\[
\bigoplus_{i=1}^r V_i \times W \longrightarrow \bigoplus_{i=1}^r (V_i\otimes W),\qquad
\Big(\sum_i v_i,\; w\Big)\longmapsto \sum_i (v_i\otimes w).
\]
Theo tính chất phổ dụng, cảm sinh một tuyến tính
\[
\Phi:\; \Big(\bigoplus_{i=1}^r V_i\Big)\otimes W \longrightarrow \bigoplus_{i=1}^r (V_i\otimes W),
\qquad
\Phi\big((v_1,\dots,v_r)\otimes w\big)=\bigoplus_i (v_i\otimes w).
\]
Ngược lại, định nghĩa $\Psi$ trên từng toạ độ bởi
$\Psi(v_i\otimes w)=(0,\dots,0,v_i,\dots,0)\otimes w$,
rồi tuyến tính hoá để được
\[
\Psi:\; \bigoplus_{i=1}^r (V_i\otimes W)\longrightarrow \Big(\bigoplus_{i=1}^r V_i\Big)\otimes W.
\]
Dễ thấy $\Phi\circ\Psi=\mathrm{id}$ và $\Psi\circ\Phi=\mathrm{id}$, do đó
\[
\Big(\bigoplus_{i=1}^r V_i\Big)\otimes W \;\cong\; \bigoplus_{i=1}^r (V_i\otimes W).
\]
Tương tự, $V\otimes\big(\bigoplus_{j=1}^s W_j\big)\cong \bigoplus_{j=1}^s (V\otimes W_j)$.
Các đẳng cấu này là \emph{tự nhiên} theo mọi biến.

\medskip
\textbf{(6) Cơ sở và số chiều.}
Giả sử $\{v_1,\dots,v_m\}$ là cơ sở của $V$ và $\{w_1,\dots,w_n\}$ là cơ sở của $W$.
\begin{itemize}
\item \emph{Sinh ra:} Với $v=\sum_i a_i v_i$, $w=\sum_j b_j w_j$, ta có
\[
v\otimes w=\sum_{i,j} a_i b_j\, (v_i\otimes w_j),
\]
nên các phần tử $v_i\otimes w_j$ sinh $V\otimes W$.
\item \emph{Độc lập tuyến tính:} Xét tuyến tính $b:V\times W\to K^{mn}$ gửi
$b(v_i,w_j)=e_{(i,j)}$ (các vectơ cơ sở) và song tuyến tính hoá.
Tính chất phổ dụng cho ánh xạ tuyến tính $\widetilde b:V\otimes W\to K^{mn}$ với
$\widetilde b(v_i\otimes w_j)=e_{(i,j)}$. 
Nếu $\sum_{i,j} c_{ij}(v_i\otimes w_j)=0$ thì áp dụng $\widetilde b$ được
$\sum_{i,j} c_{ij}e_{(i,j)}=0$, suy ra $c_{ij}=0$. Do đó $\{v_i\otimes w_j\}$ độc lập.
\end{itemize}
Vậy $\{v_i\otimes w_j\}_{1\le i\le m,\,1\le j\le n}$ là cơ sở của $V\otimes W$, và
\[
\dim_K(V\otimes W)=mn=(\dim_K V)(\dim_K W).
\qedhere\]
\end{proof}

%========================================
% Hệ quả: Mô tả ánh xạ song tuyến tính qua tensor
%========================================
\begin{proposition}[Đồng cấu hoá song tuyến tính]\label{prop:bilinear-linear}
Với mọi $K$-không gian vectơ $U$ có đẳng cấu tự nhiên
\[
\Hom_K(V\otimes W,\;U)\;\cong\;\Bil_K(V\times W,\;U),
\]
gửi $\Phi:V\otimes W\to U$ tới ánh xạ song tuyến tính $(v,w)\mapsto \Phi(v\otimes w)$,
và nghịch đảo gửi $b$ tới $\widetilde b$ như trong tính chất phổ dụng.
Các đẳng cấu này tự nhiên theo ba biến $V,W,U$.
\end{proposition}

%========================================
% Một số đẳng cấu chuẩn khác (tuỳ chọn)
%========================================
\begin{remark}
Các đẳng cấu tự nhiên sau thường được dùng:
\[
V^\ast\otimes W \;\cong\; \Hom_K(V,W),\qquad
\Hom_K(V\otimes W,\,U)\;\cong\;\Hom_K\big(V,\Hom_K(W,U)\big),
\]
được cảm sinh từ \eqref{prop:bilinear-linear}. 
Trong cơ sở hữu hạn, đẳng cấu đầu là $f\otimes w \mapsto (v\mapsto f(v)w)$.
\end{remark}

\begin{problem}[Bài tập 2]
Giả sử $X,Y$ là hai tập hữu hạn. Với mọi tập $S$, kí hiệu $K[S]$ là $K$-không gian vectơ của các hàm $S\to K$
(tức không gian vectơ tự do trên $S$). Hãy xây dựng một đẳng cấu
\[
\Phi:\;K[X]\otimes_K K[Y]\;\xrightarrow{\ \sim\ }\;K[X\times Y].
\]
\end{problem}

\begin{proof}
\textbf{Bước 1: Cơ sở chuẩn và kí hiệu.}
Với $x\in X$ đặt $\delta_x\in K[X]$ là hàm đặc trưng của $\{x\}$:
$\delta_x(x)=1$ và $\delta_x(x')=0$ nếu $x'\neq x$. Tương tự, với $y\in Y$, đặt
$\delta_y\in K[Y]$.
Tập $\{\delta_x\}_{x\in X}$ là một cơ sở của $K[X]$, và $\{\delta_y\}_{y\in Y}$ là cơ sở của $K[Y]$.
Trên tích $X\times Y$, đặt $\delta_{(x,y)}$ là hàm đặc trưng của $\{(x,y)\}$; 
$\{\delta_{(x,y)}\}_{(x,y)\in X\times Y}$ là cơ sở của $K[X\times Y]$.

\medskip
\textbf{Bước 2: Định nghĩa ánh xạ ứng viên.}
Xét ánh xạ song tuyến tính
\[
b:X\times Y\longrightarrow K[X\times Y],\qquad
b(x,y):=\delta_{(x,y)}.
\]
Bằng tuyến tính hoá theo tính chất phổ dụng của tích tensor, tồn tại duy nhất
\[
\Phi:K[X]\otimes K[Y]\longrightarrow K[X\times Y]
\quad\text{thỏa}\quad
\Phi(\delta_x\otimes \delta_y)=\delta_{(x,y)}.
\]
Với phần tử tổng quát $f=\sum_{x}a_x\delta_x\in K[X]$ và $g=\sum_{y}b_y\delta_y\in K[Y]$,
ta có công thức hiển nhiên
\[
\Phi(f\otimes g)\;=\;\sum_{x\in X}\sum_{y\in Y} a_x\,b_y\,\delta_{(x,y)}.
\]

\medskip
\textbf{Bước 3: Surjectivity.}
Các phần tử $\delta_{(x,y)}$ sinh $K[X\times Y]$; mỗi $\delta_{(x,y)}$ là ảnh của
$\delta_x\otimes \delta_y$ qua $\Phi$. Vì vậy $\Phi$ là toàn cấu.

\medskip
\textbf{Bước 4: Injectivity (đếm chiều).}
Do $X,Y$ hữu hạn, 
\[
\dim_K\bigl(K[X]\otimes K[Y]\bigr)
= \bigl(\dim_K K[X]\bigr)\cdot\bigl(\dim_K K[Y]\bigr)
= |X|\cdot |Y|
= \dim_K K[X\times Y].
\]
Ánh xạ tuyến tính toàn cấu giữa hai không gian cùng chiều là đẳng cấu, nên $\Phi$ khả nghịch.

\medskip
\textbf{Bước 5: Mô tả nghịch đảo (tường minh).}
Định nghĩa $\Psi:K[X\times Y]\to K[X]\otimes K[Y]$ trên cơ sở bởi
\[
\Psi(\delta_{(x,y)}) := \delta_x\otimes \delta_y
\qquad\text{và mở rộng tuyến tính.}
\]
Khi đó $\Psi\circ \Phi$ và $\Phi\circ \Psi$ đều là đồng nhất trên các cơ sở tương ứng,
nên $\Psi=\Phi^{-1}$.

\medskip
\textbf{Kết luận.}
$\Phi$ là đẳng cấu tự nhiên $K[X]\otimes K[Y]\xrightarrow{\sim}K[X\times Y]$ gửi
$\delta_x\otimes\delta_y\mapsto \delta_{(x,y)}$.
\end{proof}

\begin{remark}[Tính tự nhiên]
Đẳng cấu trên là tự nhiên theo cả hai biến: với mọi ánh xạ $u:X\to X'$ và $v:Y\to Y'$,
sơ đồ
\[\begin{tikzcd}[ampersand replacement=\&,cramped]
	{\K[X]\otimes \K[Y]} \&\& {\K[X\times Y]} \\
	\\
	{\K[X']\otimes \K[Y']} \&\& {\K[X\times Y]}
	\arrow["{\phi_{X,Y}}", from=1-1, to=1-3]
	\arrow["{\K[u]\otimes \K[v]}"', from=1-1, to=3-1]
	\arrow["{\K[u\times v]}", from=1-3, to=3-3]
	\arrow["{\phi_{X',Y'}}"', from=3-1, to=3-3]
\end{tikzcd}\]

giao hoán. (Ở đây $K[u]$ là ánh xạ kéo theo trên không gian hàm.)
\end{remark}

%==============================
% BÀI TẬP 3
%==============================
\begin{problem}
Chứng minh rằng tương ứng
\[
(S,T) \;\longmapsto\; S\otimes T
\]
cảm sinh một đẳng cấu tự nhiên
\[
\Hom(V_1,V_2)\otimes \Hom(W_1,W_2)
\;\xrightarrow{\ \sim\ }\;
\Hom(V_1\otimes W_1,\; V_2\otimes W_2).
\]
\end{problem}

\begin{proof}
Ta định nghĩa ánh xạ
\[
\Phi : \Hom(V_1,V_2)\otimes \Hom(W_1,W_2)
\;\longrightarrow\;
\Hom(V_1\otimes W_1,\; V_2\otimes W_2)
\]
bởi quy tắc
\[
\Phi(S\otimes T)(v\otimes w) \;:=\; S(v)\otimes T(w),
\quad \forall\, S\in\Hom(V_1,V_2),\;
T\in\Hom(W_1,W_2),\;
v\in V_1,\;
w\in W_1.
\]

\smallskip
\textbf{(1) Well-definedness.}
Trước hết, quy tắc trên là tuyến tính theo từng biến $S,T$,
nên bởi tính chất phổ dụng của tích tensor,
nó cảm sinh duy nhất một ánh xạ tuyến tính $\Phi$.

\smallskip
\textbf{(2) Tính tuyến tính.}
Với mọi $\lambda,\mu\in K$ và $S_1,S_2\in\Hom(V_1,V_2)$, $T_1,T_2\in\Hom(W_1,W_2)$,
ta có
\[
\Phi\big((\lambda S_1+\mu S_2)\otimes T\big)
= \lambda\,\Phi(S_1\otimes T) + \mu\,\Phi(S_2\otimes T),
\]
và tương tự với biến $T$.
Do đó $\Phi$ là ánh xạ tuyến tính.

\smallskip
\textbf{(3) Tính song ánh.}
Giờ ta chứng minh $\Phi$ là một đẳng cấu.
Chọn các cơ sở $(e_i)$ của $V_1$, $(f_j)$ của $W_1$,
và các cơ sở tương ứng $(e_i')$, $(f_j')$ của $V_2$, $W_2$.
Khi đó, mỗi phần tử $S\in\Hom(V_1,V_2)$ được biểu diễn bằng một ma trận $(s_{i'i})$,
và tương tự $T$ bằng $(t_{j'j})$.

Khi đó ánh xạ $S\otimes T$ tương ứng với ma trận tensor $(s_{i'i}t_{j'j})$,
nghĩa là $S\otimes T$ tác động lên phần tử cơ sở $e_i\otimes f_j$ theo công thức:
\[
(S\otimes T)(e_i\otimes f_j) = \sum_{i',j'} s_{i'i}t_{j'j}\, e_{i'}'\otimes f_{j'}'.
\]
Từ đó, không gian $\Hom(V_1\otimes W_1, V_2\otimes W_2)$
có cơ sở tự nhiên tương ứng với mọi cặp chỉ số $(i,i',j,j')$,
và ta thấy rõ rằng ánh xạ $\Phi$ gửi cơ sở của vế trái
sang cơ sở tương ứng của vế phải — do đó $\Phi$ là đẳng cấu tuyến tính.

\smallskip
\textbf{(4) Tính tự nhiên.}
Đẳng cấu $\Phi$ tương thích với mọi ánh xạ tuyến tính $U_1\to V_1$, $V_2\to U_2$, v.v.
Nói cách khác, $\Phi$ là một đẳng cấu \emph{tự nhiên} theo bốn biến $V_1,V_2,W_1,W_2$,
nghĩa là biểu đồ sau giao hoán:
\[\begin{tikzcd}[ampersand replacement=\&,cramped]
	{\Hom(V_1,V_2)\otimes \Hom(W_1,W_2)} \&\& {\Hom(V_1\otimes W_1,V_2\otimes W_2)} \\
	\\
	{\Hom(V_1',V_2')\otimes \Hom(W_1',W_2')} \&\& {\Hom(V_1'\otimes V_2',W_1'\otimes W_2')}
	\arrow["{\phi_{V,W}}", from=1-1, to=1-3]
	\arrow["{\Hom(f_1,f_2)\otimes \Hom(g_1,g_2)}"', from=1-1, to=3-1]
	\arrow["{\Hom(f_1\otimes g_1,f_2\otimes g_2)}", from=1-3, to=3-3]
	\arrow["{\phi_{V',W'}}"', from=3-1, to=3-3]
\end{tikzcd}\]
\smallskip
\textbf{(5) Kết luận.}
Do $\Phi$ là song ánh tuyến tính, ta có được đẳng cấu:
\[
\boxed{
\Hom(V_1,V_2)\otimes \Hom(W_1,W_2)
\;\cong\;
\Hom(V_1\otimes W_1,\; V_2\otimes W_2).
}
\]
Đây là đẳng cấu \emph{chuẩn tắc} trong phạm trù các không gian vectơ,
và thường được gọi là \emph{đẳng cấu tensor của các ánh xạ tuyến tính.}
\end{proof}

%========================================
% BÀI TẬP 4: Tính nhân của vết theo tích tensor
%========================================
\begin{problem}
Chứng minh rằng đối với các tự đồng cấu tuyến tính
\(S:V\to V\) và \(T:W\to W\),
ta có công thức
\[
\operatorname{tr}(S\otimes T)
= \big(\operatorname{tr} S\big)\,\big(\operatorname{tr} T\big),
\]
trong đó \(\operatorname{tr}\) ký hiệu vết (trace) của một tự đồng cấu tuyến tính.
\end{problem}

\begin{proof}
Giả sử \(\dim_K V = m\) và \(\dim_K W = n\).
Chọn các cơ sở \((v_1,\dots,v_m)\) của \(V\) và \((w_1,\dots,w_n)\) của \(W\).
Trong các cơ sở này, ta có:

\[
S(v_i) = \sum_{p=1}^m s_{pi}\,v_p,
\qquad
T(w_j) = \sum_{q=1}^n t_{qj}\,w_q.
\]

Khi đó, với cơ sở tensor \((v_i\otimes w_j)_{1\le i\le m,\,1\le j\le n}\)
của \(V\otimes W\),
phép biến đổi \(S\otimes T\) tác động như sau:
\[
(S\otimes T)(v_i\otimes w_j)
= S(v_i)\otimes T(w_j)
= \sum_{p,q} s_{pi}\,t_{qj}\, (v_p\otimes w_q).
\]

Do đó, ma trận của \(S\otimes T\) đối với cơ sở này có dạng
\[
M_{S\otimes T} = (s_{pi}\,t_{qj})_{(p,q),(i,j)},
\]
nghĩa là chỉ số hàng là cặp \((p,q)\), chỉ số cột là cặp \((i,j)\).

\smallskip
Vết của \(S\otimes T\) là tổng của các phần tử trên đường chéo chính, tức là:
\[
\operatorname{tr}(S\otimes T)
= \sum_{i,j} s_{ii}\,t_{jj}
= \left(\sum_i s_{ii}\right)\!\!\left(\sum_j t_{jj}\right)
= (\operatorname{tr}S)\,(\operatorname{tr}T).
\]

\smallskip
Vì vết của một ánh xạ tuyến tính là độc lập với việc chọn cơ sở,
nên đẳng thức này đúng với mọi lựa chọn cơ sở của \(V\) và \(W\).

\smallskip
\textbf{Kết luận.} Do đó, ta có:
\[
\boxed{
\operatorname{tr}(S\otimes T)
= (\operatorname{tr}S)(\operatorname{tr}T).
}
\]
\end{proof}

%========================================
% ĐỊNH NGHĨA — Tích tensor của hai biểu diễn
%========================================
\begin{definition}[Tích tensor ngoài và trong của biểu diễn]\label{def:outer-inner-tensor-rep}
Cho $(\rho,V)$ là một biểu diễn $K$-tuyến tính của nhóm $G$ và $(\sigma,W)$ là một biểu diễn của nhóm $H$.
\begin{enumerate}
\item \emph{Tích tensor ngoài} của $(\rho,V)$ và $(\sigma,W)$ là biểu diễn của $G\times H$ trên
$V\otimes_K W$ cho bởi
\[
(\rho \boxtimes \sigma)(g,h)\,(v\otimes w)
\;:=\; \rho(g)v \;\otimes\; \sigma(h)w
\qquad (g\in G,\;h\in H,\;v\in V,\;w\in W).
\]
Kí hiệu: $(\rho\boxtimes \sigma,\; V\otimes W)$.

\item Khi $G=H$ và ta muốn một biểu diễn của \emph{cùng} nhóm $G$ trên $V\otimes W$, ta thu hẹp
tích tensor ngoài dọc theo đường chéo $\Delta:G\to G\times G$, $\Delta(g)=(g,g)$.
Biểu diễn thu được gọi là \emph{tích tensor trong} (cũng kí hiệu $\rho\otimes\sigma$):
\[
(\rho\otimes \sigma)(g)\,(v\otimes w)
\;:=\; \rho(g)v \;\otimes\; \sigma(g)w
\qquad (g\in G).
\]
\end{enumerate}
\end{definition}

\begin{remark}
Các phép dựng trên là \emph{tự nhiên} theo phép đổi cơ sở và thỏa các đẳng cấu chuẩn tắc
(giao hoán, kết hợp, đơn vị) của tích tensor các không gian vectơ; vì vậy phạm trù các biểu diễn của $G$
là một phạm trù tensor (monoidal) với tích tensor trong.
\end{remark}


%========================================
% BÀI TẬP 6 — Đồng nhất $V^\vee\otimes W \cong \Hom_K(V,W)$
%========================================
\begin{problem}[Đẳng cấu $V^\vee\otimes W \simeq \Hom_K(V,W)$]\label{prob:dual-tensor-hom}
Cho $V$ là một $K$-không gian vectơ hữu hạn chiều và $V^\vee:=\Hom_K(V,K)$ là không gian đối ngẫu.
Với mọi $W$, xét ánh xạ song tuyến tính
\[
\beta: V^\vee \times W \longrightarrow \Hom_K(V,W),
\qquad
(\xi,y) \longmapsto \big(x \mapsto \xi(x)\,y\big).
\]
Chứng minh rằng $\beta$ cảm sinh một đẳng cấu tự nhiên
\[
\Phi_{V,W}\;:\; V^\vee \otimes_K W \;\xrightarrow{\ \sim\ }\; \Hom_K(V,W).
\]
\end{problem}

\begin{proof}
\textbf{(1) Tồn tại duy nhất của $\Phi_{V,W}$.}
Ánh xạ $\beta$ là song tuyến tính theo định nghĩa; bởi tính chất phổ dụng của tích tensor,
tồn tại duy nhất một ánh xạ tuyến tính
\[
\Phi_{V,W}:V^\vee\otimes W\to \Hom_K(V,W)
\quad\text{sao cho}\quad
\Phi_{V,W}(\xi\otimes y)(x)=\xi(x)\,y.
\]

\medskip
\textbf{(2) $\Phi_{V,W}$ là đẳng cấu.}
Chọn cơ sở $(e_1,\dots,e_n)$ của $V$ và đối cơ sở $(e^1,\dots,e^n)$ của $V^\vee$.
Với mọi $y\in W$ ta có
\[
\Phi_{V,W}(e^i\otimes y)(e_j)= e^i(e_j)\,y=\delta_{ij}\,y,
\]
suy ra các phần tử $E_{ij}(y):=\Phi_{V,W}(e^i\otimes y)$ tạo nên một hệ sinh của $\Hom(V,W)$.
Từ đó thấy $\Phi_{V,W}$ là toàn cấu. Hơn nữa,
\[
\dim (V^\vee\otimes W)= (\dim V)(\dim W)=\dim \Hom(V,W),
\]
nên toàn cấu giữa hai không gian cùng chiều là đẳng cấu. (Hoặc có thể viết nghịch đảo tường minh: 
$v\mapsto \sum_i e^i(v)\,y_i$ với $y_i=\Psi(T)$ xác định bởi $T(e_i)=y_i$.)

\medskip
\textbf{(3) Tính tự nhiên.}
Với $f:V'\to V$ và $g:W\to W'$,
sơ đồ
\[\begin{tikzcd}[ampersand replacement=\&,cramped]
	{V^\vee\otimes W} \&\& {\Hom(V,W)} \\
	\\
	{(V')^\vee\otimes W' } \&\& {\Hom(V',W')}
	\arrow["{\phi_{V,W}}", from=1-1, to=1-3]
	\arrow["{f^\vee\otimes g}"', from=1-1, to=3-1]
	\arrow["{\Hom(f,g)}", from=1-3, to=3-3]
	\arrow["{\phi_{V',W'}}"', from=3-1, to=3-3]
\end{tikzcd}\]
giao hoán (kiểm tra trên phần tử thuần $\xi\otimes y$), nên $\Phi_{V,W}$ là đẳng cấu tự nhiên theo $V,W$.
\end{proof}

\begin{remark}[Tính tương thích với cấu trúc biểu diễn]\label{rmk:equivariance}
Giả sử $(\rho,V)$ là biểu diễn của $G$ và $(\sigma,W)$ là biểu diễn của $H$.
\begin{enumerate}
\item \textbf{Ngoài:} $V^\vee$ mang biểu diễn đối ngẫu $\rho^\vee(g)\xi:=\xi\circ \rho(g)^{-1}$.
Trên $V^\vee\otimes W$ ta có biểu diễn ngoài $(\rho^\vee\boxtimes \sigma)$ của $G\times H$.
Trên $\Hom(V,W)$, cho tác động chuẩn
\[
((g,h)\cdot T)\ :=\ \sigma(h)\circ T \circ \rho(g)^{-1}.
\]
Khi đó \emph{$\Phi_{V,W}$ là một đẳng cấu $G\times H$-biểu diễn}:
\[
\Phi_{V,W}\big((\rho^\vee(g)\xi)\otimes (\sigma(h)y)\big)
\;=\;
((g,h)\cdot \Phi_{V,W}(\xi\otimes y)).
\]

\item \textbf{Trong:} Nếu $G=H$ và xét tác động chéo $(g\cdot T)=\sigma(g)T\rho(g)^{-1}$ trên $\Hom(V,W)$,
còn trên $V^\vee\otimes W$ dùng biểu diễn trong $(\rho^\vee\otimes \sigma)$,
thì $\Phi_{V,W}$ vẫn là một đẳng cấu $G$-biểu diễn.
\end{enumerate}
Ý nghĩa: \(\Phi_{V,W}\) không chỉ là đẳng cấu của các không gian vectơ, mà còn là \emph{morphisme đan dệt} (intertwiner) tự nhiên giữa các biểu diễn được dựng từ $V$ và $W$.
\end{remark}

%========================================
% HỆ QUẢ — Công thức đánh giá và đồng cấu học
%========================================
\begin{corollary}[Đánh giá và đồng đánh giá]
Với cơ sở ký hiệu chuẩn, ánh xạ đánh giá 
$\mathrm{ev}_V: V^\vee\otimes V\to K$, $(\xi\otimes v)\mapsto \xi(v)$,
và đồng đánh giá
$\mathrm{coev}_V: K\to V\otimes V^\vee$, $1\mapsto \sum_i e_i\otimes e^i$,
thỏa các tam giác hợp nhất (zig–zag) của phạm trù tensor cứng (rigid).
Qua \(\Phi_{V,V}\), $\mathrm{ev}_V$ chính là vết $\mathrm{tr}:\End(V)\to K$ trên các hạng tử hạng một.
\end{corollary}

%===========================================================
% BÀI TẬP 7 — Tích tensor và Hom của các biểu diễn
%===========================================================
\begin{problem}\label{prob:tensor-hom-GxH}
Cho $(\rho,V)$ và $(\sigma,W)$ lần lượt là các biểu diễn của các nhóm $G$ và $H$.  
Khi đó, $(\rho'\otimes\sigma,\, V^\vee\otimes W)$ là một biểu diễn của $G\times H$, trong đó
\[
(\rho'\otimes\sigma)(g,h)(\xi\otimes w)
\;=\;
(\rho'(g)\xi)\otimes(\sigma(h)w)
\;=\;
(\xi\circ\rho(g)^{-1})\otimes\sigma(h)w.
\]

Mặt khác, không gian $\Hom(V,W)$ (đẳng cấu chính tắc với $V^\vee\otimes W$)
cũng là một biểu diễn của $G\times H$, với tác động định nghĩa bởi
\[
\tau:G\times H\longrightarrow \GL(\Hom(V,W)),\qquad
\tau(g,h)(T)=\sigma(h)\circ T\circ\rho(g)^{-1}.
\]

\textbf{Yêu cầu.} Chứng minh rằng đẳng cấu tuyến tính
\[
\Phi_{V,W}:V^\vee\otimes W \longrightarrow \Hom(V,W),\qquad
\Phi_{V,W}(\xi\otimes w)(v)=\xi(v)\,w,
\]
là một \emph{đẳng cấu của các biểu diễn $G\times H$}.  
Từ đó suy ra rằng $V^\vee\otimes W$ và $\Hom(V,W)$ là hai biểu diễn đẳng cấu của $G\times H$.
\end{problem}

%===========================================================
% CHỨNG MINH CHI TIẾT
%===========================================================
\begin{proof}
Ta cần chứng minh rằng $\Phi_{V,W}$ là một đồng cấu biểu diễn, tức là
\[
\Phi_{V,W}\big((\rho'\otimes\sigma)(g,h)(\xi\otimes w)\big)
\;=\;
\tau(g,h)\big(\Phi_{V,W}(\xi\otimes w)\big)
\qquad\text{với mọi }(g,h)\in G\times H.
\]

\medskip
\noindent\textbf{Vế trái.}
\begin{align*}
\Phi_{V,W}\big((\rho'\otimes\sigma)(g,h)(\xi\otimes w)\big)
&= \Phi_{V,W}\big((\rho'(g)\xi)\otimes(\sigma(h)w)\big) \\
&= \Phi_{V,W}\big((\xi\circ\rho(g)^{-1})\otimes(\sigma(h)w)\big) \\
&: v \longmapsto (\xi\circ\rho(g)^{-1})(v)\cdot\sigma(h)w \\
&= \xi(\rho(g)^{-1}v)\,\sigma(h)w.
\end{align*}

\noindent\textbf{Vế phải.}
\begin{align*}
\tau(g,h)\big(\Phi_{V,W}(\xi\otimes w)\big)
&= \sigma(h)\circ\Phi_{V,W}(\xi\otimes w)\circ\rho(g)^{-1}, \\
\text{do đó}\quad
(\tau(g,h)\Phi_{V,W}(\xi\otimes w))(v)
&= \sigma(h)\big(\Phi_{V,W}(\xi\otimes w)(\rho(g)^{-1}v)\big) \\
&= \sigma(h)\big(\xi(\rho(g)^{-1}v)\,w\big)
= \xi(\rho(g)^{-1}v)\,\sigma(h)w.
\end{align*}

Hai biểu thức trùng nhau, nên đẳng thức cần chứng minh đúng.

\medskip
\noindent\textbf{Kết luận.}
Vì $\Phi_{V,W}$ là một đẳng cấu tuyến tính và đồng thời bảo toàn tác động của $G\times H$, nên
\[
\Phi_{V,W}: (V^\vee\otimes W,\rho'\otimes\sigma)
\;\xrightarrow{\ \sim\ }\;
(\Hom(V,W),\tau)
\]
là một đẳng cấu của các biểu diễn $G\times H$.
\end{proof}

%===========================================================
% NHẬN XÉT — Ý nghĩa phạm trù
%===========================================================
\begin{remark}[Ý nghĩa phạm trù]
Kết quả trên cho thấy phép tương ứng
\[
(V,W)\longmapsto \Hom(V,W)
\]
là \emph{tự nhiên} đối với các biểu diễn của $G$ và $H$.  
Hơn nữa, ánh xạ $\Phi_{V,W}$ là một \textit{biến đổi tự nhiên} giữa các functor
\[
(-)^\vee\otimes (-)\quad\text{và}\quad \Hom(-,-),
\]
trong phạm trù các biểu diễn của nhóm.  
Đây là ví dụ điển hình cho tính tương thích tensor–Hom trong đại số tuyến tính.
\end{remark}

%========================================================
% BÀI TẬP 8 — Tích tensor ngoài của các biểu diễn đơn
% Giả sử K đóng đại số và char(K) ∤ |G|, |H|
%========================================================
\begin{problem}\label{prob:outer-simple}
Giả sử $K$ là trường đóng đại số và đặc số của $K$ không chia hết $|G|$ cũng như $|H|$.
\begin{enumerate}
\item Chứng minh rằng nếu $(\rho,V)$ và $(\sigma,W)$ là các biểu diễn \emph{đơn} của $G$ và $H$, thì
tích tensor ngoài $(\rho\boxtimes\sigma,\;V\otimes W)$ là một biểu diễn \emph{đơn} của $G\times H$.
\item Hơn nữa, nếu $(\tau,U)$ và $(\theta,X)$ là các biểu diễn đơn của $G$ và $H$ sao cho
$\rho\boxtimes\sigma \cong \tau\boxtimes\theta$, thì $\rho\cong\tau$ và $\sigma\cong\theta$.
\end{enumerate}
\end{problem}

\begin{proof}
Trong toàn bộ lời giải, vì $\operatorname{char}(K)\nmid |G|,|H|$, theo Maschke
mọi biểu diễn hữu hạn chiều của $G$, $H$ và $G\times H$ đều khả phân hoàn toàn.  
Ta sẽ dùng \textbf{Bổ đề Schur}: nếu $K$ đóng đại số thì
\[
\End_G(V)\cong K \quad\text{khi $V$ đơn}, \qquad
\Hom_G(V,U)=\begin{cases}
0,& V\not\cong U,\\
K,& V\cong U.
\end{cases}
\]

Ngoài ra, ta dùng đẳng cấu tự nhiên (đã chứng minh trong Bài tập~3 và các mệnh đề về tính tương thích với tác động nhóm):
\begin{equation}\label{eq:Hom-factor}
\Hom_{G\times H}(V\otimes W,\;U\otimes X)
\;\cong\;
\Hom_G(V,U)\;\otimes\;\Hom_H(W,X).
\end{equation}

\medskip
\noindent\textbf{(a) $\rho\boxtimes\sigma$ là đơn.}
Xét vành nội đồng cấu của biểu diễn ngoài:
\[
\End_{G\times H}(V\otimes W)
\;\overset{\eqref{eq:Hom-factor}}{\cong}\;
\End_G(V)\otimes \End_H(W)
\;\cong\; K\otimes K \;\cong\; K,
\]
vì $V,W$ đơn (Bổ đề Schur).  

Giả sử $0\neq Y\subsetneq V\otimes W$ là một mô-đun con $G\times H$-bất biến.
Vì Maschke, tồn tại phép chiếu $G\times H$-tuyến tính $P:V\otimes W\to V\otimes W$
lên $Y$ (theo một phần bù bất biến). Do đó $P\in \End_{G\times H}(V\otimes W)$ và $P^2=P$.
Nhưng trong một \emph{trường} (ở đây là $K$), phần tử lũy đẳng chỉ có thể là $0$ hoặc $1$,
suy ra $P=0$ (tức $Y=0$) hoặc $P=1$ (tức $Y=V\otimes W$).  
Vậy không tồn tại mô-đun con bất biến phi tầm thường; $(\rho\boxtimes\sigma,V\otimes W)$ là \emph{đơn}.

\medskip
\noindent\textbf{(b) Tính duy nhất của phân rã.}
Giả sử $\rho\boxtimes\sigma \cong \tau\boxtimes\theta$, tức tồn tại một đẳng cấu
$\Phi\in \Hom_{G\times H}(V\otimes W,\;U\otimes X)$ khác $0$.
Theo \eqref{eq:Hom-factor},
\[
0\neq \Hom_{G\times H}(V\otimes W,\;U\otimes X)
\;\cong\;
\Hom_G(V,U)\otimes \Hom_H(W,X).
\]
Vì $V,U$ là $G$-biểu diễn đơn và $W,X$ là $H$-biểu diễn đơn, áp dụng Bổ đề Schur cho ta
$\Hom_G(V,U)\in\{0,K\}$ và $\Hom_H(W,X)\in\{0,K\}$.
Tích tensor của hai không gian vectơ là $0$ khi và chỉ khi một trong hai thừa số là $0$; do đó để vế phải khác $0$,
ta phải có \(\Hom_G(V,U)\cong K\) và \(\Hom_H(W,X)\cong K\).
Vì thế \(V\cong U\) và \(W\cong X\), tức \(\rho\cong\tau\) và \(\sigma\cong\theta\).

\medskip
Kết luận, (a) và (b) đều được chứng minh.
\end{proof}

\begin{remark}
(a) ở trên cho thấy các biểu diễn đơn của $G\times H$ chính là các tích tensor ngoài
của một biểu diễn đơn của $G$ với một biểu diễn đơn của $H$;  
(b) phát biểu tính duy nhất của cặp thừa số (lên đến đẳng cấu).  
Trên $K$ đóng đại số, điều này tương đương với mô tả \emph{mọi} biểu diễn đơn của $G\times H$.
\end{remark}
