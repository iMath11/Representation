% Abstract
\begin{abstract}
We develop a coherent toolkit across linear algebra and module theory. 
On the linear side, we show that $\{u_i\otimes v_j\}$ is a basis of $U\otimes_KV$ and prove the exterior-power trace identity 
$\det(\Id_V+f)=\sum_{r\ge0}\Tr(\wedge^r f)$, identifying $\Tr(\wedge^r f)$ with the $r$-th elementary symmetric function of the eigenvalues, hence with coefficients of the characteristic polynomial. 
On the commutative algebra side, we verify exactness of localization on short exact sequences; compute $\Hom$ over $\mathbb{Z}$ for finite and free parts, including 
$\Hom_{\mathbb{Z}}(\mathbb{Z},G)\cong G$, $\End_{\mathbb{Z}}(\mathbb{Z}/p^n)\cong \mathbb{Z}/p^n$, and 
$\Hom_{\mathbb{Z}}(M,N)$ for finitely generated modules via the structure theorem. 
For tensor products we establish biadditivity 
$(M\oplus N)\otimes P\cong M\otimes P\oplus N\otimes P$, compute torsion interactions 
($G\otimes H=0$ for coprime orders; $\mathbb{Z}/p^m\otimes\mathbb{Z}/p^n\cong \mathbb{Z}/p^{\min(m,n)}$), and derive a complete formula for $M\otimes_{\mathbb{Z}}N$.

In the theory of projective and injective modules, we prove that direct summands of projectives are projective; 
exhibit $\mathbb{Z}/m$ as a projective but nonfree $\mathbb{Z}/(mn)$-module when $\gcd(m,n)=1$; 
give dual-basis characterizations of (finite) projectives: $x=\sum f_i(x)x_i$ and 
$f=\sum f(x_i)f_i$ for suitable $x_i,f_i$. 
For injectivity, we show the field of fractions $K$ of a domain $R$ is divisible (hence injective over a PID), 
and present Baer's criterion in concrete form: $M$ is injective iff every $f:L\to M$ from a left ideal 
is left multiplication by some $a\in M$.

For finite-length decomposition, we apply Fitting's lemma and the locality of endomorphism rings of indecomposables to prove Krull--Schmidt: every Artinian--Noetherian module decomposes uniquely (up to isomorphism and order) into indecomposables, yielding cancellation and, in particular, the classical primary cyclic decomposition of finite abelian groups.

\textit{Keywords.} \textit{Modules, Ext, Tor, Homology groups}.

\end{abstract}